\1 Similar to Camus idea of futulity, is in Ecclesiastes, ``Everything is Meaningless'' by King Solomon
\2 He holds that all ideas are old, simply being recreated, such that the act of working to progress is a paradox in and of itself
\3 Knowledge itself is recreated, and then forgotten by future generations, in a circular flow, almost as conservation of knowledge
\2 Meaningless itself can be thought to mean futile, but also vapor in the literal form, such that spiritual 
\2 He also stated that on an individual level, wisdom and knowledge is meaningless, due to death being inevitable in the long run
\1 Solomon states in ``A Time for Everything'' that in the long run, God has created everything, such that nothing more can be done or improved
\2 Thus, the long run must be ignored, and people must find joy with their toil, and in the pleasures of life themsleves
\2 Religion itself, as well as any rote, mechanical task, is pointless without true spiritual understanding and enjoyment
\1 The sun is a recurring motif in both Camus and Solomon, and can be thought to give the advantage to the strong, making it harder to hide and easier to see your enemies, bearing down on the weak at work, while the strong is exempt


\1 In the first scene as Estragon and Vladimir are introduced, while the former plays with his boots, the latter with his hat, as a symbol for the physical and metaphysical, or the theoretical and the physical worlds, down-to-earth blue-collar and head-in-the-clouds academic
\2 This is noted when they discuss the Bible, such that he observes the maps, rather than the implications and morals of the text
\2 It has also been thought that Estragon is the body, Vladimir is the mind, such that they must get along, or have severe problems
\1 While they wait for Godot, Vladimir wishes for their salvation, willing to wait and give up his rights, to be tied down to Godot, such that with the absurdist critique in Neitsche of the metaphysical, it seems likely he is being fooled
\2 Godot, tying them down, is analogous to God, while the characters are analogous to the two theives, one saved, one fooled, such that it is likely that Vladimir, tricked by Godot, is not saved
\2 Shortly after, Lucky and Potto, the former tied down subserviently to the latter, are introduced to them, analogous to their having been tied to Godot
\1 When Potto talks especially, there is no true conversation, nothing getting done, such that language is not effective, unable to properly communicate or get a conversation, but rather get sidetracked and distracted constantly
\2 Moreso, when Estrogen states he is leaving multiple times, he doesn't actually leave, such that none of the characters follow through on their actions
\2 This implies that Godot intended to tie down the reader, unable to get a full thought through, such that they are trapped without reason
