\documentclass[11 pt, twoside]{article}
\usepackage{textcomp}
\usepackage[margin=1in]{geometry}
\usepackage[utf8]{inputenc}
\usepackage{color}
%\usepackage{indentfirst} %Comment out for no first paragraph indent
\usepackage[parfill]{parskip}
\usepackage{setspace}
\usepackage{tikz}
\usepackage{amsmath}
\usepackage{amsfonts}
\usepackage{amssymb}
\usepackage{outlines}
\usepackage{enumitem}

\usepackage{fancyhdr}
\pagestyle{fancy}
\cfoot{\hyperlink{content}{\thepage}}
\lhead{}
\chead{}
\rfoot{}
\lfoot{}
\rhead{}
\renewcommand{\headrulewidth}{0pt}
\renewcommand{\footrulewidth}{0pt}

\usepackage{hyperref}
\hypersetup {
	colorlinks,
	citecolor=black,
	filecolor=black,
	linkcolor=black,
	urlcolor=black
}

\newcommand{\sepitem}{0pt}
\newcommand{\seppar}{1pt}

\setenumerate[1]{itemsep=\sepitem, parsep=\seppar}
\setenumerate[2]{itemsep=\sepitem, parsep=\seppar}
\setenumerate[3]{itemsep=\sepitem, parsep=\seppar}
\setenumerate[4]{itemsep=\sepitem, parsep=\seppar}

\newenvironment{outline*}
{
	\begin{outline}[enumerate]
	}
	{\end{outline}
}

\begin{document}

\title{Intro to Microeconomics}
\author{Avery Karlin}
\date{Spring 2016}
\newcommand{\textbook}{Krugman's Microeconomics for AP}
\newcommand{\teacher}{Schweitzer}

\maketitle
\newpage
\hypertarget{content}{\tableofcontents}
\vspace{11pt}
\noindent
\underline{Primary Textbook}: \textbook\\
\underline{Teacher}: \teacher
\newpage

\section{Chapter 2 - Supply and Demand}
\underline{Note:} Only includes Modules 8 and 9

\subsection{Price Controls}
\begin{outline*}
\1 Governments can intervene in the market for the benefit of sellers or buyers, based on moral or political arguments, placing price controls in the form of a ceiling or floor
\2 In inefficient markets, price controls can often not hurt the market efficiency, but can rather improve it, such as the housing market in real life
\2 In any market where factors other than price play a role, or where there is not complete product information, the market is not perfectly competitive, such that the model doesn't completely apply
\1 Price ceilings are generally used during major shortages, such as wars or natural disasters, that hurt the general public, on essential raw materials
\2 In an efficient market, price ceilings can force a price below equilibrium, leading to a shortage, though if above, will have no effect
\2 Ceilings can lead to inefficient allocation to consumers, with those with less urgent needs gaining the resources, due to those willing to pay a higher price being unable, instead often leading to luck or personal connections determining who gains the products
\2 Subletting is the process of illegally renting an apartment being rented for a price above the rent control, investigated commonly in NY, preventing it from creating efficiency in allocation, and leading to black/illegal markets, creating disrespect for the law and hurting those unwilling to violate the law
\2 Ceilings also lead to wasted resources, due to spending additional effort attempting to compensate for the shortaes, such as searching for opportunities to purchase the good, creating production inefficiency as well
\2 They also lead to ineffiency by low quality due to the lack of incentive to provide better products without the ability to increase prices
\1 Rent controls, such as in NYC, originated due to preventing raw material prices from going up too high due to less supply because of being used in the war, lowering the demand for the materials to build new apartments, extended to other buildings and modified later on
\2 Low quality is especially incentivized in rent control, due to landlords trying to eliminate tenants to allow rent to be raised
\1 Ceilings are created often due to vocal support from the minority that benefits, belief that the market doesn't follow the model, lack of economic understanding by government, and more extreme ideas about how high prices would rise otherwise, due to black market prices often being above unregulated prices
\1 Price floors have been used to support the workers, such as on agricultural products for farmers or the minimum wage for unskilled workers, leading to a surplus
\2 In some cases, such as agricultural, the government buys the surplus, leading to government warehouses of unwanted goods, often paying exporters to remove the product for a loss (in the EU) or using it in schools (in the US)
\2 The US often also pays farmers to not produce the product above a certain level, to prevent a surplus, but in the case of minimum wage, leads to unemployment
\2 Low quality is especially incentivized in rent control, due to landlords trying to eliminate tenants to allow rent to be raised
\1 Price floors raises prices such that consumers purchase less than the equlibrium quantity, and create inefficient allocation of sales among sellers due to sellers willing to accept lower prices for their goods, rather than not be able to sell it
\2 It would create wasted resources due surplus not being used, incentivize illegal activity, such as black labor (off the books to avoid regulation)
\2 It also creates inefficiently high quality, forcing sellers to produce quality justifying the price, when consumers would rather a lower price with only the desired quality
\end{outline*}

\subsection{Quantity Controls}
\begin{outline*}
\1 Quantity controls/quotas are the regulation of a quantity that can be bought or sold, generally by limited numbers of licenses to supply a good, or through an added tax on a specific good
\1 Limiting the quantity supplied below the equlibrium results in a difference between the demand and supply price, or a wedge
\2 This works by creating a market for licenses, with a quota rent of the difference between the demand price paid to the renter, and the supply price desired by the renter, paid in rent to the owner
\2 If the license is not rented, the wedge is the opportunity cost of keeping the license, making up for the lack of quota rent
\2 Thus, in addition to the product income gained from the license, it is an asset by itself as a result, with supply price as the real price for the goods
\1 The main cost of quotas are inefficiency of production, due to less rides being offered than ideally provided by both, called the deadweight loss of income for production and goods for consumers, measured as the area of the triangle between the quota and equilibrium, with dollars as the unit
\2 This also incentivizes unregulated, black market transactions
\2 On the other hand, owners of the licenses have a far stronger incentive to be vocal in favor of preserving the quota, hurting revenue of owners
\1 The tax/quota burden is distrubuted based on the slopes of the curve, such that the difference in price per unit to the buyer and seller determines the distribution of the burden to each, measured in dollars per unit
\1 It must be noted that market equilibrium is assumed to be socially optimal only if all cost-benefits are accounted for and income distribution is fair, such that the market is perfectly competitive and takes into account all factors
\end{outline*}

\section{Chapter 9 - The Demand Curve}
\subsection{Income Effects, Substitution Effects, and Elasticity}
\begin{outline*}
\1 The Law of Demand is based on the Income and Substitution Effects
\1 The Substitution Effect states that as the price of a good increases, people begin buying other substitute goods instead, such that quantity demanded of the good decreases
\2 This is the main reason for goods which take a small portion of a person's income
\1 The Income Effect states that as the price of a good increases, less will have to be spent that or other goods, such that real income decreases and quantity demanded on the good can decrease
\2 This only effects goods that take up a majority of income, such as housing, working with the substitution effect
\2 Inferior goods, on the other hand, have demand increase as income decreases, such that it works against the substitution effect
\3 Giffen goods are inferior such that the Income excesses Substitution, such that demand increases as price increases, but may not exist
\1 Elasticity is the amount of change of the dependent variable from a change in the independent
\2 Price Elasticity of Demand = $\frac{\text{\% Change in Quantity Demanded}}{\text{\% Change in Price}}$, where $\% \Delta x = \frac{\Delta x}{x_i} * 100\%$, usually written as the absolute value of the elasticity by convention
\2 Since percent difference between values depends on the method of measuring, based on what is in the denominator, the midpoint method uses the average value of the variable to measure percent difference
\3 It can also be written as Price Elasticity of Demand = $\frac{dQ}{dP} * \frac{P_i}{Q_i}$ as a result
\end{outline*}
\subsection{Price Elasticity of Demand}
\begin{outline*}
\1 Perfectly elastic demand is such that if the price is raised, the quantity demanded is 0, but if lowered, is infinite
\2 Perfectly inelastic demand is an elasticity of 0, such that changes in the price result in no change in the quantity demanded
\2 If price elasticity of demand is greater than 1, it is elastic, less, it is inelastic, and if equal, then it is unit-elastic
\1 Elasticity is used to predict the change in total revenue, or the total value of sales of a product, equal to the product of the price and quantity sold, or the area from the origin to the point on the curve
\2 For unit-elastic demand as a result, increase in price results in no change in total revenue, while for inelastic, there is an increase in revenue, and elastic, a decrease
\2 This is due to the price effect of higher prices combined with the quatity effect of less sold
\1 Price elasticity often changes throughout a demand curve, such that when measured, it is used to mean the elasticity at the immediate point on the curve
\1 Price elasticity is affected by the availablility of close substitutes, if a good is a necessity or a luxary, share of income spent, and time
\2 If a large share of income is spent, elasticity tends to be far higher, due to a greater reaction to price change
\2 Over time, consumers adjust habits to a price change, even if it is difficult in the short run, such that long-run price elasticity is far higher than short-run
\end{outline*}
\subsection{Other Elasticities}
\begin{outline*}
\1 Elasticity overall is unitless, such that it does not depend on the price/quantity being related to the units, rather due to percent change ratios, to avoid confusion
\1 Cross-price elasticity of demand is the change in the quanity demanded due to the change in the price of substitutes or compliments
\2 Cross-price elasticity = $\frac{\% \Delta \text{Quantity demanded of good}}{\% \Delta \text{Price of related good}}$
\2 Substitutes have a positive cross-price elasticity, while complements have a negative, where the closer the substitution/the stronger the compliment, the greater the magnitude of the elasticity will be
\1 Income elasticity of demand is the change in the quantity demanded due to the change in income
\2 Income elasticity = $\frac{\% \Delta \text{Quantity demanded}}{\% \Delta \text{Income}}$
\2 Normal goods have positive elasticity, while inferior goods have neative
\2 Income-elastic is elasticity greater than 1, such that demand rises faster, while income-inelastic has positive, but below 1, such that demand rises slower than income
\3 Necessities are income-inelastic generally, such as good or clothing, such that the low income of producers of necessities such as farmers is the result of recent economic growth, raising other incomes
\1 Price elasticity of supply = $\frac{\% \Delta \text{Quantity supplied}}{\% \Delta \text{Price}}$
\2 Similarly, perfectly elastic supply has elasticity of $\infty$, such that below a price, 0 is supplied, above the price, infinity, and any amount is supplied at the price
\2 Perfectly inelastic supply has elasticity of 0, such that changes in the price produce no change in the quantity supplied
\2 Price elasticity of supply is determined by availablity of inputs (easier adding and removing of inputs results in higher elasticity) and time
\3 In the long run, elasticity is higher due to having more time to respond to a change in price
\3 This is also partially due to the ability to expand the availability and production of inputs, even for limited resources, building the infrastructure and technology to do so
\end{outline*}
\subsection{Consumer and Producer Surplus}
\begin{outline*}
\1 Consumer surplus is the benefit gained by buyers from being able to purchase a good, and producer surplus is the benefit gained by the sellers from being able to sell a good
\1 Consumer surplus is based on the buyer's willingness to pay, or the maximum price at which the buyer would buy the good
\2 At that price, they are indifferent about buying, willing below, and unwilling above, though it is generally assumed that it is bought at the point of indifference
\2 Individual consumer surplus is equal to the willingness to pay - the price paid, such that total consumer surplus is the sum of the individual buyers
\2 Consumer surplus can thus be gained from the area between the portion of the curve above the horizontal price line, and the horizontal price line
\1 Producer surplus is measured similarly by the area above the supply curve, under the price line
\2 This is based on the cost of each seller, or the lowest price at which they would be willing to sell the good
\end{outline*}
\subsection{Efficiency and Deadweight Loss}
\begin{outline*}
\1 Total surplus is the sum of consumer and producer surplus at any given price, proving that all parties gain from trade
\1 Market efficiency (of allocation) is such that there is no way to make some better off without hurting others, possibly able to be created by reallocating consumption among consumers, sales among sellers, or quantity traded
\2 Reallocation to consumers willing to trade less than the equilibrium price reduces consumer surplus by taking it from those who need it more, creating negative consumer surplus for those buyers, reducing total
\2 Reallocation of sales to those who would be unwilling to sell above market equilibrium results in a loss of producer surplus, creating negative producer surplus for those sellers, reducing total
\2 Changing the quantity to less prevents transactions that would increase surplus, while more creates forced transactions, such that there is negative surplus
\1 Market equilibrium thus allocates consumption to those who value the good the most and sales to those who value it the least
\2 It also makes it so that each buyer gains it from someone who values the good less than they do (mutually beneficial)
\2 It also makes it so there are no buyers who can't purchase the good who value it more than any sellers who can't sell it (no missed transactions)
\2 On the other hand, while this maximizes efficiency, it does not maximize equity (fairness), does not make each participant the happiest, just the total, and has the possibility of the market failing
\1 Equity is based on personal values, and helps determine taxation, either progressive, regressive, or proportional (same percentage for all incomes)
\2 In a perfectly competitive market, there is a trade-off between economic efficiency and equity, in the form of redistributive taxes and services
\1 Excise taxes, or taxes on each unit sold, are able to be levied on either the producer or the consumer, but the tax incidence, or the measure of how the tax burden is divided, is the same regardless for a given tax
\2 Excise taxes for equally elastic supply and demand curves are divided evenly, while for unequal curves, it falls more on the more inelastic curve
\2 The benefit from the tax is the revenue, or the area of the rectangle made by the wedge and quantity sold, and the services provided
\2 While the revenue is to provide services back to the taxpayers such that it is not a cost, the cost is the inefficiency caused by the tax, by the loss of mutually beneficial transactions/deadweight loss
\3 The loss in producer and consumer surplus is equal to the sum of the revenue and the deadweight loss
\3 The larger the wedge, the greater the inefficiency created by the tax by the loss
\3 There is an added cost of administrative costs of the tax, meaning the resources used to collect and pay, avoid, or evade the tax
\1 Taxes have different amounts of equity, generally a trade-off, such as a lump-sum tax, which is the same on all people, such as a poll tax, with high efficiency, creating little deadweight loss, but low equity
\end{outline*}
\subsection{Utility Maximization}
\begin{outline*}
\1 Utility is the measure of satisfaction individuals gain from consumption, which they can be assumed to attempt to maximize production of through consumption
\2 Utility is measured in utils, where a utility function shows the relationship between utility (y-axis) and the consumption bundle consumed (x-axis)
\2 Marginal utility is the change in total utility based on additional consumption of one more unit of the good, plotted in the midpoint on the x-axis between the two quantities
\1 The principle of diminishing marginal utility states that the marginal utility curve is generally downward sloping, such that there is eventually no gain in utility from additional consumption, and may be a loss
\2 This is the final rule, along with the substitution and income effect, to explain the shape of the demand curve
\1 Budget constraints also limit consumers, such that the total set of consumption bundles under constraints are the consumption possibilities
\2 The budget line is the set of affordable consumption bundles of two items that cost exactly the income
\1 The optimal consumption bundle is that which is both affordable and maximizes utility
\2 Marginal utility per dollar is the change in total utility based on additional consumption of one more dollar of the good, calculated at the midpoint, equal to the marginal utility divided by the price for a unit
\2 The optimal consumption rule states that the maximum total utility is at the point where the marginal utility per dollar is equal for each good
\2 As a result, increase in price leads to decrease in marginal utility per dollar, such that less is demanded for the optimal consumption bundle
\end{outline*}
\section{Chapter 10 - The Supply Curve}
\underline{Note:} Does not include Modules 56 and 57
\subsection{Definition of Profit}
\begin{outline*}
\1 The supply curve is based on the assumption that the goal of firms is to maximize profit, or total revenue minus total cost
\1 Explicit costs are those requiring payment of money, while implicit costs are measured in terms of dollars, but based on the value of benefits lost, such as possible income
\2 Both types must be taken into account, rather than just explicit costs, often less than implicit
\2 Implicit can also be thought of as the cost of expending ones own resources on the activity, such as labor time
\1 Accounting profit = total revenue - explicit cost - deprecation of capital, generally reported to the IRS and investors
\1 Economic profit = accounting profit - implicit cost, used to determine whether the resources could be better used elsewhere and the implied form of profit in economics
\2 If capital is not owned but rented, it is an explicit cost, but there is no deprecation, while if owned, there is also the implicit cost of capital, or the income that could have been generated by the capital (physical or finanical used to purchase it) if used elsewhere
\2 Positive economic profit means resources are being used the most productive way, while negative means there are more productive alternatives
\2 Normal profit, or economic profit of 0, means that it is the best use of resources, equal to the alternative in terms of profit, acting as a natural profit amount
\3 Actual profit is equal to the economic profit plus the normal profit
\3 Normal profit can also be thought of as the revenue from entrepeneurialism as one of the major resources
\end{outline*}
\subsection{Profit Maximization}
\begin{outline*}
\1 After determining whether the enterprise is economically profitable by making at least normal profit, it must be determined the quantity of production that will maximize profit
\1 The principle of marginal analysis states that the ideal quantity is when the marginal benefit is equal to the marginal cost
\2 In this case the optimal output rule, balancing marginal revenue and marginal cost
\1 The marginal cost curve and marginal revenue curve can be drawn, the latter as a flat line equal to the price in perfectly competitive markets, the former generally as a vaguely parabolic curve, going down then rising
\end{outline*}
\subsection{Production Function}
\begin{outline*}
\1 Fixed inputs are those whose quantity cannot be changed during the short-run, such as land or capital, while variable are those able to varied, such as labor
\2 All inputs are assumed to be able to be changed in the long-run
\1 The total product curve measures the total amount of a good produced based on the quantity of a variable input, where all other are assumed to be fixed
\2 The marginal product of input = $\frac{\Delta \text{Quantity}}{\Delta \text{Input}}$, taken between adding a unit of the input
\2 Diminishing returns to an input is generally true, such that as more input is added, the marginal product of the input decreases
\1 Increasing the fixed input generally causes both the marginal product of input and the total product curves to shift upward at all points except 0
\end{outline*}
\subsection{Firm Costs}
\begin{outline*}
\1 Resource costs are broken up into fixed/overhead costs, not depending on the amount produced, and variable costs, added to get the total cost, combined explicit and implicit from giving up the rental income
\2 Total cost curve is the total cost on the y-axis and the quantity produced on the x-axis, getting steeper as it progresses due to diminishing returns for added variable input (increasing marginal cost)
\2 Marginal cost = $\frac{\Delta \text{Total Cost}}{\Delta \text{Quantity Produced}}$
\1 Average total cost = $\frac{\text{Total Cost}}{\text{Quantity Produced}}$
\2 Average costs are measured at each quantity point, unlike marginal cost which is measured between quantity points
\2 Average variable cost is measured similarly for only the variable input cost
\1 Average total cost is said to decrease at first, and then increase, as quantity increases, such that it is U-shaped
\2 This is due to the spreading effect (larger quantity to spread over the fixed cost) along with the diminishing returns (from adding variable inputs as quantity increases)
\3 Since diminishing returns gains strength as it increases, and spreading decreases strength due to the overall change in quantity getting smaller, this causes the shape
\2 The minimum average cost is the point where the average variable cost equals the average fixed cost, when the effects offset equally
\3 The amount of output produced at this point is the minimum-cost output
\3 Outputs below the minimum cost output have marginal less than average total, while above the minimum cost output have marginal greater than average total
\1 Marginal cost curves often don't purely rise, but rather initial fall as production increases until some point where they start to rise
\2 This is due to specialization of labor and division of tasks as the production and number of workers increases at first
\2 On the other hand, this occurs before the minimum point of the total cost or variable cost curve, such that the model holds true
\end{outline*}
\section{Chapter 11 - Perfect Competition and Monopolies}
\subsection{Intro to Perfect Competition}
\begin{outline*}
\1 The needed conditions for perfect competition are many producers, each with a small market share and a standardized product
\1 In a price-taking firm, by the optimal output rule, the marginal revenue is equal to the market price, such that the price is the marginal cost at the optimal quantity of output
\2 This is due to being in a perfect market, the price is external, unable to be influenced by a single firm by changing quantity sold, such that the firm is called price taking
\1 The marginal revenue and marginal cost curves can be thought as supply and demand curves for the firm, such that consumer surplus is positive profit, producer surplus is negative profit
\2 Thus, in a perfectly competitive market, the firm demand is perfectly inelastic
\1 Firms only produce a particular good to begin with if the economic profit is greater or equal to 0, such that price has to be greater or equal than average total cost to be profitable
\2 If this is the case, they would produce until the optimal output point, which may not be the minimum average total cost point, unless average total cost is at the marginal revenue curve, such that profit is normal
\end{outline*}
\subsection{Graphing Perfect Competition}
\begin{outline*}
\1 Profit = (Price - Average Total Cost) * Quantity, where the Quantity is determined by the optimal output rule, and will be after minimum total average cost if there is a profit, at if there is normal, before if a loss
\2 The break-even price is the price where the firm gets normal profit, such that average total cost is equal to the price at the quantity (such that it is also equal to marginal cost)
\end{outline*}
\section{Joseph Stiglitz - Global Economy and Inequality, 2/12/16}
\begin{outline*}
\1 The growth of the disparity of income inequality has grown into a large problem within recent years, with 8 people having the same amount as the bottom 44\% of the American population
\2 A recent report estimated that worldwide, a group of 88 people (shrinking since to 66 and continuing to shrink), would have the same wealth as the bottom 50\% of the worldwide population (3B people)
\2 In the US and nations which follow the US model, economic growth has led to almost no increase in wages for the middle class, leading to the hollowing middle class as average incomes are stagnant over the last 40 years
\3 Thus, groups such as median full-time male worker wages have decreased over the last 40 years, and unskilled workers have decreased from 60 years ago wages
\2 Worker productivity itself has also drastically increased, while compensation has not, resulting in a decrease in real wages over time
\2 Globally, OECD (advanced countries) countries have all seen different degrees of increase, but show that inequality is a problem worldwide, though some countries have been able to generally offset these changes
\1 There is a systematic relationship between income and opportunity inequality, contrary to popular American belief, regardless of the occasional story, rather discussing with opportunity inequality, the probability of moving from the bottom to top of wealth
\2 Opportunity inequality is also measured as the coorelation between parental and child income
\2 The US has far greater opportunity inequality than most other advanced nations by either measure, such that economic forces are not the cause of the inequality, due to existing in each nation, rather due to government policies
\1 As a result of this relationship, it can be assumed that decreasing income inequality would allow the decreasing of opportunity inequality, which in turn has been shown to allow more economic growth
\1 Globally, the change in income over the past 20 years with respect to the percentile of global income distribution show a drastic increase among the 1\%, as well as an increase among the 20\% to 70\%, more so as the percentile rises
\2 On the other hand, the bottom 10\% and the 80\% to 97\%, which generally includes the middle class of advanced nations, has grown the least
\2 The increase in the lower percentiles is attributed to the large increase in industrializing markets, such as India or China
\1 The idea of trickle-down economics stated that the increase in wealth to the top by governmental policies would trickle-down to the lower classes, but was never proven by theory or evidence, and has been shown not to work in America
\2 Many policies by the GOP and the Obama administration through the bail-out of the banks, while qualitative easing by the Fed has attempted to increase stock market prices, but has mainly benefited the top as well
\2 The ability of money to influence the political process by Citizens United created political inequality, translating to the creation of further economic inequality, leading to a cycle of inequality
\1 Kiznet's Law was the idea that inequality would grow immediately during industrialization and growth, around the World Wars, believing it would come down over time as the markets adapted
\2 Since 1980 on the other hand, this has been shown to be false as inequality drastically increased nationally for a reason under debate
\2 One possible theory is the idea of supply-side economics, providing incentives businesses and the rich to expand, and removing restrictions to provide a more free market to cause the economy to grow faster, resulted in this economic inequality
\2 This resulted in lower tax rates on all brackets, especially the wealthy, ignored the fact that during the 1950s, the greatest period of economic growth of the century, had the highest tax rates, with 90\% on the most wealthy
\2 In addition, a more free market economy without regulations doesn't result in an increase of productivity, but rather results in speculative investment, lowering real investment, moving toward monopolies and non-productive assets
\2 The lack of regulations also resulted in fewer anti-trust regulations and enforcement, leading to the growth of monopolies
\1 The economy overall has moved to the short-term, rather than investing in long-term economic growth, working to create the highest immediate gains within the next quarter, due to this move toward supply-side economics
\2 Higher taxes would allow for a greater increase in education and infrastructure needed for fast long-term economic growth, as well as preventing income inequality
\1 Okuns Law, stating that income/opportunity inequaliy and economic performance are trade-offs, has also been shown to be false, due to less spending by the rich, preventing optimal economic growth
\end{outline*}
\end{document}
