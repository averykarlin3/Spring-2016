\documentclass[11 pt, twoside]{article}
\usepackage{textcomp}
\usepackage[margin=1in]{geometry}
\usepackage[utf8]{inputenc}
\usepackage{color}
%\usepackage{indentfirst} %Comment out for no first paragraph indent
\usepackage[parfill]{parskip}
\usepackage{setspace}
\usepackage{tikz}
\usepackage{amsmath}
\usepackage{amsfonts}
\usepackage{amssymb}
\usepackage{outlines}
\usepackage{enumitem}

\usepackage{fancyhdr}
\pagestyle{fancy}
\cfoot{\hyperlink{content}{\thepage}}
\lhead{}
\chead{}
\rfoot{}
\lfoot{}
\rhead{}
\renewcommand{\headrulewidth}{0pt}
\renewcommand{\footrulewidth}{0pt}

\usepackage{hyperref}
\hypersetup {
	colorlinks,
	citecolor=black,
	filecolor=black,
	linkcolor=black,
	urlcolor=black
}

\newcommand{\sepitem}{0pt}
\newcommand{\seppar}{1pt}

\setenumerate[1]{itemsep=\sepitem, parsep=\seppar}
\setenumerate[2]{itemsep=\sepitem, parsep=\seppar}
\setenumerate[3]{itemsep=\sepitem, parsep=\seppar}
\setenumerate[4]{itemsep=\sepitem, parsep=\seppar}

\newenvironment{outline*}
{
	\begin{outline}[enumerate]
	}
	{\end{outline}
}

\begin{document}

\title{Microeconomics}
\author{Avery Karlin}
\date{Spring 2016}
\newcommand{\textbook}{Krugman's Microeconomics for AP}
\newcommand{\teacher}{Schweitzer}

\maketitle
\newpage
\hypertarget{content}{\tableofcontents}
\vspace{11pt}
\noindent
\underline{Primary Textbook}: \textbook\\
\underline{Teacher}: \teacher
\newpage

\section{Chapter 2 - Supply and Demand}

\subsection{Price Controls}
\begin{outline*}
\1 Governments can intervene in the market for the benefit of sellers or buyers, based on moral or political arguments, placing price controls in the form of a ceiling or floor
\2 In inefficient markets, price controls can often not hurt the market efficiency, but can rather improve it, such as the housing market in real life
\2 In any market where factors other than price play a role, or where there is not complete product information, the market is not perfectly competitive, such that the model doesn't completely apply
\1 Price ceilings are generally used during major shortages, such as wars or natural disasters, that hurt the general public, on essential raw materials
\2 In an efficient market, price ceilings can force a price below equilibrium, leading to a shortage, though if above, will have no effect
\2 Ceilings can lead to inefficient allocation to consumers, with those with less urgent needs gaining the resources, due to those willing to pay a higher price being unable, instead often leading to luck or personal connections determining who gains the products
\2 Subletting is the process of illegally renting an apartment being rented for a price above the rent control, investigated commonly in NY, preventing it from creating efficiency in allocation, and leading to black/illegal markets, creating disrespect for the law and hurting those unwilling to violate the law
\2 Ceilings also lead to wasted resources, due to spending additional effort attempting to compensate for the shortaes, such as searching for opportunities to purchase the good, creating production inefficiency as well
\2 They also lead to ineffiency by low quality due to the lack of incentive to provide better products without the ability to increase prices
\1 Rent controls, such as in NYC, originated due to preventing raw material prices from going up too high due to less supply because of being used in the war, lowering the demand for the materials to build new apartments, extended to other buildings and modified later on
\2 Low quality is especially incentivized in rent control, due to landlords trying to eliminate tenants to allow rent to be raised
\1 Ceilings are created often due to vocal support from the minority that benefits, belief that the market doesn't follow the model, lack of economic understanding by government, and more extreme ideas about how high prices would rise otherwise, due to black market prices often being above unregulated prices
\1 Price floors have been used to support the workers, such as on agricultural products for farmers or the minimum wage for unskilled workers, leading to a surplus
\2 In some cases, such as agricultural, the government buys the surplus, leading to government warehouses of unwanted goods, often paying exporters to remove the product for a loss (in the EU) or using it in schools (in the US)
\2 The US often also pays farmers to not produce the product above a certain level, to prevent a surplus, but in the case of minimum wage, leads to unemployment
\2 Low quality is especially incentivized in rent control, due to landlords trying to eliminate tenants to allow rent to be raised
\1 Price floors raises prices such that consumers purchase less than the equlibrium quantity, and create inefficient allocation of sales among sellers due to sellers willing to accept lower prices for their goods, rather than not be able to sell it
\2 It would create wasted resources due surplus not being used, incentivize illegal activity, such as black labor (off the books to avoid regulation)
\2 It also creates inefficiently high quality, forcing sellers to produce quality justifying the price, when consumers would rather a lower price with only the desired quality
\end{outline*}

\subsection{Quantity Controls}
\begin{outline*}
\1 Quantity controls/quotas are the regulation of a quantity that can be bought or sold, generally by limited numbers of licenses to supply a good, or through an added tax on a specific good
\1 Limiting the quantity supplied below the equlibrium results in a difference between the demand and supply price, or a wedge
\2 This works by creating a market for licenses, with a quota rent of the difference between the demand price paid to the renter, and the supply price desired by the renter, paid in rent to the owner
\2 If the license is not rented, the wedge is the opportunity cost of keeping the license, making up for the lack of quota rent
\2 Thus, in addition to the product income gained from the license, it is an asset by itself as a result, with supply price as the real price for the goods
\1 The main cost of quotas are inefficiency of production, due to less rides being offered than ideally provided by both, called the deadweight loss of income for production and goods for consumers
\2 This also incentivizes unregulated, black market transactions
\2 On the other hand, owners of the licenses have a far stronger incentive to be vocal in favor of preserving the quota, hurting revenue of owners
\end{outline*}

\section{Chapter 9 - The Demand Curve}

\end{document}
