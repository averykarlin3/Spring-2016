\documentclass[11 pt, twoside]{article}
\usepackage{textcomp}
\usepackage[margin=1in]{geometry}
\usepackage[utf8]{inputenc}
\usepackage{color}
\usepackage{indentfirst} %Comment out for no first paragraph indent
\usepackage[parfill]{parskip}
\usepackage{setspace}
\usepackage{tikz}
\usepackage{amsmath}
\usepackage{amsfonts}
\usepackage{amssymb}
\usepackage{enumitem}
\usepackage{outlines}

\usepackage{fancyhdr}
\pagestyle{fancy}
\cfoot{\hyperlink{content}{\thepage}}
\lhead{}
\chead{}
\rfoot{}
\lfoot{}
\rhead{}
\renewcommand{\headrulewidth}{0pt}
\renewcommand{\footrulewidth}{0pt}

\usepackage{hyperref}
\hypersetup {
	colorlinks,
	filecolor=black,
	linkcolor=black,
	urlcolor=black
}

\newcommand{\sepitem}{0pt} %Added room between items on the list, not including a list and its sublist
\newcommand{\seppar}{1pt} %Between items and lists overall

\setenumerate[1]{itemsep=\sepitem, parsep=\seppar}
\setenumerate[2]{itemsep=\sepitem, parsep=\seppar}
\setenumerate[3]{itemsep=\sepitem, parsep=\seppar}
\setenumerate[4]{itemsep=\sepitem, parsep=\seppar}

\newenvironment{outline*}
{
	\begin{outline}[enumerate]
	}
	{\end{outline}
}

\newcommand{\foota}[1]{\hyperlink{#1}{$_#1$}}
\newcommand{\footb}[2]{\hyperlink{#1}{$_{#1.#2}$}}

\begin{document}

\title{Existentialism}
\author{Avery Karlin}
\date{Spring 2016}
%\newcommand{\textbook}{}
\newcommand{\teacher}{Mazzurco}

\maketitle
\newpage
\hypertarget{content}{\tableofcontents}
\vspace{11pt}
\noindent
%\underline{Primary Textbook}: \textbook\\
\underline{Teacher}: \teacher
\newpage

\section{The Problem of Other People}
\subsection{Hyperbolic Doubt}
\begin{outline*}
\1
\end{outline*}
\subsection{The Look}
\begin{outline*}
\1 Arthur Aron's study emphasizes that intimacy and trust are needed in a relationship, and can be created by stimuli, and the look facilitates this idea of trust, forced by the questions, so that the other person can properly understand about their partner, without a facade\foota{2}
\2 The goal of a relationship is to form a proper understanding, rather than a two-dimensional image, combining how they see themselves with how you percieve them
\3 It is also noted that to form a relationship based on this, one has to be looked for, providing the willingness, which is one of the reasons the experiments works, by taking those willing, such that they wouldn't close off
\2 It is especially emphasized by the author that the interesting thing is not to look into the eyes of the other, but to be seen
\2 In addition, the emphasis on sharing aspects that are liked creates an atmosphere of absorbing without reflection, forced by the situation, but then a confirmation of what the other actually thinks, emphasized further by sharing three things they agree on
\end{outline*}
\subsection{Photography}
\begin{outline*}
\1
\end{outline*}
\subsection{Solitary Confinement}
\begin{outline*}
\1 The effects of solitary confinement are shown to be hallucinating voices, talking to themselves \foota{1}, due to long term lack of human contact, with even automated systems to provide food \foota{1}
\2 Senses began to become redundant, due to the unchanging light \foota{1} and grey walls \foota{1}, making time and sight pointless, with the exception of being able to leave for an hour each day to exercise \foota{1}
\3 There is nothing to think about other than yourself, but no objective standard to observe based on, such that you lose the reason for existance, and lose your ability to define yourself or those around you
\3 Thus, people often cling to minute details of the world around \foota{1}
\3 Desperately working to preserve senses \foota{1}, such that on some level, he must believe it is real
\2 There is no method of protecting himself or confirming any information, at the mercy of the guards \foota{1}, such that he only trusts things he and others can see
\2 Lack of the feeling of existance, along with the space itself, due to not being percieved by anyone other than himself \foota{1}
\3 There are no necissary actions, such that there is no reason for existing, and no way of affecting anybody outside of the cell, such that they are fully forgotten \foota{1}
\4 This leads to experiments where isolated infants eventually stop eating and starve themselves, and relates to the experiment that humans are evolutionarily strengthened through society, and are not able to survive otherwise
\3 Humans rely on others to confirm the existance of phenomenon, such that if nobody else can see something, there is no proof it exists in his eyes
\1 This is related to the feeling that time moves faster during faster music and slower during slower music, such that time is observed in relation to other factors, rather than objective
\end{outline*}

\subsection{Eternal Awakeness}
\begin{outline*}
\1 Before Inez's entrance, Garcin emphasized both the fear of neverending awareness, but also having to live with his own thoughts \footb{5}{4}
\2 Sleeping is a way to escape, as well as blinking, turning off sight of the world \footb{5}{4}
\2 Harsh light that is always present, never letting one forget they have work to do, forced to remain aware \footb{5}{3}
\1 Life without breaks is similar to solitary confinement, without a proper passage of time, nonstop listening to ones own thoughts
\2 The removal of mirrors removes the last attempts to percieve other people, and prevents awareness of ones self as an object
\2 Lack of eyelids prevents the ability to prevent oneself from being observed, such that only others can observe
\1 **GET NOTES FROM MONDAY** 
\1 Inez's main function appears to be to irritate and create dissent to the point of insanity, creating dissent and provoking insecurity of other characters \footb{5}{22}
\2 She is also the only realistic one, willing to confront the reality of being in hell, used by Sartre to show his point of view  \footb{5}{26}
\1 The end is the basis of Sartre's idea of existance presiding essence, such that something must exist before it is thought of, rather than theorizing before executing \footb{5}{25}
\2 Sartre argues that ``a priori'' is not true of humans, and the definition of self is created by ourselves through actions, rather than inherant from birth
\2 Garcin dreaming of being a hero doesn't make him a hero, since in his final action, he fled from real danger and became a coward
\2 His actions don't back up his words, so while people can make themselves by willpower, it is based on action purely, not words
\2 Thus, positive intentions are not valid, and emotional responses prevnting actions is something to be purely overcome
\1 Near the end, Garcin and Inez acknoledge that everybody is watching and the idea of a crowd, and their thoughts, breaking the first wall \footb{5}{26}
\2 ``Hell is other people'' is the idea of fear of being mocked, judged by others for eternity for their actions, as Garcin is for being with Estelle and his cowardice by Inez
\2 Garcin is metaphoric for the French, not fighting the Nazi regime, while Inez can be interpreted as the regime not allowing them to live their lives without judgement, or just the resistance critisizing their lack of action
\end{outline*}
\section{The Problem of Nothingness}
\subsection{Freud's Levels of Conciousness}
\begin{outline*}
\1 The Id is the primal part of the brain, containing the basic desires, located in the subconcious mind
\1 The Superego is the part of the mind which learns from our environment, within the subconcious mind, taking in rules and morals
\1 The Ego is the concious mind, acting to balance the desires of the Id and the values of the Superego
\end{outline*}
\subsection{Bad Faith}
\begin{outline*}
\1 Bad faith is a form of self-desception, when we view ourself as an object without responsibility and free will, to escape the budens of free will
\2 Sartre emphasizes that this is an active action, rather than a state
\1 The structure of a lie doesn't make sense, since the liar and the lied to cannot both fool and be fooled when they are the same
\1 Freud believed that certain drives are determined to be too dangerous to even be recognized by the ego, rather repressed and redirected in some other form of action
\2 This is the reason psychological patients give resistance as it moves closer to the repressed material, as the ego attempts to defend the repression
\2 This recreates the paradox of self-deception, as the ego both does not know about the lie, and must keep the lie repressed within the id
\1 Sartre countered this by three patterns of bad faith, deferring the moment of decision, molding onesself to society, and becoming an object \footb{7}{55}
\2 Thus, the first pattern is simply ignoring the urgency of the situation, ignoring implications in favor of objectivity, observing only what is present, similar to Estelle \footb{5}{9}
\2 The second pattern is the seperation of the object and subject, or the mind and body, passively allowing circumstances around them from society to control themselves
\2 The third pattern is taking on a role, such that oneself is the Other/object, but lose a self-image distinct from the role by which the world sees you, fully working on perfecting this role
\end{outline*}
\section{Readings}
\begin{enumerate}
\hypertarget{1}{\item ``The Concrete Abyss'' by Lisa Guenther}
\hypertarget{2}{\item ``To Fall in Love with Anyone, Do This'' by Mandy Len Catron}
\hypertarget{3}{\item ``Action Philosopher \#133 - Rene Descartes'' by Fred Van Lente}
\hypertarget{4}{\item ``NY Rethinks Solitary Confinement'' by NYT Editorial Board; Feb 20th, 2014}
\hypertarget{5}{\item ``No Exit'' by Jean-Paul Sartre, Stored PDF}
\hypertarget{6}{\item ``The Flies'' by Jean-Paul Sartre, Stored PDF}
\hypertarget{7}{\item ``Being and Nothingness'' by Jean-Paul Sartre, Stored PDF}
\end{enumerate}

\end{document}
