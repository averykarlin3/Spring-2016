\documentclass[11 pt, twoside]{article}
\usepackage{textcomp}
\usepackage[margin=1in]{geometry}
\usepackage[utf8]{inputenc}
\usepackage{color}
\usepackage{indentfirst} %Comment out for no first paragraph indent
\usepackage[parfill]{parskip}
\usepackage{setspace}
\usepackage{tikz}
\usepackage{amsmath}
\usepackage{amsfonts}
\usepackage{amssymb}
\usepackage{enumitem}
\usepackage{outlines}

\usepackage{fancyhdr}
\pagestyle{fancy}
\cfoot{\hyperlink{content}{\thepage}}
\lhead{}
\chead{}
\rfoot{}
\lfoot{}
\rhead{}
\renewcommand{\headrulewidth}{0pt}
\renewcommand{\footrulewidth}{0pt}

\usepackage{hyperref}
\hypersetup {
	colorlinks,
	filecolor=black,
	linkcolor=black,
	urlcolor=black
}

\newcommand{\sepitem}{0pt} %Added room between items on the list, not including a list and its sublist
\newcommand{\seppar}{1pt} %Between items and lists overall

\setenumerate[1]{itemsep=\sepitem, parsep=\seppar}
\setenumerate[2]{itemsep=\sepitem, parsep=\seppar}
\setenumerate[3]{itemsep=\sepitem, parsep=\seppar}
\setenumerate[4]{itemsep=\sepitem, parsep=\seppar}

\newenvironment{outline*}
{
	\begin{outline}[enumerate]
	}
	{\end{outline}
}

\newcommand{\foota}[1]{\hyperlink{#1}{$_#1$}}
\newcommand{\footb}[2]{\hyperlink{#1}{$_{#1.#2}$}}

\begin{document}

\title{Existentialism}
\author{Avery Karlin}
\date{Spring 2016}
%\newcommand{\textbook}{}
\newcommand{\teacher}{Mazzurco}

\maketitle
\newpage
\hypertarget{content}{\tableofcontents}
\vspace{11pt}
\noindent
%\underline{Primary Textbook}: \textbook\\
\underline{Teacher}: \teacher
\newpage

\section{The Problem of Other People}
\subsection{Hyperbolic Doubt}
\begin{outline*}
\1
\end{outline*}
\subsection{The Look}
\begin{outline*}
\1 Arthur Aron's study emphasizes that intimacy and trust are needed in a relationship, and can be created by stimuli, and the look facilitates this idea of trust, forced by the questions, so that the other person can properly understand about their partner, without a facade\foota{2}
\2 The goal of a relationship is to form a proper understanding, rather than a two-dimensional image, combining how they see themselves with how you perceive them
\3 It is also noted that to form a relationship based on this, one has to be looked for, providing the willingness, which is one of the reasons the experiments works, by taking those willing, such that they wouldn't close off
\2 It is especially emphasized by the author that the interesting thing is not to look into the eyes of the other, but to be seen
\2 In addition, the emphasis on sharing aspects that are liked creates an atmosphere of absorbing without reflection, forced by the situation, but then a confirmation of what the other actually thinks, emphasized further by sharing three things they agree on
\end{outline*}
\subsection{Photography}
\begin{outline*}
\1
\end{outline*}
\subsection{Solitary Confinement}
\begin{outline*}
\1 The effects of solitary confinement are shown to be hallucinating voices, talking to themselves \foota{1}, due to long term lack of human contact, with even automated systems to provide food \foota{1}
\2 Senses began to become redundant, due to the unchanging light \foota{1} and grey walls \foota{1}, making time and sight pointless, with the exception of being able to leave for an hour each day to exercise \foota{1}
\3 There is nothing to think about other than yourself, but no objective standard to observe based on, such that you lose the reason for existence, and lose your ability to define yourself or those around you
\3 Thus, people often cling to minute details of the world around \foota{1}
\3 Desperately working to preserve senses \foota{1}, such that on some level, he must believe it is real
\2 There is no method of protecting himself or confirming any information, at the mercy of the guards \foota{1}, such that he only trusts things he and others can see
\2 Lack of the feeling of existence, along with the space itself, due to not being percieved by anyone other than himself \foota{1}
\3 There are no necissary actions, such that there is no reason for existing, and no way of affecting anybody outside of the cell, such that they are fully forgotten \foota{1}
\4 This leads to experiments where isolated infants eventually stop eating and starve themselves, and relates to the experiment that humans are evolutionarily strengthened through society, and are not able to survive otherwise
\3 Humans rely on others to confirm the existance of phenomenon, such that if nobody else can see something, there is no proof it exists in his eyes
\1 This is related to the feeling that time moves faster during faster music and slower during slower music, such that time is observed in relation to other factors, rather than objective
\end{outline*}

\subsection{Hell == Other People}
\begin{outline*}
\1 Before Inez's entrance, Garcin emphasized both the fear of neverending awareness, but also having to live with his own thoughts \footb{5}{4}
\2 Sleeping is a way to escape, as well as blinking, turning off sight of the world \footb{5}{4}
\2 Harsh light that is always present, never letting one forget they have work to do, forced to remain aware \footb{5}{3}
\1 Life without breaks is similar to solitary confinement, without a proper passage of time, nonstop listening to ones own thoughts
\2 The removal of mirrors removes the last attempts to percieve other people, and prevents awareness of ones self as an object
\2 Lack of eyelids prevents the ability to prevent oneself from being observed, such that only others can observe
\1 **GET NOTES FROM MONDAY** 
\1 Inez's main function appears to be to irritate and create dissent to the point of insanity, creating dissent and provoking insecurity of other characters \footb{5}{22}
\2 She is also the only realistic one, willing to confront the reality of being in hell, used by Sartre to show his point of view  \footb{5}{26}
\1 The end is the basis of Sartre's idea of existance presiding essence, such that something must exist before it is thought of, rather than theorizing before executing \footb{5}{25}
\2 Sartre argues that ``a priori'' is not true of humans, and the definition of self is created by ourselves through actions, rather than inherant from birth
\2 Garcin dreaming of being a hero doesn't make him a hero, since in his final action, he fled from real danger and became a coward
\2 His actions don't back up his words, so while people can make themselves by willpower, it is based on action purely, not words
\2 Thus, positive intentions are not valid, and emotional responses prevnting actions is something to be purely overcome
\1 Near the end, Garcin and Inez acknoledge that everybody is watching and the idea of a crowd, and their thoughts, breaking the first wall \footb{5}{26}
\2 ``Hell is other people'' is the idea of fear of being mocked, judged by others for eternity for their actions, as Garcin is for being with Estelle and his cowardice by Inez
\2 Garcin is metaphoric for the French, not fighting the Nazi regime, while Inez can be interpreted as the regime not allowing them to live their lives without judgement, or just the resistance critisizing their lack of action
\end{outline*}
\section{The Problem of Self-Deception}
\subsection{Freud's Levels of Consciousness}
\begin{outline*}
\1 The Id is the primal part of the brain, containing the basic desires, located in the subconscious mind
\1 The Superego is the part of the mind which learns from our environment, within the subconscious mind, taking in rules and morals
\1 The Ego is the conscious mind, acting to balance the desires of the Id and the values of the Superego
\end{outline*}
\subsection{Bad Faith}
\begin{outline*}
\1 Bad faith is a form of self-deception, when we view ourself as an object without responsibility and free will, to escape the burdens of free will
\2 Sartre emphasizes that this is an active action, rather than a state
\1 The structure of a lie doesn't make sense, since the liar and the lied to cannot both fool and be fooled when they are the same
\1 Freud believed that certain drives are determined to be too dangerous to even be recognized by the ego, rather repressed and redirected in some other form of action
\2 This is the reason psychological patients give resistance as it moves closer to the repressed material, as the ego attempts to defend the repression
\2 This recreates the paradox of self-deception, as the ego both does not know about the lie, and must keep the lie repressed within the id
\1 Sartre countered this by three patterns of bad faith, deferring the moment of decision, molding oneself to society, and becoming an object \footb{7}{55}
\2 Thus, the first pattern is simply ignoring the urgency of the situation, ignoring implications in favor of objectivity, observing only what is present, similar to Estelle \footb{5}{9}
\2 The second pattern is the separation of the object and subject, or the mind and body, passively allowing circumstances around them from society to control themselves
\2 The third pattern is taking on a role, such that oneself is the Other/object, but lose a self-image distinct from the role by which the world sees you, fully working on perfecting this role
\end{outline*}
\subsection{New and Old Order}
\begin{outline*}
\1 The Argives are free, but unaware that they are, based on the perception of the ruler and Zeus as requiring repentance and focus on the leader, forcing them to maintain the image to preserve their power \footb{6}{56}
\2 Gods don't have true power over those who are already free, such that they have to keep people from noticing they are free to preserve order
\2 Zeus and Aegestus represent the old order of authority, while Orestes represents the new order of acting without regard for self, to create the traits he believes, the idea of choice reigning
\1 %Get rest of notes from Thursday
\end{outline*}
\section{The Definition of Existentialism}
\subsection{Tenets of Existentialism}
\begin{outline*}
\1 This lecture was an apology, or philisophical defense, of existentialism in response to those he felt misunderstood it, even after \textit{Being and Nothingness}\foota{8}
\1 Existentialism is said to be a philosophy of wallowing and inaction, only able to be done by the wealthy, since solutions are unattainable \foota{8}
\2 It also can be thought to focus on the unpleasant aspects of human nature and existence, ignoring the positive
\2 It focuses purely on subjectivity, isolating the individual, but loses out on community
\2 It also has no built-in, divine moral code, such that there is no method of determining if actions are correct or not
\1 Existentialism argues against the more pessimistic general notion that people must preserve the status quo and old order, to prevent anarchy, in favor of choice \foota{8}
\1 Existentialism believes foremost in the idea of existence before essence, that the subjective is needed to analyze human traits \foota{8}
\2 It rejects the notion that to create an object, the characteristics and method of production must be known before it is created, such that the essence/nature of the object is predetermined
\2 While Catholic existentialism exists, atheistic states that without the existence of God, man must define himself, rather than there being human nature, unlike other objects
\2 Men define themselves by a series of choices by respect to others, in the process judging the traits of others to give them value, contributing to the definitions of others, due to no objective ideal value
\1 If our actions are misread, nothing can be done other than knowing our own actions, bearing in mind that our ideas were a choice behind our action
\2 In addition, the idea of murder seems to be in contradiction, such that it would be removing the free will of others
\2 It also states that people are compelled to choose the trait they feel is best, not just for themselves, but for humanity as a whole
\1 Man is defined to be in ``anguish'', or the fear with regard to each decision, due to deciding traits of both themselves and humanity through actions, either appearing anxious, hiding it, or in denial
\2 External influences have their accuracy and existence determined through subjective judgment, such that all men must be able to choose the traits, since external sources can be discounted
\1 Abandonment is the idea that there is no God, and humanity must accept the consequences, such as the lack of moral codes, rather than state that human moral codes are universal, regardless of God, as if God existed
\2 In addition, since there is no human nature, there is no external force to fault, but rather all decisions are explicitly chosen
\2 Values and codes are too uncertain and ambiguous in most real life situations to be a guide, such that men must make their own judgments
\2 The action of trusting an instinct or alternate source is the action of knowing what result is desired, a choice
\2 Emotional strength is determined by action, such that the emotion can only be seen to be strong enough to justify an action after the action has been taken
\1 Despair is the idea of relying on probability which affect actions, such that since there is no God, there can be no optimism of bending probability
\2 
\end{outline*}

\section{The Problem of Free Will}
\subsection{Free Will and Determinism}
\begin{outline*}

\end{outline*}
\subsection{Ethical Theory}
\begin{outline*}
\1 Moral illusions are experienced, similar to illusions of the other senses, based on our initial perception of people translating to their actions, showing a moral sense \foota{9}
\1 The idea of a moralization switch is the idea of a frame of mind  that deems actions morally wrong, rather than incorrect \foota{9}
\1 Deontology, or nonconsequentialism, is duty-based ethics, based on the nature of actions and goals, rather than results, since people cannot control outcomes
\2 Kant believed that morality must be categorical (such that it is applied in any situation, rather than theory), beyond individual opinions, opportunities, abilities, and other conditions
\3 He furthered that lying is one of the worst atrocities, due to the removing or hiding of truth from an individual, written about in ``Metaphysical Principles on Virtue''
\2 He believed that actions must be intended such that the rule behind it could be a universal law, and treating each action onto another person as an end, rather than a means as the foundation of morality
\3 It is noted that the action is only an end if applying to another person, or oneself, rather than an object, based on respect for humanity
\3 This also creates the idea that all actions must be performed with the desire that all of humanity follows
\2 The main flaws are the ideas of the lack of a grey area, and the lack of a method to compare two contradictory actions, when not doing either may be a moral wrong
\3 This is the result of the murderer question, of providing refuge to someone, at which point a murderer approaches and asks if they have been seen, creating a necessary lie to one
\1 Utilitarianism, or consequentialism, based on the idea that only consequences of an act matter, rather than the motives, rules 
\end{outline*}
\section{Absurdism}
\begin{outline*}
\1 In ``The Stranger'', Camus attempts to outline his philosophy, related, but a branch off from Existentialism, called Absurdism, based on the response to nihilism that life is meaningless, but that humans have an obligation to make something of their life
\2 This is done by creating their own meaning, although in the long run, there is no meaning in life, such that the creation of meaning is considered necessary for life with consciousness, but absurd
\1 In ``The Myth of Sisyphus'', Camus tries to make the meaning of absurdism more apparent by showing the meaning, by Sisyphus, condsidered wise but imprisoned to purposelessly rolling a rock up a hill for eternity
\2 He is considered to have been punished for refusing to remain in the Underworld, chaining death, escaping back under the pretense of punishing his wife for not burying his body, and for not treating the Gods with the proper respect
\2 His enjoyment of life and passion for the world created the punishment of futility, attempting to achieve a task, watching it completed for a second before watching it fail
\3 This is the absurd victory, endeavoring to achieve a useless task simply to create meaning, knowing that it will fail in the long run, but accepting the victory in the short run
\2 The meaning comes from the desire to complete the pointless task, an absurd endeavor, but necessary for humans due to their own conciousness to have a purpose to live, opposing Meursault's lack of need to live
\3 It is accepted by absurdism that there will always be a force of chaos/nature against the completion of the task, but true happiness comes from the struggle itself to complete the task, which comes about from the meaning
\3 Camus causes the meaning produced ``a manual of happiness'', by which happiness is obtained, such that while the outcome is not under their control, their actions are, and meaning must be produced by that
\3 Both as a result, have scorn for others (the other people by Meursalt, the gods by Sisyphus), based on their idea that they rise above others, because the actions of others cannot effect them or their efforts to achieve their goals
\2 In addition, this is the idea of defiant joy, protesting the idea that life is pointless because of death, through the idea of lucid indifference to the concept to death, not caring about the inevitable end, such that there is no incorrect decision to life
\3 Camus considers the idea of lucid indifference to be tragic when concious, due to accepting that life will inevitable end in futility
\3 Meursalt, on the other hand, has lucidity, in that he accepts he will die, but is not indifferent, instead embracing the idea of death
\3 Sisyphus though, is different in that he cannot die, being dead already, such that he must be indifference to his inevitable fate, but not necessarily death, but more the lot to which he is given in life
\2 In this piece, similar to ``The Stranger'', nature is the force of opposition, the force of chaos and destruction, going against the creation of meaning, and showing the pointlessness of it
\3 On the other hand, both Meursalt and Sisyphus have an obsession with nature, especially water, as an element of happiness or of desire, the former approaching it before the murder, the latter after escaping the Underworld
\end{outline*}
\section{Readings}
\begin{enumerate}
\hypertarget{1}{\item ``The Concrete Abyss'' by Lisa Guenther}
\hypertarget{2}{\item ``To Fall in Love with Anyone, Do This'' by Mandy Len Catron}
\hypertarget{3}{\item ``Action Philosopher \#133 - Rene Descartes'' by Fred Van Lente}
\hypertarget{4}{\item ``NY Rethinks Solitary Confinement'' by NYT Editorial Board; Feb 20th, 2014}
\hypertarget{5}{\item ``No Exit'' by Jean-Paul Sartre, Stored PDF}
\hypertarget{6}{\item ``The Flies'' by Jean-Paul Sartre, Stored PDF}
\hypertarget{7}{\item ``Being and Nothingness'' by Jean-Paul Sartre, Stored PDF}
\hypertarget{8}{\item ``Existentialism Is a Humanism'' by Jean-Paul Sartre}
\hypertarget{9}{\item ``The Moral Instinct'' by Steven Pinker, NYT; Jan 13th, 2008}
\end{enumerate}

\end{document}
