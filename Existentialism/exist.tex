\documentclass[11 pt, twoside]{article}
\usepackage{textcomp}
\usepackage[margin=1in]{geometry}
\usepackage[utf8]{inputenc}
\usepackage{color}
\usepackage{indentfirst} %Comment out for no first paragraph indent
\usepackage[parfill]{parskip}
\usepackage{setspace}
\usepackage{tikz}
\usepackage{amsmath}
\usepackage{amsfonts}
\usepackage{amssymb}
\usepackage{enumitem}
\usepackage{outlines}

\usepackage{fancyhdr}
\pagestyle{fancy}
\cfoot{\hyperlink{content}{\thepage}}
\lhead{}
\chead{}
\rfoot{}
\lfoot{}
\rhead{}
\renewcommand{\headrulewidth}{0pt}
\renewcommand{\footrulewidth}{0pt}


\usepackage{hyperref}
\hypersetup {
	colorlinks,
	filecolor=black,
	linkcolor=black,
	urlcolor=black
}

\newcommand{\sepitem}{0pt} %Added room between items on the list, not including a list and its sublist
\newcommand{\seppar}{1pt} %Between items and lists overall

\setenumerate[1]{itemsep=\sepitem, parsep=\seppar}
\setenumerate[2]{itemsep=\sepitem, parsep=\seppar}
\setenumerate[3]{itemsep=\sepitem, parsep=\seppar}
\setenumerate[4]{itemsep=\sepitem, parsep=\seppar}

\newenvironment{outline*}
{
	\begin{outline}[enumerate]
	}
	{\end{outline}
}

\newcommand{\foota}[2]{\hyperlink{#1}{(#1, Paragraph #2)}}
\newcommand{\footb}[3]{\hyperlink{#1}{(#1, Page #2, Paragraph #3)}}

\begin{document}

\title{Existentialism}
\author{Avery Karlin}
\date{Spring 2016}
%\newcommand{\textbook}{}
\newcommand{\teacher}{Mazzurco}

\maketitle
\newpage
\hypertarget{content}{\tableofcontents}
\vspace{11pt}
\noindent
%\underline{Primary Textbook}: \textbook\\
\underline{Teacher}: \teacher
\newpage

\section{The Problem of Other People}
\subsection{Hyperbolic Doubt}
\begin{outline*}
\1
\end{outline*}
\subsection{The Look}
\begin{outline*}
\1
\end{outline*}
\subsection{Photography}
\begin{outline*}
\1
\end{outline*}
\subsection{Solitary Confinement}
\begin{outline*}
\1 The effects of solitary confinement are shown to be hallucinating voices, talking to themselves \foota{1}{4}, due to long term lack of human contact, with even automated systems to provide food \foota{1}{5}
\2 Senses began to become redundant, due to the unchanging light \foota{1}{6} and grey walls \foota{1}{3}, making time and sight pointless, with the exception of being able to leave for an hour each day to exercise \foota{1}{5}
\3 There is nothing to think about other than yourself, but no objective standard to observe based on, such that you lose the reason for existance, and lose your ability to define yourself or those around you
\3 Thus, people often cling to minute details of the world around \foota{1}{3}
\3 Desperately working to preserve senses \foota{1}{4}, such that on some level, he must believe it is real
\2 There is no method of protecting himself or confirming any information, at the mercy of the guards \foota{1}{6}, such that he only trusts things he and others can see
\2 Lack of the feeling of existance, along with the space itself, due to not being percieved by anyone other than himself \foota{1}{6}
\3 There are no necissary actions, such that there is no reason for existing, and no way of affecting anybody outside of the cell, such that they are fully forgotten \foota{1}{6}
\4 This leads to experiments where isolated infants eventually stop eating and starve themselves, and relates to the experiment that humans are evolutionarily strengthened through society, and are not able to survive otherwise
\3 Humans rely on others to confirm the existance of phenomenon, such that if nobody else can see something, there is no proof it exists in his eyes
\1 This is related to the feeling that time moves faster during faster music and slower during slower music, such that time is observed in relation to other factors, rather than objective
\end{outline*}

\section{Readings}
%For books, make sure to write the publishing copy, such that page number is valid
\begin{enumerate}
\hypertarget{1}{\item ``The Concrete Abyss'' by Lisa Guenther}
\hypertarget{2}{\item ``Action Philosopher \#133 - Rene Descartes'' by Fred Van Lente}
\hypertarget{3}{\item ``NY Rethinks Solitary Confinement'' by NYT Editorial Board; Feb 20th, 2014}
\end{enumerate}

\end{document}
