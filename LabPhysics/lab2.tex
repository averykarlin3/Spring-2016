\documentclass[11pt, titlepage]{article}
\usepackage{amsmath,amsthm,amssymb}
\usepackage{hyperref, pgf, tikz}
\usepackage{fancyhdr}
\usetikzlibrary{arrows}
\usepackage[margin=1.25in]{geometry}
\usepackage{graphicx}                     
\pagestyle{fancy}
\usepackage{array}
\usepackage{indentfirst}
%\usepackage{wrapfig}

\lhead{Lab \#1}
\rhead{\thepage}
\cfoot{}

\title{The RC Time Constant \\ \ \\ \large Lab \#1}
\author{Name: Avery Karlin \\ Partner: Jeffrey Zou}
\date{}
\begin{document}

\maketitle

\begin{center}
\LARGE The RC Time Constant
\end{center}

\section*{Objective}
The objective of the lab is to measure the time constant of an RC circuit based on the rate of charging and discharging.

\section*{Introduction}
Capacitors are defined by the equation $Q = CV$, such that the voltage of the capacitor is directly proportional to the charge, with the proportionality constant called the capacitance. Similar, resistor voltage is defined by Ohm's Law, such that $V = IR$. Thus, for a circuit purely with a resistor and a capacitor (RC circuit), the differential equation by Kirchoff's Voltage Law states that for charging (connected to the battery), $V_0 = \frac{Q}{C} + R\frac{dQ}{dt}$, and when discharging (disconnected from the battery), $0 = \frac{Q}{C} + r\frac{dQ}{dt}$. Thus, this can be solved to get an equation for charging, $V = V_0(1 - e^{\frac{-t}{RC}}) = V_0(1 - e^{\frac{-t}{\tau}}$ and for discharging, $V = V_0e^{\frac{-t}{RC}} = V_0e^{\frac{-t}{\tau}}$, where $\tau = RC$, called the time constant for an RC circuit.

These equations can be modified such that for charging, $-ln(\frac{V_0 - V}{V_0}) = -\frac{t}{RC}$ and for discharging, $-ln(\frac{V}{V_0}) = -\frac

\section*{Procedures and Results}

First, the Ammeter-Voltmeter methods are used with a small resistor, first creating a circuit such that the ammeter reading is slightly higher than actual, second so that the voltmeter reading is slightly higher than actual, with an additional variable resistor, a rheostat in this case using only one of the resistors, separate from the resistor being measured, in series. The resistance of the ammeter and the voltmeter can be separately measured by an ohmmeter, so that they can be taken into account. The voltage and current across the resistor can thus be found as well. It is then taken for three different rheostat resistances, such that the voltage and current through the unknown resistor varies. It is then redone by the same means for a large resistor.

\begin{figure}[h]
\centering
\hspace*{0cm}
\includegraphics[scale=0.12, angle=90]{lab11.jpg}
\vspace*{0cm}
\end{figure}

\begin{figure}[h]
\centering
\hspace*{0cm}
\includegraphics[scale=0.12, angle=270]{lab12.jpg}
\vspace*{0cm}
\end{figure}

The slide-wire form of the Wheatstone Bridge is then made in a variation, with a rheostat used as the two unknown resistors, due to acting as two varying resistors in series, with a divider between, similar to a slide-wire. It is measured for some small and large unknown resistor, taken for three different known resistors with each part of the rheostat resistance measured by an ohmmeter when there is no current passing through the bridge, measured by an ammeter.

\begin{figure}[h]
\centering
\hspace*{0cm}
\includegraphics[scale=0.7, angle=270]{lab13.jpg}
\vspace*{0cm}
\end{figure}

\underline{Small Resistor:}
\begin{center}
$$R_V \text{(Voltmeter Resistance)}= 10 M\Omega$$
$$R_X \text{(Expected Resistance)}= 50.2 \Omega$$
\begin{tabular}
{|m{9em}|m{7em}|m{7em}|m{7em}|}
\hline
Rheostat Setting & 1 & 2 & 3 \\
\hline
Current, I (A) & 0.048 & 0.039 & 0.031 \\
\hline
Voltage, V (V) & 2.6 & 2.1 & 1.6 \\
\hline
\end{tabular}
\end{center}

\begin{center}
$$R_A \text{(Ammeter Resistance)}= 0.48 \Omega$$
$$R_X \text{(Expected Resistance)}= 50.2 \Omega$$
\begin{tabular}
{|m{9em}|m{7em}|m{7em}|m{7em}|}
\hline
Rheostat Setting & 1 & 2 & 3 \\
\hline
Current, I (A) & 0.048 & 0.039 & 0.032 \\
\hline
Voltage, V (V) & 2.6 & 2.1 & 1.6 \\
\hline
\end{tabular}
\end{center}

\underline{Large Resistor:}
\begin{center}
$$R_V \text{(Voltmeter Resistance)}= 10 M\Omega$$
$$R_X \text{(Expected Resistance)}= 74.7 \Omega$$
\begin{tabular}
{|m{9em}|m{7em}|m{7em}|m{7em}|}
\hline
Rheostat Setting & 1 & 2 & 3 \\
\hline
Current, I (A) & 0.033 & 0.029 & 0.024 \\
\hline
Voltage, V (V) & 2.6 & 2.3 & 1.9 \\
\hline
\end{tabular}
\end{center}

\begin{center}
$$R_A \text{(Ammeter Resistance)}= 0.48 \Omega$$
$$R_X \text{(Expected Resistance)}= 74.7 \Omega$$
\begin{tabular}
{|m{9em}|m{7em}|m{7em}|m{7em}|}
\hline
Rheostat Setting & 1 & 2 & 3 \\
\hline
Current, I (A) & 0.032 & 0.029 & 0.021 \\
\hline
Voltage, V (V) & 2.6 & 2.3 & 1.7 \\
\hline
\end{tabular}
\end{center}

\underline{Wheatstone Bridge:}
\begin{center}
$$R_X \text{(Expected Resistance)}= 25.6 \Omega$$
\begin{tabular}
{|m{9em}|m{7em}|m{7em}|m{7em}|}
\hline
Trial & 1 & 2 & 3 \\
\hline
Known Resistor, $R_K (\Omega)$ & 50.2 & 74.6 & 25.6 \\
\hline
Varying Resistor 1, $R_1 (\Omega)$ & 24.2 & 26.1 & 17.5 \\
\hline
Varying Resistor 2, $R_2 (\Omega)$ & 12.8 & 9.4 & 17.5 \\
\hline
\end{tabular}
\end{center}

\begin{center}
$$R_X \text{(Expected Resistance)}= 74.7 \Omega$$
\begin{tabular}
{|m{9em}|m{7em}|m{7em}|m{7em}|}
\hline
Trial & 1 & 2 & 3 \\
\hline
Known Resistor, $R_K (\Omega)$ & 25.6 & 50.2 & 74.6 \\
\hline
Varying Resistor 1, $R_1 (\Omega)$ & 8.7 & 13.9 & 17 \\
\hline
Varying Resistor 2, $R_2 (\Omega)$ & 26.1 & 21.2 & 17.9 \\
\hline
\end{tabular}
\end{center}

\section*{Discussion}
Sample calculations for the non-measured data are as shown using the formulas found above:

$$R \text{(Voltmeter Interference, Small Resistor, Trial 1)} = \frac{V}{I - \frac{V}{R_V}} = \frac{2.6}{0.048 - \frac{2.6}{10^7}} = 54.16 \Omega$$
$$R \text{(Ammeter Interference, Small Resistor, Trial 1)} = \frac{V}{I} - R_A = \frac{2.6}{0.048} - 0.48 = 53.69 \Omega$$
$$R_{avg} \text{(Voltmeter Interference, Small Resistor)} = \frac{R_1 + R_2 + R_3}{3} = \frac{54.16 + 53.85 + 51.61}{3} = 53.21 \Omega$$
$$\text{Percent Error (Small Resistor, Voltmeter Interference)} = \frac{\text{$|$Expected - Actual$|$} * 100\%}{\text{Expected}} =$$\\$$\frac{|50.2 - 53.21| * 100\%}{53.21} = 5.66\%$$
$$R \text{(Wheatstone, Small Resistor, Trial 1)} = \frac{R_2R_K}{R_1} = \frac{12.8*50.2}{24.2} = 26.55 \Omega$$

\underline{Small Resistor:}
\begin{center}
\begin{tabular}
{|m{9em}|m{7em}|m{7em}|m{7em}|}
\hline
Rheostat Setting & 1 & 2 & 3 \\
\hline
Resistance, R $(\Omega)$ & 54.16 & 53.85 & 51.61\\
\hline
\end{tabular}
\\Average Resistance ($\Omega$) = 53.21
\\Percent Error = 5.66\%
\end{center}

\begin{center}
\begin{tabular}
{|m{9em}|m{7em}|m{7em}|m{7em}|}
\hline
Rheostat Setting & 1 & 2 & 3 \\
\hline
Resistance, R $(\Omega)$ & 53.69 & 53.37 & 49.52\\
\hline
\end{tabular}
\\Average Resistance ($\Omega$) = 52.19
\\Percent Error = 3.96\%
\end{center}

\underline{Large Resistor:}
\begin{center}
\begin{tabular}
{|m{9em}|m{7em}|m{7em}|m{7em}|}
\hline
Rheostat Setting & 1 & 2 & 3 \\
\hline
Resistance, R $(\Omega)$ & 78.79 & 79.31 & 79.17\\
\hline
\end{tabular}
\\Average Resistance ($\Omega$) = 79.09
\\Percent Error = 5.88\%
\end{center}

\begin{center}
\begin{tabular}
{|m{9em}|m{7em}|m{7em}|m{7em}|}
\hline
Rheostat Setting & 1 & 2 & 3 \\
\hline
Resistance, R $(\Omega)$ & 80.77 & 78.83 & 80.47\\
\hline
\end{tabular}
\\Average Resistance ($\Omega$) = 80.02
\\Percent Error = 7.12\%
\end{center}

\underline{Wheatstone Bridge:}
\begin{center}
\begin{tabular}
{|m{9em}|m{7em}|m{7em}|m{7em}|}
\hline
Trial & 1 & 2 & 3 \\
\hline
Resistance, R $(\Omega)$ & 26.55 & 26.87 & 25.6\\
\hline
\end{tabular}
\\Average Resistance $(\Omega)$ = 26.34
\\Percent Error = 2.9\%
\end{center}

\begin{center}
\begin{tabular}
{|m{9em}|m{7em}|m{7em}|m{7em}|}
\hline
Trial & 1 & 2 & 3 \\
\hline
Resistance, R $(\Omega)$ & 76.8 & 76.56 & 78.55 \\
\hline
\end{tabular}
\\Average Resistance $(\Omega)$ = 77.3
\\Percent Error = 3.48\%
\end{center}

The most likely cause of error is compounding loss of specificity due to the incorrect settings used on the multimeters. In addition, since the large resistors tend to have greater percent error than the other resistors, it indicates the measurement for that resistor's resistance may have been too low in some regard. Otherwise though, the overall percent error for each of the pieces of data was relatively low, such that it is mainly accurately done.

\section*{Conclusion}

The measured average resistance of the small resistor with the voltmeter needing to be accounted for, with an actual resistance of $50.2 \Omega$, was $53.21 \Omega$ with a percent error of 5.66\%. The measured average resistance of the small resistor with the ammeter needing to be accounted for, with an actual resistance of $50.2 \Omega$, was $52.19 \Omega$ with a percent error of 3.96\%. The measured average resistance of the large resistor with the voltmeter needing to be accounted for, with an actual resistance of $74.7 \Omega$, was $79.09 \Omega$ with a percent error of 5.88\%. The measured average resistance of the large resistor with the ammeter needing to be accounted for, with an actual resistance of $74.7 \Omega$, was $80.02 \Omega$ with a percent error of 7.12\%. The measured average resistance of the small resistor by the wheatstone bridge, with an actual resistance of $25.6 \Omega$, was $26.34 \Omega$ with a percent error of 2.9\%. The measured average resistance of the large resistor by the wheatstone bridge, with an actual resistance of $74.7 \Omega$, was $77.3 \Omega$ with a percent error of 3.48\%.

\end{document}
