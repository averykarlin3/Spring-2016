\documentclass[11 pt, twoside]{article}
\usepackage{textcomp}
\usepackage[margin=1in]{geometry}
\usepackage[utf8]{inputenc}
\usepackage{color}
\usepackage{indentfirst} %Comment out for no first paragraph indent
\usepackage[parfill]{parskip}
\usepackage{setspace}
\usepackage{tikz}
\usepackage{amsmath}
\usepackage{amsfonts}
\usepackage{amssymb}
\usepackage{enumitem}
\usepackage{outlines}

\usepackage{fancyhdr}
\pagestyle{fancy}
\cfoot{\hyperlink{content}{\thepage}}
\lhead{}
\chead{}
\rfoot{}
\lfoot{}
\rhead{}
\renewcommand{\headrulewidth}{0pt}
\renewcommand{\footrulewidth}{0pt}

\usepackage{hyperref}
\hypersetup {
	colorlinks,
	citecolor=black,
	filecolor=black,
	linkcolor=black,
	urlcolor=black
}

\newcommand{\sepitem}{0pt} %Added room between items on the list, not including a list and its sublist
\newcommand{\seppar}{0pt} %Between items and lists overall

\setenumerate[1]{itemsep=\sepitem, parsep=\seppar}
\setenumerate[2]{itemsep=\sepitem, parsep=\seppar}
\setenumerate[3]{itemsep=\sepitem, parsep=\seppar}
\setenumerate[4]{itemsep=\sepitem, parsep=\seppar}

\newenvironment{outline*}
{
	\begin{outline}[enumerate]
	}
	{\end{outline}
}

\begin{document}

\title{Physics II: Electromagnetism}
\author{Avery Karlin}
\date{Spring 2016}
\newcommand{\textbook}{Fundementals of Physics by Halliday, Resnick, and Walker}
\newcommand{\teacher}{Ali}

\maketitle
\newpage
\hypertarget{content}{\tableofcontents}
\vspace{11pt}
\noindent
\underline{Primary Textbook}: \textbook\\
\underline{Teacher}: \teacher
\newpage

\section{Chapter 21 - Electric Charge}
\subsection{Charge}
\begin{outline*}
\1 Matter is composed of specific particles (electrons, protons, and neutrons), each with a property of charge (-ve, +ve, or 0)
\2 Charge is measured in Coulumbs, generally a unit derived from Amperes for current, such that 1 C = 1 A*s
\2 Electrons have a charge of $-1.6*10^{-19} C$ and a mass of $9.11*10^{-31} kg$
\2 Protons have a charge of $-1.6*10^{-19} C$ and a mass of $1.673*10^{-27} kg$
\2 Neutrons have a charge of $0 C$ and a mass of $1.674*10^{-27} kg$, such that the mass is approximately the sum of that of an electron and proton
\2 The signs for electrons and protons were chosen by Benjamin Franklin by arbitrary convention
\1 The Law of Charges state that like charges repel and unlike charges attract
\1 Quantization of Charge states that the charge of any body is equal to the multiple of the charge of an electron/proton, called the elementary charge
\2 Quantized quantities are those which can only have discrete values of specific integer multiples of some constant, rather than any value
\1 Conservation of Charge states that charge is conserved, in all bodies, including subatomic, nuclear, and large-scale
\2 Thus, in nuclear reactions, charge is either annihilated or pair-produced (producing both charges simultaneously)
\end{outline*}
\subsection{Material Types}
\begin{outline*}
\1 Materials can be divided into conductors, insulators, and semiconductors
\2 The distance between the atomic radiuses of each of the atoms in the material determine the type, such that the further the distance, the less conductive, making conductors dense
\1 Conductors have electrons free to move from orbit to orbit, such as metals or impure water
\2 It is noted that this creates the difference of charged conducting bodies, such that the charges isolate themselves on the outside surface when there is no external field, unlike insulators which are dispersed evenly throughout
\3 The charges are only uniformly dispersed on the surface of a conducting sphere, while they vary in other conducting bodies, or in conductors with an external field acting on it
\3 In non-spherical conductors, the charge tends to condense near sharp edges and points, increasing the electric field around it, causing ionization of the air around, called a corona discharge, and often leading to a lightening strike by creating a high voltage
\3 This is also the reason why during lightening strikes, it is suggested to go inside spherical conductors
\3 Thus, spherical conductors also have constant potential within, due to having no internal electric field
\2 For conductors in an external field, since there is no constant current within, the charges must reorganize themselves such that there is still no electric field inside, due to the charges counteracting it
\3 By extension, all conductors in an external field have a constant electric potential within
\2 Thus, conducting spheres act as a spherical shell due to all charges on the outer layer
\1 Superconductors are conductors without any hindrance to movement of elections, such that there is no resistance
\2 Superconductivity is found at low temperatures, varying based on the material, with no relationship between conductivity and superconductivity, such that some conductors cannot become superconductors at any temperature, and some insulators easily are
\2 Within the last 40 years, higher temperature superconductors have been discovered as well, hurting attempts to develop theories of superconductors
\2 Increasing the temperature of conductors decreases the average collision time, such that the conductivity tends to decrease as temperature rises
\1 Insulators have a large energy gap ($E_g$), such that the electrons are bound to atoms, unable to jump/move, such as wood, glass, rubber, or pure water
\1 Semiconductors are not free or bound, such that they can be converted to either a conductor or insulator, such as silicon or germanium
\2 Semiconductors generally have the property of negative temperature resistivity coefficients, such that resistivity decreases as temperature increases, unlike most conductors, and lower charge carrier density and higher resistivity than conductors
\2 Adding impurity atoms to the material through doping allows electrons to jump easier, such that it conducts better, by providing atoms with charge carriers that are loosely held
\3 This functions to lower resistivity of the object by increasing charge carrier density, functionally reducing the amount of energy to cause the same amount of moving electrons
\2 Increasing temperature increases electron energy, allowing jumps temporarily, while decreasing temperature does the opposite
\3 While it decreases the average collision time, it drastically increases the charge carrier density, such that the effect of that lower resistivity renders the increase in collision time negligable
\end{outline*}

\subsection{Charging}
\begin{outline*}
\1 Charging by friction is done by rubbing objects together, using the frictional force to force electron transfer to the material with greater attraction to electrons
\2 Valence electrons are the ones lost from the material, specifically called conduction elections when able to be lost (in conductors)
\2 Glass and silk are known for producing positively charged glass
\2 Plastic and fur are known for producing negatively charged plastic
\1 Charging by contact is done by contact between two non-insulators, causing the charge to move until equilibrium
\1 Charging by induction is done by placing a charged rod near a neutral conductor, such that the material is polarized into a seperation of charges, giving a temporary/induced charge
\2 If the object is then split up in the case of a conductor, where the polarization is on a super-molecular scale, it can be charged
\2 If the object is an insulator, such that it is on a dipole molecular scale, it is temporary, but can allow it to be attracted to the object
\1 Electroscopes have two flat leaves connected to a rod, which is attached to the ball on top, such that it detects charge by charging by contact, after which the leaves seperate
\2 It can be discharged by attaching it to a grounding wire, attached to the Earth, which acts as a giant neutral body to neutralize the charges
\end{outline*}

\subsection{Coulomb's Law}
\begin{outline*}
\1 Coulomb's Law determines the electrostatic force of charged bodies, where electrostatic denotes stationary/negligible-movement charges
\1 $F_e = \frac{k|q_1||q_2|}{r^2}$, where the direction of the vector is determined by the law of charges
\2 $k = 9 * 10^9 \frac{Nm^2}{C^2} = \frac{1}{4\pi\epsilon_0}$
\2 The overall force must be found before breaking it up into components, rather than in finding it as components first
\1 The ratio of electrostatic force to gravitational force on a molecular scale approximates as $10^{40}$, such that gravity can be ignored on that scale
\2 Based on the hydrogen atom, with a radius of 0.53 Angstrom (0.53 * $10^{-10} m)$
\1 Shells of uniform charge density are found to have force interactions with charged point masses outside as if it was concentrated at the center
\2 Point masses inside the shell have no net electrostatic force acting on it
\end{outline*}

\section{Chapter 22 - Electric Field}
\subsection{Point Charges}
\begin{outline*}
\1 The electric field ($\vec{E}$) is a vector field, or a function of vectors at each point on the cartesian grid, allowing a measurement of the influence of a particle on any other particle at that point
\2 It is proportional to the number of field lines per unit area, such that high field line density means a stronger field
\3 Electric field lines are drawn from positive to negative
\2 Electric fields are found by measuring the force on a positive test charge with a small enough size that it doesn't disrupt the surrounding field, though it is noted that fields exist independently of the test charge
\1 $\vec{E} = \frac{\vec{F}}{q_0} = \frac{k|q|}{r^2}$, where the direction is determined by the type of charge, where $q_0$ is the test charge, and q is the charge of the point mass producing the field
\2 The unit of electric fields is N/C
\2 $\vec{E}_{net} = \sum_i \vec{E}_i$, or the electric field at some point/test charge due to a series of additional charges, such that superposition applies
\3 Superposition assumes that the charges remain in the same locations as when calculated individually, when together
\1 Electric dipoles are created by two point charges of equal magnitude, but opposite charge, seperated by some distance
\2 Dipole moment $(\vec{P}) = qr$, where r is the distance between them, and q is the charge of each point, where the vector goes from negative to positive
\3 The direction of the dipole is based on the direction of the electric field vectors acting on the point charge for any point on the dipole line
\3 This is able to be expressed by the asymmetrical charge distribution of even more complicated molecules
\2 $\tau_{external \to dipole} = P x E$, where E is the uniform external electric field
\2 $PE = -E \cdot P$ for dipoles in an external electric field, where E is the uniform external electric field
\2 $E_{dipole} = \frac{kP}{r^3}$, where r is the distance from the center of the dipole, where the point is on the dipole axis as r becomes far greater than the distance between the dipole charges
\3 On the other hand, E is proportional to $\frac{1}{r^3}$ at all points on the dipole axis, rather than just at a large distance
\3 This proportionality is due to the charges canceling each other out faster as the distance gets further
\2 Dipoles can be induced by an electric field, even in a nonpolar molecule, or single atom, such that an induced dipole moment forms in the same direction as the field
\3 The molecule/atom is then said to have been polarized temporarily by the field
\1 Sparks are due to electrical breakdown as electrons are forced out from atoms as the electrical field gets past the critical value ($E_c$)
\2 These electrons create blue light when they collide with atoms after being separated from their original atoms
\end{outline*}

\subsection{Solid Bodies}
\begin{outline*}
\1 $\vec{E} = \sum_i \frac{k\Delta q_i}{r_i^2} = \int \frac{k}{r^2}dq$
\2 $q = \lambda x, q = \sigma A, q = \rho V$ (depending on the dimensions of the solid body)
\2 If the density functions are non-uniform, the integral is taken to solve for q, with respect to each dimension variable 
\2 These calculations can be made simpler if one of the dimensions of the field cancels out, but a multiplier of some trig function must be added to the equation to remove that component
\1 For parallel, nonconducting plates, $\vec{E} = \frac{\sigma}{\epsilon_0}$, such that the field from each plate is half of that value
\2 $\epsilon_0$ is the permittivity constant of free space, or $8.854 * 10^{-12} C^2/Nm^2$, and $\sigma$ is the charge density of the plates
\2 This is due to all field lines moving in a different direction being canceled, assuming infinitely long parallel plates (or assumed to exist far from the edge of the plate, creating a uniform electric field
\2 Uniform electric fields are those with the same magnitude and direction at every point
\2 The fact that parallel plates have double the field of a single plate can also be thought to be due to all the charge moving to the edge, such that double the charge is present, rather than split on both edges
\3 This is not due to superposition, due to the charges when the plates are together, such that it must be calculated when together, rather than the sum of apart
\end{outline*}

\section{Chapter 23 - Electric Flux}
\subsection{Electric Flux}
\begin{outline*}
\1 Electric flux is the measure of flow of the field through an area, or the  number of electric field lines passing through an area, such that $\Phi_E = \vec{E} \cdot \vec{A}_n = \vec{E}\vec{A}_ncos(\theta)$, where $\vec{A}_n$ is the normal to the surface, perpendicular and with a magnitude equal to the area
\2 For an enclosed object, lines entering the surface of the object is negative, outward is positive
\2 Thus, positively charged objects have lines leaving but not entering, negative objects have vice versa 
\1 $\Phi = \vec{E} \cdot \vec{A}_n$ for uniform surfaces
\2 $\Phi = \lim_{\Delta A_n \to 0} \sum_i \vec{E}_i \cdot \Delta A_n = \oint \vec{E} \cdot dA_n$
\2 Electric flux is a scalar with $N*m^2$ as the unit
\end{outline*}

\subsection{Gauss's Law}
\begin{outline*}
\1 Gauss’s Law is used to calculate the graviational field at a point, by creating a gaussian surface including that point
\2 Gaussian surfaces must be imaginary, closed (compact/effectively-continuous without boundary in any direction), 3D surfaces, with a constant field throughout
\2 They also must be symmetrical objects to minimize calculations
\2 The electric field must also be parallel to the normal of the tangent plane, $A_N$, on the surface A
\1 Gauss's Law states that $\Phi_e \epsilon_0 = q_{enc}$, where $q_{enc}$ is the charge of the body inside the surface creating the field
\2 Gauss's Law is able to be derived from Coulumb's Law, such that $\Phi_e = \oint \vec{E} dA = \vec{E} \oint dA = \frac{kq_{enc}}{r^2} (4\pi r^2) = \frac{q_{enc}}{\epsilon_0}$
\2 By the definition of a surface integral, $\oint dA$ equals the surface area of the Gaussian surface
\2 This law only holds true in this form for charges in either a vacuum or air (effect of air is negligible)
\2 The electric field can be influenced by charges outside the surface, but the charge calculated is only that within the surface, due to outside charges not contributing to the net electric flux of the surface, but rather just adding field lines that pass fully through
\1 Gauss's Law can be applied to an infinitely long line of charge (or line of charge far from the ends), such that a cylinder can be constructed, since the flux through the top and bottom is parallel, and thus 0, such that the flux through the curved section is counted only
\2 This is used to derive the field of an infinite (or middle of the) sheet of charge at some distance as well, by creating a cylinder in the center around some area of the sheet of some height, thus the flux of the caps of the cylinder are the only ones used
\3 This is also used to derive the field between two parallel plates, due to the density doubling by the movement of charges toward the single edge near the center, rather than divided between the two edges when isolated, resulting in the field doubling
\3 This is also used to calculate the external field outside a conductor with nonuniform charge distribution, due to it being uniform and the surface flat over a tiny area
\4 On the other hand, it must be a conductor, or the charge would not be isolated in a flat layer on the edge of the surface
\2 It can also be used for spherical symmetry about some body, allowing the shell charge calculations to be shortened, as well as calculate the field within for nonconducting spheres by the charge enclosed
\2 Thus, bodies can be created without complete uniform field at the surface, such that the field is parallel to all surfaces that are not uniform, and thus can be not counted
\2 On the other hand, symmetry cannot be applied to the edges of an object, due to the edge effect/fringing causing the electric field lines to curve
\1 For varying charge density, $q_{enc} = \int \rho dV$ is substituted in, where dV can be converted to cartesian, cylindrical, or spherical depending
\1 Gauss's Law is also used to prove that conductors have the charge on the edges, due to if charge existed within, an electric field would, creating a constant internal current which isn't the case
\2 The same reasoning can be used to show that it is only in the outer surface if there exists a cavity within, rather than the cavity surface as well
\2 It is also used to show that a conductor works equally to an infinitely thin insulating shell
\end{outline*}
\section{Chapter 24 - Electric Potential}
\begin{outline*}
\1 Electric potential energy can be defined in a system from infinity as a reference as the negative work done by electrostatic force to move a particle from infinity towards some stationary, unaffected second charge ($\Delta U = U_r - U_\infty = U_r = -W_{\infty \to r}$)
\2 The electric potential energy can also be thought of as the work done by an external action to counteract the electric field, and move the particle from infinity to some stationary final setup
\2 Thus, the reference configuration is some system where each charged particle is an infinite distance from each other, such that reference potential energy is 0
\2 Electrostatic force is conservative, and thus path independent
\2 Energy is often measured in electron-volts, such that it is the energy required to move an electron through one volt (1.6 * $10^-19$ J)
\1 Electric potential (V) is a scalar quantity not dependent on the particle being moved, but rather existing at each point in a field, equal to the potential per unit charge
\2 Thus, $V_rq_0 = U_0$, where $q_0$ is the test charge, and r is the distance from the other particle
\2 As a result, it can be found that $V = \frac{kq}{r}$ for a field produced by a point charge, such that positive charges produce positive potential and vice versa
\3 By the superposition principle for potential energy, the potential at some point due to a group of charges is found by $V = \sum_i \frac{kq_i}{r}$
\1 $\Delta V$ is the electric potential difference, also called the voltage, and measured in Volts, or J/C
\2 $\Delta V = \frac{\Delta U}{q} = \frac{-W}{q}$, such that it can be calculated for point charge systems by the work done
\2 By extension, $\Delta V = \frac{-\int^f_i \vec{F}\cdot d\vec{s}}{q} = -\int^f_i \vec{E} \cdot d\vec{s}$, allowing it to be calculated by the movement of a particle in both an external field and a system of particles
\3 In the external field, the initial point can be thought to be the point of zero potential, such that $\Delta V = V_f$
\3 As a result, for fields created by point charges, since $V(\infty) = 0, V(r) = \int^r_\infty -E \cdot dr$
\4 In this case, the electric field must be found, not just at the point itself, but all points from there to infinity, such that if it is measured by multiple functions at various regions, must be broken up
\4 This function to find V is commonly used for symmetrical objects by which Gauss's Law can be applied, while the defining function is more commonly used for non-symmetrical ones
\3 This is used to prove that the potential within a conducting sphere creates the same potential within the sphere, due to no electric field inside
\2 For fields generated by dipoles, $V = \frac{kq(r_- - r_+)}{r_-r_+}$
\3 As r becomes far greater than the distance between the charges, $r_-$ and $r_+$ becomes parallel, such that $r_-r_+ = r^2$ and $r_- - r_+ = dcos\theta$, where $\theta$ is the angle between the dipole axis and each r
\4 $\theta$ is also noted to be the angle between the radius from the center of the dipole and the dipole axis
\3 Thus, $V = \frac{kqdcos(\theta)}{r^2} = \frac{kpcos(\theta)}{r^2}$ for dipoles as the distance from the dipole is far greater than the distance between the charges
\2 For a continuous charge distribution, $V = \int \frac{kdq}{r}$
\1 The work moving a series of charges is equal to the sum of the work to move each charge with respect to the work of the charges moved previously
\1 By the relationship of potential energy and force, $E = -\vec{\nabla} V(\vec{r})$
\1 Equipotential regions/points are those where the electric potential is equal, either a real or imaginary surface/region/points
\2 By extension, points are equipotential if the work done moving a charge from one to the other is 0, such that either electric field is 0 or the electric field is perpendicular to the movement of the charge
\2 Equipotential surfaces can be found by the IVT, such that if on a line between two points, the value moves past the desired potential, there must be some point such that it is equipotential
\end{outline*}

\section{Chapter 25 - Capacitance}
\subsection{Capacitors}
\begin{outline*}
\1 Capacitors are devices which can store electrical energy, made up of two conductor plates, represented by -$\|$- in a circuit schematic, conventionally in the form of a parallel plate capacitor
\2 Fully charged capacitors have both plates of equal charge, but opposite sign, with some region between the plates creating a field within, such that there is a voltage present, represented as V, rather than $\Delta V$ by convention
\3 The plates must be conductors, such that they are equipotential surfaces, creating parallel potential gradient
\2 Capacitance is the proportionality constant between the fully charged charge of each plate and the voltage, such that q = CV, where q is the charge of each plate
\3 Capacitance is measured in Farads, or C/V
\1 Capacitors are charged by the creation of an electric circuit, or a path allowing the flow of charge, with a battery to produce a specific electric potential between the terminals
\2 The terminals are the locations of the battery where charge can leave and enter into the circuit, but not directly to the other terminal
\2 Batteries are represented as a capacitor, with the terminal of higher potential as a longer line, and positive, and the terminal of lower potential as a shorter line, negative
\2 Switches are used to prevent current, by disconnecting the high and low potential of the battery, such that the charges are static when open until it is closed
\2 When the circuit is completed, equal magnitudes of charge are produced on both sides of the capacitor by the electric field created by the battery, until the potential of each capacitor plate is equal to that created by the battery
\3 It is generally assumed that charge cannot move through the capacitor, such that it cannot discharge unless the circuit has a mechanism to connect the plates
\3 Realistically, the battery does not bring the charges to the plates through the whole circuit, but rather forces charges to move from within the plate to the edges, only moving those closest to the edge to begin with
\3 The positive current going toward the capacitor plate forces positive charges to the edge, such that the plate the current hits first is positive
\1 For a parallel plate capacitor, $C = \frac{q}{Ed}$, where d is the distance between the plates, such that q can be found by Gauss's Law for some Gaussian cylinder such that the top field is equal to the uniform field within the plates
\2 Thus, $C = \frac{\epsilon_0EA}{Ed} = \frac{\epsilon_0A}{d}$
\2 Cylindrical capacitors can have the Gaussian surface constructed around the inner cylinder, such that $E = \frac{q}{\epsilon_02\pi rL}$, where L is the height of the cylinder, varying with r
\3 Thus, $V = \int^b_a Edr$, where a is the radius of the inner, b is the radius of the outer, such that $C = \frac{2\pi\epsilon_0L}{ln(b/a)}$
\2 Spherical capacitors similarly have $E = \frac{q}{4\pi\epsilon_0r^2}$, such that an integral is similarly found for V, to get $C = \frac{4\pi\epsilon_0ab}{b-a}$, where b is the outer radius, a is the inner
\3 A single sphere has a capacitance, where the outer sphere has an infinite radius, approximated by some object in which the field lines end, such as walls, such that $C = 4\pi\epsilon_0R$, where R is the radius of the inner sphere
\4 People can be similarly modeled as a spherical capacitor, allowing the maximum potential able to be achieved to be calculated, provided generally by friction charging
\1 Combinations of capacitors in circuits can be replaced by an equivalent capacitor to simplify
\2 In parallel, since the voltage of each is the same, the charge of the equivalent capacitor is equal to the sum of the individual charges
\3 The equivalent capacitance is also equal to the sum of the individual capacitance
\2 In series, the charge of each capacitor is the same, while the equivalent voltage is equal to the sum of the individuals, such that $\frac{1}{C_1} + \frac{1}{C_2} = \frac{1}{C_{eq}}$
\3 Each capacitor gains the same charge, due to each one filling to produce the same potential as the battery, acting as the battery for the next one, such that the battery only directly acts on the first
\4 This can be thought of as the battery only providing charges for the first, though technically, it only moves those there as well
\3 As a result, the equivalent capacitance is less than all of the individual capacitance in the series
\1 It requires work to charge a capacitor, creating the electric field within the capacitor, stored as potential energy within, such that $U = W = \int^q_0 Vdq = \frac{1}{C} \int^q_0 qdq = \frac{q^2}{2C} = \frac{1}{2}CV^2$
\2 Thus, the potential energy can be thought to be stored within the electric field of the capacitor plates
\2 Parallel plate capacitors have a uniform field, such that the energy density, or potential energy per unit volume is uniform as well
\3 Thus, $u = \frac{U}{Ad} = \frac{CV^2}{2Ad} = \frac{1}{2}\epsilon_0(\frac{V}{d})^2 = \frac{1}{2}\epsilon_0E^2$
\3 While this is most significant and derived in the case of a parallel plate capacitor, it allows any point within an electric field to be viewed as a storage of potential energy within a volume
\3 Thus, $u = \frac{1}{2}\epsilon_0E^2$ for all points in an electric field
\end{outline*}
\subsection{Dielectrics}
\begin{outline*}
\1 Dielectric are insulating materials, which increase the capacitance of the capacitor in which they are placed by the dielectric constant, k
\1 The dielectric constant of air is ~1, while the constant of a vacuum is exactly 1
\1 Dielectric materials also create a maximum voltage, $V_{max}$, called the breakdown potential, in each capacitor, at which point the dielectric breaks down and becomes a conductor
\2 The dielectric strength is the maximum electric field that can be withstood without breakdown, unique for each material
\1 All electrostatic equations taking place within a dielectric medium also have $\epsilon_0$ replaced with $k\epsilon_0$
\1 Dielectric mediums work by weakening the electric field, such that if it was charged, changing the voltage and potential energy stored, rather than the charge, converting the excess energy to mechanical energy in the dielectric
\2 This results in work acting on the movement of the dielectric, done by the field, while being inserted, remaining if there is no friction or external force acting on it
\1 Dielectrics line up opposed to the electric field, creating a small electric field in the opposite direction to decrease the electric field
\2 Polar dielectrics orient themselves to do this, while nonpolar ones gain temporary, induced dipole moments
\1 Gauss's law can be applied similarly, just with modifications due to a dielectric, derived by a parallel plate, either taking into account the added non-free charges within the dielectric, or the changing electric field (not both, due to one resulting from the other)
\2 Thus, either $\epsilon_0 \oint k\vec{E} \cdot d\vec{A} = q$, where q is purely the free charges, or $\epsilon_0 \oint \vec{E} \cdot d\vec{A} = q - q_{di}$, where $q_{di}$ are the opposite charges attracted to the plate within the dielectric, reducing the electric field
\3 k can be non-constant throughout the surface of the Gaussian surface, though to easily simplify the integral, it must not be
\3 $k\epsilon_0\vec{E}$ is also called the electric displacement, or $\vec{D}$
\2 Thus, $q - q_{di} = \frac{q}{k}$ by the relationship between the two
\1 Dielectric bodies inserted such that there are regions of differing dielectric constants within the capacitor allow the electric field of each section to be found by Gauss's Law
\2 The total voltage can be measured as the sum of the individual voltages of each section, proven by integration of electric fields
\2 Due to the free charges of the plates themselves being constant and the voltage being the sum of the individuals, the capacitance is calculated as if there are multiple capacitors in series
\1 Inserting conductors into a capacitor render the capacitance as 0 by the induced field canceling out the external, if inserted in only part, break up the capacitor in series, with that one as 0
\end{outline*}

\section{Chapter 26 - Current and Resistance}
\subsection{Current}
\begin{outline*}
\1 Electric currents are moving charges, such that there is a net flow of charge through the surface, considered a scalar quantity, though often an arrow is added to show the general direction of current flow
\2 As a result, the random movement of charge in electrons within a conductor without internal potential difference or the flow of neutral molecules are not considered currents
\3 It is noted that questions which ask specifically for the current of negative or positive charge among a neutral molecule flow ignore the lack of net charge flow
\2 Currents are generally assumed to be steady (not varying with time) movements of conduction electrons in metallic conductors, where there is assumed to be no resistance in the wire themselves
\3 Due to conductors having equal potential within, a battery must be added to create a forced potential, creating a current within
\2 Currents are measured in Amperes (A)
\2 $I = \frac{dq}{dt}$
\1 Due to conservation of charge, current is the same for any cross-section within the conductor
\1 The direction of current is drawn as the movement of positive charges, although the actual flow of charges is negative, in the opposite direction, due to the effect generally assumed to be the same
\2 In cases where it would change the effect of the current, this convention is not used
\1 Current density is the charge per unit area in the cross section, as a vector with the same direction as the current, such that $I = \int \vec{J} \cdot d\vec{A}$, where $\vec{A}$ is the normal of the area
\2 Current density can be represented by streamlines, such that the spacing between the lines/the density of the lines represents the current density at the cross-section
\1 Drift velocity is the velocity of the electrons in the direction of the electric field, minute compared to the random motion of the charges in both conductors with and without current ($10^{-5}$ m/s, rather than $10^6$)
\2 The random motion, rather than based on the temperature as classical physics expects, is based on quantum mechanics, at an effective velocity generally of $1.6 * 10^6$ m/s
\3 Due to the drastically smaller size of the drift velocity, the distance travelled by electrons is based on the effective velocity, with the drift velocity negligible comparatively
\2 This is the result of the slight random motion net movement in the direction of the field, creating the current
\2 Since the charge per length L of wire is q = nALe, where A is the wire cross section and n is the number of charges per unit volume, $J = IA$ if J is uniform over the length of wire, and L/t = $v_d$, then $\vec{v}_d = \frac{I}{nAe} = \frac{J}{ne}$ (if J is uniform)
\3 ne is considered to be the carrier charge density, or charge per unit volume, negative for electrons, such that the direction of $\vec{J}$ and $\vec{v}_d$ are opposite
\3 By the convention of assuming protons make up charge though, ne is generally assumed to be positive
\2 On the other hand, changes in the electric field strength throughout the wire are transmitted almost at the speed of light magnitude ($3*10^9$), causing all electrons to begin to drift
\3 This results in changes in the field causing immediate effects in circuits, such as the closing of a switch
\2 The drift speed is due to the acceleration from the electric field, where electrons are only allowed to drift until a collision, at which point they move back to the effective speed
\3 Thus, $v_d = a\tau = \frac{eE\tau}{m}$, where $\tau$ is the average time between collisions, independent of the electric field magnitude due to the comparatively small size of the drift velocity to the effective velocity
\1 Power of a circuit (P) = $\frac{dU}{dt} = \frac{dU}{dq}\frac{dq}{dt} = VI$, signifying the rate of transfer of electric energy in the circuit as work on the device connected
\end{outline*}

\subsection{Resistance}
\begin{outline*}
\1 Resistance is the difference in the currents resulting from the same potential in different objects, such that $R = \frac{V}{I}$, where it is measured in ohms ($\Omega$)
\2 Resistors are conductors, represented by a jagged line, used in a circuit to provide a specific resistance
\2 Resistance is based on the current density created in the resistors by a given voltage
\3 This is due to the fact that greater current density in a given area results in greater current, such that the resistance is less in that area, based on the object itself, rather than the material
\1 Resistivity ($\rho$) = $\frac{E}{J}$, is measured from a particular point, not depending on the object of the resistor, but the property of the material itself
\2 This is the microscopic variation of the macroscopic definition of resistance, using measures that apply to a material/point, rather than an object
\2 This assumes the material is isotropic, such that the electrical properties of the material are identical in all directions
\2 As a result, $R = \frac{\rho L}{A}$, allowing resistance and resistivity to be related by the definitions of each, assuming that current density/electric field and cross section are uniform in a homogeneous, isotropic conductor material
\3 Thus, resistance in an object can be determined to another object based on the relative length and cross-sectional area
\2 Similarly, conductivity ($\omega$) = $\frac{1}{\rho}$
\2 Resistivity varies with temperature, such that $\rho - \rho_0 = \rho_0\alpha(T - T_0)$, where $\rho_0$ and $T_0$ are at the reference temperature
\3 $\alpha$ is the temperature coefficient of resistivity, different for each material
\1 Ohm's Law states that resistance and resistivity is constant for a material for changes in applied electric field and potential difference
\2 Homogeneous materials of any type of conductivity follow Ohm's Law within a range of applied electric field strengths, but diverge eventually
\2 Thus is due to $v_d = \frac{eE\tau}{m}$ and $\frac{J}{ne}$, such that $E = \rho J = (\frac{m}{e^2n\tau})J$ and $\rho = \frac{m}{e^2n\tau}$
\3 Thus, since the average/mean average time, $\tau$, and the number of electrons per volume are independent of the field strength, and m and e are constants, $\rho$ is independent of the field strength and polarity
\1 Resistance converts electrical energy to thermal energy by the amount of collisions lowering the kinetic energy and releasing it as dissipated thermal energy
\2 $P = I^2R = \frac{V^2}{R}$ as a result for resistive dissipation
\end{outline*}
\section{Chapter 27 - Circuits}
\begin{outline*}
\1 Circuits can be powered either by a charged capacitor or a charge pump, the former temporarily until it discharges, the latter an object which produces a steady current by maintaining a voltage
\2 Charge pumps are also called emf devices, producing an emf, or electromotive force, often in the form of a battery and electric generator, or less commonly in the form of solar, fuel cells, thermophiles, or biological generators
\2 EMF is represented as an arrow from the positive to negative terminals of the device, with internal chemistry in the device to create current unless a circuit joins the terminals
\2 The EMF device forces charges in the EMF direction within the battery to create a constant potential, converting some other form of energy to produce the electrical energy, such as mechanical energy in generators
\2 $EMF = \frac{dW}{dq}$, or the work that must be done on each charge by the device to move a unit positive charge to the high voltage terminal, measured in Volts
\3 This is due to the need for the battery to move an equal amount of charge that is traveling in any given amount of time through a cross-section of the circuit, through the battery
\2 These EMF devices produce electrical energy, providing the energy to allow the movement of charges as current, allowing it to be used in other devices and converted/used as other forms of energy
\2 Current moving through a battery in the wrong direction (opposite EMF) is used to recharge the battery, converting electrical energy to chemical energy
\1 Ideal EMF devices are those without any internal resistance on the circuit, such that the voltage of the battery is equal to the emf (V = EMF)
\2 Real EMF devices are those with internal resistance in the battery, such that when not connected, the voltage is equal to the emf, but when connected to the circuit, is not equal (V $<$ EMF)
\2 Power of an EMF device = I*EMF, where in a real device, some of the power is dissipated by the internal resistance, and the remainder is used by the circuit
\3 Real EMF devices act as if they could be separated into an ideal EMF device and a resistor of the internal resistance in either order
\3 Thus, the voltage across a real battery, V = EMF - IR, where R is the internal resistance by application of the resistance rule and Kirchoff's Voltage Law, assuming the battery is discharging
\4 As a result, discharging real batteries have a lower potential difference between the terminal sides than recharging batteries, due to potential change opposing
\1 Kirchoff's Voltage Law states that the change in potential of a complete traversal of some loop (or the sum of changes that make a complete traversal) is equal to 0 $(V_{aa} = V_{a\ to a} = \Delta V_{a \to a} = V_{aend} - V_{astart} = 0)$
\2 Kirchoff's Current Law states that the sum of a current entering a junction is equal to the sum of the current leaving it
\1 Current in a single-loop circuit can be calculated either by conservation of energy (V = IR) in single-loop, single-resistor circuits, and by potential for more complex
\2 Kirchoff's Voltage Law on a loop to calculate by potential, such that the sum of each small potential change as the loop is moved through sums to 0
\3 The Resistance Rule states that resistance while traversing in the direction of current have a negative change in potential (-IR), in the opposite direction, positive (IR)
\3 The EMF Rule states that ideal EMF devices while traversing in the direction of negative plate to positive is positive, conversely negative
\2 Potential between two points is calculated similarly using Kirchoff's Voltage Law over a small region as the sum of the individual changes in potential
\2 Multi-loop circuits can have the voltage taken for multiple loops to form a system of equations by which it can be solved
\2 If not stated, in a circuit with junctions or multiple batteries, the current can be chosen to go in any direction, due to only causing a negative sign if incorrect, moving in the direction of decreasing potential
\3 This is due to each battery creating a secondary current, interacting with the main current to enhance or cancel
\1 Resistors in series can be thought of as a single resistor with equivalent resistance of the sum of the individual, due to constant current and voltage as the sum of the individuals
\2 Parallel resistors have an equivalent resistance of the reciprocal of the sum of the reciprocals of the individual resistors, due to equal voltage and the sum of the individual currents as the total current
\3 Thus, the equivalent resistance is smaller than any of the individual resistances
\1 Grounding a circuit is drawn as a branch off, with three parallel lines, decreasing in size as it moves away from the branch-off
\2 This does not change the circuit function itself, but simply provides a reference potential by which the potential itself can be found at specific points, rather than just the voltage
\2 The grounding point is considering to have 0 Volt electric potential
\1 Amnmeters are used in series to measure the current of a wire, and must have extremely low resistance, while voltmeters are in parallel, used to measure the potential difference across a point, and must have an extremely high resistance
\2 Multimeters are those either to act as both, and ohmmeters are used to measure resistance
\1 RC circuits are those consisting of a capacitor, battery, and resistor, with a switch to disconnect the battery from the circuit, time-dependent
\2 By Kirchoff's Voltage Law, since $V = -\frac{q}{C}$, since the capacitor is made such that there is a drop in potential opposite the direction of the battery, such that $V_0 - R\frac{dq}{dt} - \frac{q}{C}$
\3 This is used to derive the differential equation, $q_{cap} = -CV_0e^{-\frac{t}{RC}} + CV_0$, such that as time increases, it approaches $CV_0$ (maximum capacitor charge), starting at 0
\3 Further, this is used to derive current, such that $I = \frac{V_0}{R}e^{-\frac{t}{RC}}$, such that initial current is $\frac{V_0}{R}$, approaching 0
\2 When the switch is moved such that the battery is disconnected, the capacitor acts to discharge, such that the differential equation $-\frac{q}{C} - \frac{dq}{t}R = 0$ is used
\3 This derives $q_{cap} = CV_0e^{-\frac{t}{RC}}$, such that initial charge is $CV_0$, approaching 0 as time continues
\3 This is also used to derive current, such that $I = \frac{V_0}{R}e^{-\frac{t}{RC}}$, such that it begins at $\frac{V_0}{R}$, approaching 0 as time increases
\2 This shows that the capacitor when empty, acts as a conducting wire, eventually becoming non-effective when filled
\2 RC is considered to be the capacitive time constant of the circuit, shown by $\tau$, measured in seconds, used to give the general form of the equation
\end{outline*}

\section{Chapter 28 - Magnetic Field}
\begin{outline*}
\1 Magnetic fields are produced by a magnetic charge, only proved to exist in dipole pairs of positive and negative
\2 Electromagnets are current-created magnets, while permanent magnets are objects where the magnetic field of the particles, as an intrinsic/characteristic field, does not cancel out in the material
\2 Monopoles have been theorized to exist for certain theories, but are not proven
\1 Magnetic fields are defined as $\vec{B} = \frac{\vec{F}_B}{q\vec{v}}$, such that $F_B = q\vec{v} \times \vec{B}$
\2 It is defined in terms of velocity of a particle in the field, due to being unable to define it by a monopole as electric fields
\2 The right hand rule is then used, such that the fingers point toward the field, thumb toward velocity, and palm toward the force on the right hand
\3 For negatively charged particles, the opposite direction of the velocity is used instead
\2 Magnetic fields are measured in Tesla (T), or N/A*m
\3 Gauss (G) is also used occasionally, such that 1 Tesla = $10^4$ Gauss, approximately the Earth's field at the surface
\1 Magnetic fields are represented by field lines, where the density determines the strength of the field at that point, drawn from the north pole to south pole
\2 It is notable that the south magnetic pole of the Earth is at the geographic north and vice versa
\1 C-shaped magnets are those where the poles are fend into a curved shape, such that the poles are facing each other, producing a uniform magnetic field between
\2 Horseshoe and bar magnets are the other common types, changing the drawing of the field lines
\1 Electric and magnetic fields can be placed perpendicular to each other, such that they are crossed, able to push the particles in the same directional axis
\2 This was used to discover the electron in a cathode ray tube, which emits electrons from a hot filament, accelerated temporarily by an applied voltage, passing through a slit to form a beam
\3 After, it passed through crossed fields, forcing the particles in opposite directions, such that the electric field is turned on, then the magnetic field is modified until they have an equal effect
\3 If the undeflected value is considered to be 0 on the y-axis, y = $\frac{|q|EL^2}{2mv^2}$, where v is the speed of the particle, m is the mass, q is the charge, and L is the length of the plates between which the fields exist
\3 This was used to measure the ratio, $\frac{m}{|q|}$, considered to be the discovery of the electron by Thomson in 1897
\2 It was also used in the discovery of the Hall Effect, stating that drifting electrons in a wire can be deflected, by which the charge of charge carriers in a conductor can be determined
\3 As the magnetic field forces the protons to one side of the wire, this creates a charge separation within, such that a Hall potential difference forms
\4 The potential difference increases until the force balances out that from the magnetic field
\3 Voltmeters can then be connected to determine the potential, as well as the side of higher potential, the latter of which allows determination of the charge type of the current
\4 By the definition of potential difference, drift velocity, and the equilibrium, $n = \frac{IBd}{VeA}$, where n is the number of charge carriers per unit volume
\4 This can also be used to determine drift velocity by moving the wire until the relative velocity of the charges to the field is 0, such that there is no magnetic field, and the Hall potential disappears
\1 Due to magnetic force acting perpendicularly to the field and the velocity, it causes the particular to move in uniform circular motion within a uniform field, such that $|q|vB = \frac{mv^2}{r}$
\2 Thus, angular frequency ($\omega$) = $2\pi f = \frac{2\pi}{T} = \frac{|q|B}{m}$, such that the time of revolution depends purely on the field and charge to mass ratio, rather than on the velocity of the particle
\3 As a result, it is the radius of the circle that changes based on the velocity, rather than the period/frequency
\2 For particles with a velocity component parallel to the magnetic field, it moves in a helical path with constant upward velocity
\3 The pitch of the helix, or distance between adjacent turns, can thus be found as the velocity parallel multiplies by the period of the motion
\3 Helices where the magnetic field is strong enough at the end that the period is approximately 0, causes the reflection of the particle down the helix, such that if both ends are like that, is called a magnetic bottle
\2 Mass spectrometers measure the mass of a particle by using a produced voltage to produce the velocity, firing it into a magnetic field, and measuring the radius of the circle revolved around in a uniform magnetic field
\1 Particle accelerators are used to create particle collisions with high energy, easy for electrons due to the small mass, difficult for large particles due to requiring large travel distance
\2 Cyclotrons use two dees, or semi-circle chambers, with a gap of potential difference between them, and a magnetic field applied to cause circular motion, starting from the center, with the velocity increasing until it moves outside the dee/the magnetic field in a beam
\3 The electric potential is reversed before each move between the dees, such that it is accelerated each time, with no electric field allowed inside the dees
\3 This requires a fixed frequency of oscillation to be set equal to the field oscillation as the resonance condition
\3 The kinetic energy it leaves with in the beam is assumed to be the kinetic energy required to move in circular motion around the full cyclotron for simplicity
\2 Proton Synchrotrons are used for particles approaching the speed of light (> 0.1c), due to the frequency of revolution no longer being independent of the speed when in a relativistic setting, but rather decreasing as velocity increases
\3 In addition, as required velocity rises, the size of the magnets and the dees become much larger
\3 Thus, rather than a constant magnetic field, it uses a varying field and varying oscillator frequency, to produce a purely circular, rather than spiraling path, though it still requires a fairly large magnet, but smaller than a cyclotron would
\1 Uniform magnetic fields acting on a wire produce a force on the wire itself, due to the electrons within being unable to escape
\2 Since $F_B = qv_d \times B$, where $v_d$ is the drift speed, and $q = It = I\frac{L}{v_d}$, then $F_B = IL\times B$
\3 $\vec{L}$ is the vector length of the wire, considered to go in the direction of the conventional/positive current
\2 It can also be measured for some small section of wire, $d\vec{L}$, such that the total force on the wire is summed, equal to the force on the wire as a whole
\1 Current loops within a magnetic field produces a separate force on each side of the loop, rotating it on the central axis, the mechanism behind most electric motors
\2 The normal vector for the orientation of the loop is determined by the right hand rule, curling fingers in the direction of the current, such that the thumb is the normal vector
\2 For there to be torque, the magnetic field has to be perpendicular to both the normal vector and the rotational axis, with the angle between the normal vector and the magnetic field denoted as $\theta$
\3 Thus, for a rectangular shape, $F_{side} = ILBcos(\theta)$, where L is the length of the side on which torque is being created, such that $\tau = 2F_{side}r = 2ILBrsin(\theta)$, where r is the radius of the sides from the rotational axis
\3 This can be further modified, such that $\tau = ILBRsin(\theta)$, where R is the total length of the loop about the axis, extended to $\tau = IBAsin(\theta)$, which applies for all loop shapes
\2 Similarly, coils have the same torque acting on each turn, such that the total torque is the individual torque multiplied by the number of turns
\3 This assumes that the coil is flat enough that each turn appears as a loop, forming a flat coil
\2 Coils thus turn, such that the normal vector is parallel and in the same direction as the magnetic field itself, acting as a magnetic dipole, where the normal vector points from the north
\1 The magnetic dipole moment, $\vec{\mu}$, points in the direction of the normal vector, such that it points from the south to north pole, existing for all dipoles, such as bar magnets, current loops in a magnetic field, or rotating spheres of charge such as the Earth
\2 $\mu = IA$ for some current-carrying loop, such that for a coil, it is that value multiplied by the number of turns, such that $\tau = \vec{\mu} \times \vec{B}$
\2 Similarly, it is found that for some dipole rotating around an axis, $U = -\vec{\mu} \cdot \vec{B}$, where the lowest energy is when the magnetic field and normal vector are parallel in the same direction, greatest when opposite
\3 As a result, the zero reference for the rotating dipole is determined to be when the normal vector and magnetic field are parallel
\end{outline*}
\section{Chapter 29 - Current-Produced Magnetic Fields}
\begin{outline*}
\1 Magnetic fields are produced by current within a wire, such that the Biot-Savart Law, experimentally found, states that $d\vec{B} = \frac{\mu_0 Id\vec{s} \times \hat{r}}{4\pi r^2}$
\2 $d\vec{s}$ is the current carrying length element in the direction of the current, r is the distance to the point at which the field is being determined, and $\hat{r}$ is the unit vector in the direction of r from the wire length element
\2 $\mu_0$ is the permeability constant of free space, or $4\pi * 10^{-7} T*m/A = 1.26 * 10^{-6} T*m/A$
\2 The direction is determined by $d\vec{s} \times \hat{r}$
\2 It can be easily proven from this that $B = \frac{\mu_0 I}{2\pi R}$ for an infinitely long, straight wire
\3 The direction is found by the right hand rule for current-carrying wires, where the thumb is in the direction of current, and the fingers curl in the direction of the field
\4 The left hand is used instead for negative current
\4 This rule can be used for some minute length of curved wire as well, summing the field directions of each section
\2 Similarly, the field produced by a curved length of wire can be found, such that $B = \frac{\mu_0I\theta}{4\pi R}$, where R is the distance from the arc and $\theta$ is the angle spanned by it
\2 Magnetic fields for more complex wires or systems of wires can be found by breaking it up into parts and finding the sum of the resultant field from each individual part
\2 This can be combined with the formula for the force acting on a wire by an external field to find the force acting on a wire by another wire
\3 Thus, it is shown by the right hand rule that parallel currents attract, antiparallel repel
\3 1 Ampere is defined as the current in two parallel wires of infinite length, 1 meter apart in a vacuum, such that the force produced is $2*10^{-7}$ N per meter
\3 Railguns work by passing a current in one direction through the barrel, through a conducting material/fuse to the other wire, made such that it moves in the opposite direction 
\4 The current causes the material to melt, producing a conducting gas, in which current continues to flow, creating an outward current above it when the bullet is, forcing the bullet out with $5 * 10^6g m/s^2$ acceleration
\1 Ampere's Law states that $\oint \vec{B} \cdot d\vec{s} = \mu_0 I_{enc}$, such that the integral is a line integral, taken over some closed/Amperian loop
\2 The direction of integration along the loop is arbitrarily chosen, such that current direction such that it moves up inside the loop as integration is counterclockwise is positive, current which moves down is negative
\2 This ignores current outside the loop due to the magnetic field lines both entering and leaving the loop, such that it cancels out
\2 This is simplified easily for symmetrical magnetic fields, such that $\vec{B}$ is constant throughout the loop, able to be removed from the integral
\2 Similarly to Gauss's Law, for varying enclosed current density, the total enclosed current can be found by $I_{enc} = \int JdA$
\2 It is notable that the current inside is equal to the total current passing through the loop, such that it is the sum of each current carrying wire cross-section going through
\2 Ampere's Law is commonly used to determine the magnetic field produced by a solenoid, or a tightly wound helical coil of current-carrying wire around a metal core, where the length is assumed to be far greater than the diameter
\3 Ideal solenoids are assumed to be tightly/close packed and infinitely long, such that the field inside the core is uniform and parallel to the axis
\4 In addition, ideal solenoids have the magnetic field outside the solenoid equal to 0
\4 These features are mainly true if the length far exceeds the diameter (except at the ends)
\3 The Amperian loop is taken as a rectangle through the wire, such that one side is fully within, one fully without, such that the inside is the only part of the loop whose integral is non-zero
\4 Further, the current enclosed by the loop is equal to the sum of the individual currents of each wire part passing through the loop, such that $I_{enc} = Inl$, where n is the number of turns per unit length and l is the length of the loop
\4 Thus, $B = \mu_0In$ for an ideal solenoid, depending only on the number of turns per unit length and the current, such that it can be used easily to create a known uniform magnetic field
\4 For solenoids with multiple layers as a result, that simply increases the number of turns, since the radius doesn't matter
\3 Toroids, or hollow coiled solenoids twisted such that the ends meet, can be determined similarly, through an Amperian circular loop inside the hollow space, such that $B = \mu_0In$
\4 Thus, $B = \frac{\mu_0IN}{2\pi r}$, where N is the total number of turns and r is the radius to the loop
\1 Current-carrying loops and coils, due to forming a magnetic field, form a dipole by themselves as well, acting as a bar magnet, such that the magnetic field on the central axis can be calculated in terms of the dipole moment
\2 By integrating over the loop by Biot-Savart's Law and cancelling out perpendicular field lines, it is found that $B(z) = \frac{\mu_0IR^2}{2(R^2 + z^2)^{3/2}}$, where z is the distance from the center of the loop, where north is positive
\3 The magnetic field is generally assumed to be constant within the coil itself, such that z = 0, where it is measured starting from outside the coil
\2 As the distance on the axis becomes much greater than the radius, $B = \frac{\mu_0IR^2}{2z^3}$, multiplied by the number of loops for a coil, such that it is generalized to $B = \frac{\mu_0\vec{\mu}}{2\pi z^3}$
\end{outline*}
\section{Chapter 30 - Induction}
\subsection{Faraday-Lenz Law}
\begin{outline*}
\1 Faraday discovered that the movement of a magnet produces an induced current by providing induced EMF, calling it induction
\2 It was also discovered then that faster motion produces a greater current
\3 He also found that moving one pole towards causes one direction of current, moving away the opposite, and the other pole is the opposite directions
\2 He later discovered that with two conducting loops next to each other, when one connected to a voltage, an induced current in the opposite direction is produced in the other temporarily
\3 The induced current is only during the change in the current of the first loop
\1 Magnetic flux is the amount of magnetic field lines passing through an area, such that $\phi_B = \int \vec{B} \cdot d\vec{A}$, measured in Wb, or $T*m^2$
\2 The EMF produced by the change in flux of an area is the EMF around the outside of the area, such that the current is on the boundary of the area
\2 For a coil, the magnetic field lines pass through each coil seperately, with equal flux if it is closely packed enough, such that the flux is the flux through 1 coil multiplied by the number of coils
\1 The Faraday-Lenz Law states that $EMF_{induced} = -\frac{d\phi_B}{dt}$
\2 Faraday's Law of Induction states that the induced EMF is the result of the magnitude of change in the number of field lines through an area
\3 Thus, change in the flux of a coil creates that EMF in each individual coil, multiplying the total EMF by the number of coils
\3 This can be done by modifying the angle between the field to the area, modifying the area, or modifying the strength of the field at the area (such as by moving the pole of a magnet closer)
\2 Lenz's Law states that the induced current has a direction such that the magnetic field opposes the change in magnetic flux that induced it
\3 This means that the induced magnetic field doesn't necessarily oppose the external field, but only the change in the external flux
\1 When magnetic flux is changed by object motion, the magnetic force resists the motion by Lenz's Law, such that applied force is needed to counteract it, while simultaneously, thermal energy is created by the resistance of the wire
\2 The applied force energy creates the thermal energy, such that the power of the applied force creates the power of the energy transfer
\2 By the movement of a loop out of the magnetic field of height L, EMF = BLv, such that $I = \frac{BLv}{R}$
\3 The magnetic field of the portions of the wire within the field, parallel to the movement of the loop, cancel each other out, such that only the other perpendicular side creates an opposing force, where $F = ILB = \frac{B^2L^2v}{R}$
\3 By P = Fv, the power produced by the movement of the wire can be found by this function
\2 Similarly, by the formula for the power dissipation of thermal energy of resistors, it is calculated to be the same value, $P = \frac{B^2L^2v^2}{R}$, such that all work done on the loop is converted to thermal energy
\1 Eddy currents are currents produced in solid conductors by change in flux from a magnetic field, swirling in a whirlpool rather than on a single path, but represented as if following a single path
\1 The Faraday-Lenz Law, in addition to predicting the creation of current and a magnetic field, predicts an induced electric field to create the current, produced even without a conductor
\2 Due to the creation of circular motion of charges, the electric field lines form a series of concentric circles within the magnetic field, with the magnitude based on the amount of area by which flux changes
\3 It is noted that electric potential as a result of these concentric circles has no meaning, such that while the voltage on a closed path is required to be zero, there is an induced EMF produced, such that only EMF is used, not potential
\2 For some circular charge movement, $W = \oint \vec{F} \cdot d\vec{s} = q_0E(2\pi r)$, and by the definition of voltage, $W = EMF*q_0$, such that $EMF = 2\pi rE$
\3 Thus, $EMF = \oint \vec{E} \cdot d\vec{s} = -\frac{d\phi_B}{dt}$, giving a new form of the law, where the flux is determined within the path
\3 As a result, if the change in magnetic field is zero, the sum of the induced electric field over the loop will be, even if there exists a field at certain points on it
\2 This provides a definition of EMF other than work per unit charge due to changing magnetic flux/to produce current, but rather purely in terms of an electric field on a closed path
\end{outline*}
\subsection{Inductors}
\begin{outline*}
\1 Inductors, represented in circuit diagrams by a coiled line, are used to produce a uniform magnetic field, defined by a solenoids, but which includes toroids or coils
\2 Inductance, L, is defined such that $LI = N\phi_B$, where N is the number of turns of the coil, measured in Henry (H)
\3 The turns are considered to be linked by the shared equal flux within, such that $N\phi_B$ is called the magnetic flux linkage
\3 As a result, $L = \frac{N\phi_B}{I} = \frac{nLBA}{I} = n^2L\mu_0A$ for a solenoid, showing that it is independent of the current, purely based on the device structure itself, similar to capacitors
\4 This assumes the length is far greater than the radius, such that it ignores the fringing on the end of the solenoid, similar to parallel plate capacitors
\2 For simplicity, it is assumed that there are no magnetic fields near inductors, which would distort the uniform field produced
\1 Due to the magnetic field of an inductor being dependent on the current, changes in the current produce an induced EMF, derived by Faraday-Lenz such that EMF = $-L\frac{di}{dt}$, opposing the change in current, called self-induction
\2 While potential cannot be defined within the inductor, the overall EMF (and voltage in an ideal inductor, seperated into a resistor otherwise) produced by the inductor can be
\1 Similar to RC circuits, RL circuits can be constructed with a resistor and inductor connected to a battery, able to be switched such that the battery is disconnected from the circuit
\2 While the battery is connected, the equation $0 = -IR - L\frac{dI}{dt} + V$ describes it due to opposing the initial change in current and initial current is 0
\3 Thus, $I = \frac{V}{R}(1 - e^{-\frac{tR}{L}}) = \frac{V}{R}(1 - e^{-\frac{t}{\tau_L}})$, such that the inductor initially allows no current through it, acting as a broken wire
\3 $\tau_L$ is the inductive time constant, or $\frac{L}{R}$
\3 By this equation, as time approaches $\infty$, the current approaches $\frac{V}{R}$, such that it acts as a normal wire
\2 When the battery is disconnected, the equation $0 = IR + L\frac{dI}{dt}$ describes it, opposing the drop in current, with the initial current as if there was only a resistor
\3 Thus, $I = \frac{V}{R}e^{\frac{-tR}{L}} = \frac{V}{R}e^{\frac{-t}{\tau_L}}$, approaching zero as time approaches $\infty$
\2 Since $P = LI\frac{dI}{dt}$, the energy stored in the magnetic field of the inductor is calculated as $U_B = \frac{1}{2}LI^2$, varying with time
\3 This can be used on a solenoid to calculate the energy density, such that $u_B = \frac{U_B}{Al} = \frac{LI^2}{2Al} = \frac{1}{2}\mu_0n^2I^2 = \frac{B^2}{2\mu_0}$
\4 This is able to be used in general within a magnetic field although derived by for a solenoid only
\1 Mutual induction is induction produced by the change in the current of a second conducting loop resulting in the creation of a current in the first, distinct from self-induction
\2 Mutual inductance is defined as a constant similar to inductance, such that $M_2I_1 = N_2\phi_{21}$, where $\phi_{21}$ is the flux in coil 2 as a result of the current from coil 1
\2 As a result, similarly to self-induction, $EMF_2 = -M_2\frac{dI_1}{d2}$
\2 It has been found separately that the mutual inductance of two coils is dependent on both coils, such that it is equal for each
\end{outline*}
\end{document}
