\documentclass[11 pt, twoside]{article}
\usepackage{textcomp}
\usepackage[margin=1in]{geometry}
\usepackage[utf8]{inputenc}
\usepackage{color}
\usepackage{indentfirst} %Comment out for no first paragraph indent
\usepackage[parfill]{parskip}
\usepackage{setspace}
\usepackage{tikz}
\usepackage{amsmath}
\usepackage{amsfonts}
\usepackage{amssymb}
\usepackage{enumitem}
\usepackage{outlines}

\usepackage{fancyhdr}
\pagestyle{fancy}
\cfoot{\hyperlink{content}{\thepage}}
\lhead{}
\chead{}
\rfoot{}
\lfoot{}
\rhead{}
\renewcommand{\headrulewidth}{0pt}
\renewcommand{\footrulewidth}{0pt}

\usepackage{hyperref}
\hypersetup {
	colorlinks,
	citecolor=black,
	filecolor=black,
	linkcolor=black,
	urlcolor=black
}

\newcommand{\sepitem}{0pt} %Added room between items on the list, not including a list and its sublist
\newcommand{\seppar}{0pt} %Between items and lists overall

\setenumerate[1]{itemsep=\sepitem, parsep=\seppar}
\setenumerate[2]{itemsep=\sepitem, parsep=\seppar}
\setenumerate[3]{itemsep=\sepitem, parsep=\seppar}
\setenumerate[4]{itemsep=\sepitem, parsep=\seppar}

\newenvironment{outline*}
{
	\begin{outline}[enumerate]
	}
	{\end{outline}
}

\begin{document}

\title{Physics II: Electromagnetism}
\author{Avery Karlin}
\date{Spring 2016}
%\newcommand{\textbook}{}
\newcommand{\teacher}{Ali}

\maketitle
\newpage
\hypertarget{content}{\tableofcontents}
\vspace{11pt}
\noindent
%\underline{Primary Textbook}: \textbook\\
\underline{Teacher}: \teacher
\newpage

\section{Chapter 21 - Electric Charge}
\subsection{Charge}
\begin{outline*}
\1 Matter is composed of specific particles (electrons, protons, and neutrons), each with a property of charge (-ve, +ve, or 0)
\2 Electrons have a charge of $-1.6*10^{-19} C$ and a mass of $9.11*10^{-31} kg$
\2 Protons have a charge of $-1.6*10^{-19} C$ and a mass of $1.673*10^{-27} kg$
\2 Neutrons have a charge of $0 C$ and a mass of $1.674*10^{-27} kg$, such that the mass is approximately the sum of that of an electron and proton
\1 The Law of Charges state that like charges repel and unlike charges attract
\1 Quantization of Charge states that the charge of any body is equal to the multiple of the charge of an electron/proton
\1 Materials can be divided into conductors, insulators, and semiconductors
\2 The distance between the atomic radiuses of each of the atoms in the material determine the type, such that the further the distance, the less conductive, making conductors dense
\2 Conductors have electrons free to move from orbit to orbit, such as metals
\2 Insulators have a large energy gap ($E_g$), such that the electrons are bound to atoms, unable to jump/move, such as wood, glass, and rubber
\2 Semiconductors are not free or bound, such that they can be converted to either a conductor or insulator
\3 Addings impurities to the material through doping allows electrons to jump easier, such that it conducts better
\3 Increasing temperature increases electron energy, allowing jumps temporarily, while decreasing temperature does the opposite
\end{outline*}

\subsection{Charging}
\begin{outline*}
\1 Charging by friction is done by rubbing objects together, using the frictional force to force electron transfer to the material with greater attraction to electrons
\2 Valence electrons are the ones lost from the material
\2 Glass and silk are known for producing positively charged glass
\2 Plastic and fur are known for producing negatively charged plastic
\1 Charging by contact is done by contact between two non-insulators, causing the charge to move until equilibrium
\1 Charging by induction is done by placing a charged rod near a neutral conductor, such that the material is polarized into a seperation of charges, giving a temporary charge
\2 If the object is then split up in the case of a conductor, where the polarization is on a super-molecular scale, it can be charged
\2 If the object is an insulator, such that it is on a dipole molecular scale, it is temporary, but can allow it to be attracted to the object
\1 Electroscopes have two flat leaves connected to a rod, which is attached to the ball on top, such that it detects charge by charging by contact, after which the leaves seperate
\2 It can be decharged by attaching it to a grounding wire, attached to the Earth, which acts as a giant neutral body to neutralize the charges
\end{outline*}

\subsection{Coulomb's Law}
\begin{outline*}
\1 $F_e = \frac{k|q_1||q_2|}{r^2}$, where the direction of the vector is determined by the law of charges
\2 $k = 9 * 10^9 \frac{Nm^2}{C^2}$
\2 The overall force must be found before breaking it up into components, rather than in finding it as components first
\1 The ratio of electrostatic force to gravitational force on a molecular scale approximates as $10^{40}$, such that gravity can be ignored on that scale
\2 Based on the hydrogen atom, with a radius of 0.53 Angstrom (0.53 * $10^{-10} m)$
\end{outline*}

\section{Chapter 22 - Electric Field}
\subsection{Point Charges}
\begin{outline*}
\1 The electric field ($\vec{E}$) is related to the number of field lines per unit area, such that lines are drawn from positive to negative
\2 Electric fields are found by measuring the force on a positive test charge with a small enough size that it doesn't disrupt the surrounding field
\2 $\vec{E} = \frac{\vec{F}}{q_0} = \frac{k|q|}{r^2}$, where the direction is determined by the type of charge, where $q_0$ is the test charge, and q is the charge of the point mass producing the field
\1 Dipole moments are created by two point charges of opposite charge, seperated by some distance
\2 Dipole moment $(P) = qr$, where r is the distance between them, and q is the charge of each point, for cases where the charges are of equal, but opposite charge
\2 Thus, $\tau = P x E$ and $PE = -E \cdot P$ for dipole moments
\end{outline*}
\subsection{Solid Bodies}
\begin{outline*}
\1 $\vec{E} = \sum_i \frac{k\Delta q_i}{r_i^2} = \int \frac{k}{r^2}dq$
\2 $q = \lambda x, q = \sigma A, q = \rho V$ (depending on the dimensions of the solid body)
\2 If the density functions are non-uniform, the integral is taken to solve for q, with respect to each dimension variable 
\1 These calculations can be made simpler if one of the dimensions of the field cancels out, but a multiplier of some trig function must be added to the equation to remove that component
\1 For parallel plates, $\vec{E} = \frac{\sigma}{\epsilon_0}$, such that each plate is half of that value, where $\epsilon_0$ is the permittivity of free space, or $8.854 * 10^{-12}$
\end{outline*}

\end{document}
