\documentclass[11 pt, twoside]{article}
\usepackage{textcomp}
\usepackage[margin=1in]{geometry}
\usepackage[utf8]{inputenc}
\usepackage{color}
\usepackage{indentfirst} %Comment out for no first paragraph indent
\usepackage[parfill]{parskip}
\usepackage{setspace}
\usepackage{tikz}
\usepackage{amsmath}
\usepackage{amsfonts}
\usepackage{amssymb}
\usepackage{enumitem}
\usepackage{outlines}

\usepackage{fancyhdr}
\pagestyle{fancy}
\cfoot{\hyperlink{content}{\thepage}}
\lhead{}
\chead{}
\rfoot{}
\lfoot{}
\rhead{}
\renewcommand{\headrulewidth}{0pt}
\renewcommand{\footrulewidth}{0pt}

\usepackage{hyperref}
\hypersetup {
	colorlinks,
	citecolor=black,
	filecolor=black,
	linkcolor=black,
	urlcolor=black
}

\newcommand{\sepitem}{0pt} %Added room between items on the list, not including a list and its sublist
\newcommand{\seppar}{0pt} %Between items and lists overall

\setenumerate[1]{itemsep=\sepitem, parsep=\seppar}
\setenumerate[2]{itemsep=\sepitem, parsep=\seppar}
\setenumerate[3]{itemsep=\sepitem, parsep=\seppar}
\setenumerate[4]{itemsep=\sepitem, parsep=\seppar}

\newenvironment{outline*}
{
	\begin{outline}[enumerate]
	}
	{\end{outline}
}

\begin{document}

\title{Physics II: Electromagnetism}
\author{Avery Karlin}
\date{Spring 2016}
\newcommand{\textbook}{Fundementals of Physics by Halliday, Resnick, and Walker}
\newcommand{\teacher}{Ali}

\maketitle
\newpage
\hypertarget{content}{\tableofcontents}
\vspace{11pt}
\noindent
\underline{Primary Textbook}: \textbook\\
\underline{Teacher}: \teacher
\newpage

\section{Chapter 21 - Electric Charge}
\subsection{Charge}
\begin{outline*}
\1 Matter is composed of specific particles (electrons, protons, and neutrons), each with a property of charge (-ve, +ve, or 0)
\2 Charge is measured in Coulumbs, generally a unit derived from Amperes for current, such that 1 C = 1 A*s
\2 Electrons have a charge of $-1.6*10^{-19} C$ and a mass of $9.11*10^{-31} kg$
\2 Protons have a charge of $-1.6*10^{-19} C$ and a mass of $1.673*10^{-27} kg$
\2 Neutrons have a charge of $0 C$ and a mass of $1.674*10^{-27} kg$, such that the mass is approximately the sum of that of an electron and proton
\2 The signs for electrons and protons were chosen by Benjamin Franklin by arbitrary convention
\1 The Law of Charges state that like charges repel and unlike charges attract
\1 Quantization of Charge states that the charge of any body is equal to the multiple of the charge of an electron/proton, called the elementary charge
\2 Quantized quantities are those which can only have discrete values of specific integer multiples of some constant, rather than any value
\1 Conservation of Charge states that charge is conserved, in all bodies, including subatomic, nuclear, and large-scale
\2 Thus, in nuclear reactions, charge is either annihilated or pair-produced (producing both charges simultaneously)
\1 Materials can be divided into conductors, insulators, and semiconductors
\2 The distance between the atomic radiuses of each of the atoms in the material determine the type, such that the further the distance, the less conductive, making conductors dense
\2 Conductors have electrons free to move from orbit to orbit, such as metals or impure water
\3 It is noted that this creates the difference of charged conducting bodies, such that the charges isolate themselves on the outside surface when there is no external field, unlike insulators which are dispersed evenly throughout
\4 The charges are only uniformly dispersed on the surface of a conducting sphere, while they vary in other conducting bodies, or in conductors with an external field acting on it
\4 In non-spherical conductors, the charge tends to condense near sharp edges and points, increasing the electric field around it, causing ionization of the air around, called a corona discharge, and often leading to a lightening strike by creating a high voltage
\4 This is also the reason why during lightening strikes, it is suggested to go inside spherical conductors
\4 Thus, spherical conductors also have constant potential within, due to having no internal electric field
\3 For conductors in an external field, since there is no constant current within, the charges must reorganize themselves such that there is still no electric field inside, due to the charges counteracting it
\4 By extension, all conductors in an external field have a constant electric potential within
\3 Thus, conducting spheres act as a spherical shell due to all charges on the outer layer
\3 Superconductors are conductors without any hindrance to movement of elections
\2 Insulators have a large energy gap ($E_g$), such that the electrons are bound to atoms, unable to jump/move, such as wood, glass, rubber, or pure water
\2 Semiconductors are not free or bound, such that they can be converted to either a conductor or insulator, such as silicon or germanium
\3 Adding impurities to the material through doping allows electrons to jump easier, such that it conducts better
\3 Increasing temperature increases electron energy, allowing jumps temporarily, while decreasing temperature does the opposite
\end{outline*}

\subsection{Charging}
\begin{outline*}
\1 Charging by friction is done by rubbing objects together, using the frictional force to force electron transfer to the material with greater attraction to electrons
\2 Valence electrons are the ones lost from the material, specifically called conduction elections when able to be lost (in conductors)
\2 Glass and silk are known for producing positively charged glass
\2 Plastic and fur are known for producing negatively charged plastic
\1 Charging by contact is done by contact between two non-insulators, causing the charge to move until equilibrium
\1 Charging by induction is done by placing a charged rod near a neutral conductor, such that the material is polarized into a seperation of charges, giving a temporary/induced charge
\2 If the object is then split up in the case of a conductor, where the polarization is on a super-molecular scale, it can be charged
\2 If the object is an insulator, such that it is on a dipole molecular scale, it is temporary, but can allow it to be attracted to the object
\1 Electroscopes have two flat leaves connected to a rod, which is attached to the ball on top, such that it detects charge by charging by contact, after which the leaves seperate
\2 It can be discharged by attaching it to a grounding wire, attached to the Earth, which acts as a giant neutral body to neutralize the charges
\end{outline*}

\subsection{Coulomb's Law}
\begin{outline*}
\1 Coulomb's Law determines the electrostatic force of charged bodies, where electrostatic denotes stationary/negligible-movement charges
\1 $F_e = \frac{k|q_1||q_2|}{r^2}$, where the direction of the vector is determined by the law of charges
\2 $k = 9 * 10^9 \frac{Nm^2}{C^2} = \frac{1}{4\pi\epsilon_0}$
\2 The overall force must be found before breaking it up into components, rather than in finding it as components first
\1 The ratio of electrostatic force to gravitational force on a molecular scale approximates as $10^{40}$, such that gravity can be ignored on that scale
\2 Based on the hydrogen atom, with a radius of 0.53 Angstrom (0.53 * $10^{-10} m)$
\1 Shells of uniform charge density are found to have force interactions with charged point masses outside as if it was concentrated at the center
\2 Point masses inside the shell have no net electrostatic force acting on it
\end{outline*}

\section{Chapter 22 - Electric Field}
\subsection{Point Charges}
\begin{outline*}
\1 The electric field ($\vec{E}$) is a vector field, or a function of vectors at each point on the cartesian grid, allowing a measurement of the influence of a particle on any other particle at that point
\2 It is proportional to the number of field lines per unit area, such that high field line density means a stronger field
\3 Electric field lines are drawn from positive to negative
\2 Electric fields are found by measuring the force on a positive test charge with a small enough size that it doesn't disrupt the surrounding field, though it is noted that fields exist independently of the test charge
\1 $\vec{E} = \frac{\vec{F}}{q_0} = \frac{k|q|}{r^2}$, where the direction is determined by the type of charge, where $q_0$ is the test charge, and q is the charge of the point mass producing the field
\2 The unit of electric fields is N/C
\2 $\vec{E}_{net} = \sum_i \vec{E}_i$, or the electric field at some point/test charge due to a series of additional charges, such that superposition applies
\3 Superposition assumes that the charges remain in the same locations as when calculated individually, when together
\1 Electric dipoles are created by two point charges of equal magnitude, but opposite charge, seperated by some distance
\2 Dipole moment $(\vec{P}) = qr$, where r is the distance between them, and q is the charge of each point, where the vector goes from negative to positive
\3 The direction of the dipole is based on the direction of the electric field vectors acting on the point charge for any point on the dipole line
\3 This is able to be expressed by the asymmetrical charge distribution of even more complicated molecules
\2 $\tau_{external \to dipole} = P x E$, where E is the uniform external electric field
\2 $PE = -E \cdot P$ for dipoles in an external electric field, where E is the uniform external electric field
\2 $E_{dipole} = \frac{kP}{r^3}$, where r is the distance from the center of the dipole, where the point is on the dipole axis as r becomes far greater than the distance between the dipole charges
\3 On the other hand, E is proportional to $\frac{1}{r^3}$ at all points on the dipole axis, rather than just at a large distance
\3 This proportionality is due to the charges canceling each other out faster as the distance gets further
\2 Dipoles can be induced by an electric field, even in a nonpolar molecule, or single atom, such that an induced dipole moment forms in the same direction as the field
\3 The molecule/atom is then said to have been polarized temporarily by the field
\1 Sparks are due to electrical breakdown as electrons are forced out from atoms as the electrical field gets past the critical value ($E_c$)
\2 These electrons create blue light when they collide with atoms after being separated from their original atoms
\end{outline*}

\subsection{Solid Bodies}
\begin{outline*}
\1 $\vec{E} = \sum_i \frac{k\Delta q_i}{r_i^2} = \int \frac{k}{r^2}dq$
\2 $q = \lambda x, q = \sigma A, q = \rho V$ (depending on the dimensions of the solid body)
\2 If the density functions are non-uniform, the integral is taken to solve for q, with respect to each dimension variable 
\2 These calculations can be made simpler if one of the dimensions of the field cancels out, but a multiplier of some trig function must be added to the equation to remove that component
\1 For parallel, nonconducting plates, $\vec{E} = \frac{\sigma}{\epsilon_0}$, such that the field from each plate is half of that value
\2 $\epsilon_0$ is the permittivity constant of free space, or $8.854 * 10^{-12} C^2/Nm^2$, and $\sigma$ is the charge density of the plates
\2 This is due to all field lines moving in a different direction being canceled, assuming infinitely long parallel plates (or assumed to exist far from the edge of the plate, creating a uniform electric field
\2 Uniform electric fields are those with the same magnitude and direction at every point
\2 The fact that parallel plates have double the field of a single plate can also be thought to be due to all the charge moving to the edge, such that double the charge is present, rather than split on both edges
\3 This is not due to superposition, due to the charges when the plates are together, such that it must be calculated when together, rather than the sum of apart
\end{outline*}

\section{Chapter 23 - Electric Flux}
\subsection{Electric Flux}
\begin{outline*}
\1 Electric flux is the measure of flow of the field through an area, or the  number of electric field lines passing through an area, such that $\Phi_E = \vec{E} \cdot \vec{A}_n = \vec{E}\vec{A}_ncos(\theta)$, where $\vec{A}_n$ is the normal to the surface, perpendicular and with a magnitude equal to the area
\2 For an enclosed object, lines entering the surface of the object is negative, outward is positive
\2 Thus, positively charged objects have lines leaving but not entering, negative objects have vice versa 
\1 $\Phi = \vec{E} \cdot \vec{A}_n$ for uniform surfaces
\2 $\Phi = \lim_{\Delta A_n \to 0} \sum_i \vec{E}_i \cdot \Delta A_n = \oint \vec{E} \cdot dA_n$
\2 Electric flux is a scalar with $N*m^2$ as the unit
\end{outline*}

\subsection{Gauss's Law}
\begin{outline*}
\1 Gauss’s Law is used to calculate the graviational field at a point, by creating a gaussian surface including that point
\2 Gaussian surfaces must be imaginary, closed (compact/effectively-continuous without boundary in any direction), 3D surfaces, with a constant field throughout
\2 They also must be symmetrical objects to minimize calculations
\2 The electric field must also be parallel to the normal of the tangent plane, $A_N$, on the surface A
\1 Gauss's Law states that $\Phi_e \epsilon_0 = q_{enc}$, where $q_{enc}$ is the charge of the body inside the surface creating the field
\2 Gauss's Law is able to be derived from Coulumb's Law, such that $\Phi_e = \oint \vec{E} dA = \vec{E} \oint dA = \frac{kq_{enc}}{r^2} (4\pi r^2) = \frac{q_{enc}}{\epsilon_0}$
\2 By the definition of a surface integral, $\oint dA$ equals the surface area of the Gaussian surface
\2 This law only holds true in this form for charges in either a vacuum or air (effect of air is negligible)
\2 The electric field can be influenced by charges outside the surface, but the charge calculated is only that within the surface, due to outside charges not contributing to the net electric flux of the surface, but rather just adding field lines that pass fully through
\1 Gauss's Law can be applied to an infinitely long line of charge (or line of charge far from the ends), such that a cylinder can be constructed, since the flux through the top and bottom is parallel, and thus 0, such that the flux through the curved section is counted only
\2 This is used to derive the field of an infinite (or middle of the) sheet of charge at some distance as well, by creating a cylinder in the center around some area of the sheet of some height, thus the flux of the caps of the cylinder are the only ones used
\3 This is also used to derive the field between two parallel plates, due to the density doubling by the movement of charges toward the single edge near the center, rather than divided between the two edges when isolated, resulting in the field doubling
\3 This is also used to calculate the external field outside a conductor with nonuniform charge distribution, due to it being uniform and the surface flat over a tiny area
\4 On the other hand, it must be a conductor, or the charge would not be isolated in a flat layer on the edge of the surface
\2 It can also be used for spherical symmetry about some body, allowing the shell charge calculations to be shortened, as well as calculate the field within for nonconducting spheres by the charge enclosed
\2 Thus, bodies can be created without complete uniform field at the surface, such that the field is parallel to all surfaces that are not uniform, and thus can be not counted
\2 On the other hand, symmetry cannot be applied to the edges of an object, due to the edge effect/fringing causing the electric field lines to curve
\1 For varying charge density, $q_{enc} = \int \rho dV$ is substituted in, where dV can be converted to cartesian, cylindrical, or spherical depending
\1 Gauss's Law is also used to prove that conductors have the charge on the edges, due to if charge existed within, an electric field would, creating a constant internal current which isn't the case
\2 The same reasoning can be used to show that it is only in the outer surface if there exists a cavity within, rather than the cavity surface as well
\2 It is also used to show that a conductor works equally to an infinitely thin insulating shell
\end{outline*}
\section{Chapter 24 - Electric Potential}
\begin{outline*}
\1 Electric potential energy can be defined in a system from infinity as a reference as the negative work done by electrostatic force to move a particle from infinity towards some stationary, unaffected second charge ($\Delta U = U_r - U_\infty = U_r = -W_{\infty \to r}$)
\2 The electric potential energy can also be thought of as the work done by an external action to counteract the electric field, and move the particle from infinity to some stationary final setup
\2 Thus, the reference configuration is some system where each charged particle is an infinite distance from each other, such that reference potential energy is 0
\2 Electrostatic force is conservative, and thus path independent
\2 Energy is often measured in electron-volts, such that it is the energy required to move an electron through one volt (1.6 * $10^-19$ J)
\1 Electric potential (V) is a scalar quantity not dependent on the particle being moved, but rather existing at each point in a field, equal to the potential per unit charge
\2 Thus, $V_rq_0 = U_0$, where $q_0$ is the test charge, and r is the distance from the other particle
\2 As a result, it can be found that $V = \frac{kq}{r}$ for a field produced by a point charge, such that positive charges produce positive potential and vice versa
\3 By the superposition principle for potential energy, the potential at some point due to a group of charges is found by $V = \sum_i \frac{kq_i}{r}$
\1 $\Delta V$ is the electric potential difference, also called the voltage, and measured in Volts, or J/C
\2 $\Delta V = \frac{\Delta U}{q} = \frac{-W}{q}$, such that it can be calculated for point charge systems by the work done
\2 By extension, $\Delta V = \frac{-\int^f_i \vec{F}\cdot d\vec{s}}{q} = -\int^f_i \vec{E} \cdot d\vec{s}$, allowing it to be calculated by the movement of a particle in both an external field and a system of particles
\3 In the external field, the initial point can be thought to be the point of zero potential, such that $\Delta V = V_f$
\3 As a result, for fields created by point charges, since $V(\infty) = 0, V(r) = \int^r_\infty -E \cdot dr$
\4 In this case, the electric field must be found, not just at the point itself, but all points from there to infinity, such that if it is measured by multiple functions at various regions, must be broken up
\4 This function to find V is commonly used for symmetrical objects by which Gauss's Law can be applied, while the defining function is more commonly used for non-symmetrical ones
\3 This is used to prove that the potential within a conducting sphere creates the same potential within the sphere, due to no electric field inside
\2 For fields generated by dipoles, $V = \frac{kq(r_- - r_+)}{r_-r_+}$
\3 As r becomes far greater than the distance between the charges, $r_-$ and $r_+$ becomes parallel, such that $r_-r_+ = r^2$ and $r_- - r_+ = dcos\theta$, where $\theta$ is the angle between the dipole axis and each r
\4 $\theta$ is also noted to be the angle between the radius from the center of the dipole and the dipole axis
\3 Thus, $V = \frac{kqdcos(\theta)}{r^2} = \frac{kpcos(\theta)}{r^2}$ for dipoles as the distance from the dipole is far greater than the distance between the charges
\2 For a continuous charge distribution, $V = \int \frac{kdq}{r}$
\1 The work moving a series of charges is equal to the sum of the work to move each charge with respect to the work of the charges moved previously
\1 By the relationship of potential energy and force, $E = -\vec{\nabla} V(\vec{r})$
\1 Equipotential regions/points are those where the electric potential is equal, either a real or imaginary surface/region/points
\2 By extension, points are equipotential if the work done moving a charge from one to the other is 0, such that either electric field is 0 or the electric field is perpendicular to the movement of the charge
\2 Equipotential surfaces can be found by the IVT, such that if on a line between two points, the value moves past the desired potential, there must be some point such that it is equipotential
\end{outline*}

\section{Chapter 25 - Capacitance}
\begin{outline*}
\1 Capacitors are devices which can store electrical energy, made up of two conductor plates, represented by -$\|$- in a circuit schematic, conventionally in the form of a parallel plate capacitor
\2 Fully charged capacitors have both plates of equal charge, but opposite sign, with some region between the plates creating a field within, such that there is a voltage present, represented as V, rather than $\Delta V$ by convention
\3 The plates must be conductors, such that they are equipotential surfaces, creating parallel potential gradient
\2 Capacitance is the proportionality constant between the fully charged charge of each plate and the voltage, such that q = CV, where q is the charge of each plate
\3 Capacitance is measured in Farads, or C/V
\1 Capacitors are charged by the creation of an electric circuit, or a path allowing the flow of charge, with a battery to produce a specific electric potential between the terminals
\2 The terminals are the locations of the battery where charge can leave and enter into the circuit, but not directly to the other terminal
\2 Batteries are represented as a capacitor, with the terminal of higher potential as a longer line, and positive, and the terminal of lower potential as a shorter line, negative
\2 Switches are used to prevent current, by disconnecting the high and low potential of the battery, such that the charges are static when open until it is closed
\2 When the circuit is completed, equal magnitudes of charge are produced on both sides of the capacitor by the electric field created by the battery, until the potential of each capacitor plate is equal to that created by the battery
\3 It is generally assumed that charge cannot move through the capacitor, such that it cannot discharge unless the circuit has a mechanism to connect the plates
\3 Realistically, the battery does not bring the charges to the plates through the whole circuit, but rather forces charges to move from within the plate to the edges, only moving those closest to the edge to begin with
\1 For a parallel plate capacitor, $C = \frac{q}{Ed}$, where d is the distance between the plates, such that q can be found by Gauss's Law for some Gaussian cylinder such that the top field is equal to the uniform field within the plates
\2 Thus, $C = \frac{\epsilon_0EA}{Ed} = \frac{\epsilon_0A}{d}$
\2 Cylindrical capacitors can have the Gaussian surface constructed around the inner cylinder, such that $E = \frac{q}{\epsilon_02\pi rL}$, where L is the height of the cylinder, varying with r
\3 Thus, $V = \int^b_a Edr$, where a is the radius of the inner, b is the radius of the outer, such that $C = \frac{2\pi\epsilon_0L}{ln(b/a)}$
\2 Spherical capacitors similarly have $E = \frac{q}{4\pi\epsilon_0r^2}$, such that an integral is similarly found for V, to get $C = \frac{4\pi\epsilon_0ab}{b-a}$, where b is the outer radius, a is the inner
\3 A single sphere has a capacitance, where the outer sphere has an infinite radius, approximated by some object in which the field lines end, such as walls, such that $C = 4\pi\epsilon_0R$, where R is the radius of the inner sphere
\1 Combinations of capacitors in circuits can be replaced by an equivalent capacitor to simplify it
\2 In parallel, since the voltage of each is the same, the charge of the equivalent capacitor is equal to the sum of the individual charges
\3 The equivalent capacitance is also equal to the sum of the individual capacitance
\2 In series, 
\end{outline*}

\section{Chapter 26 - Current and Resistance}
\subsection{Current}
\begin{outline*}
\1 Electric currents are moving charges, such that there is a net flow of charge through the surface, considered a scalar quantity, though often an arrow is added to show the general direction of current flow
\2 As a result, the random movement of charge in electrons within a conductor without internal potential difference or the flow of neutral molecules are not considered currents
\3 It is noted that questions which ask specifically for the current of negative or positive charge among a neutral molecule flow ignore the lack of net charge flow
\2 Currents are generally assumed to be steady (not varying with time) movements of conduction electrons in metallic conductors, where there is assumed to be no resistance in the wire themselves
\3 Due to conductors having equal potential within, a battery must be added to create a forced potential, creating a current within
\2 Currents are measured in Amperes (A)
\2 $I = \frac{dq}{dt}$
\1 Due to conservation of charge, current is the same for any cross-section within the conductor
\1 The direction of current is drawn as the movement of positive charges, although the actual flow of charges is negative, in the opposite direction, due to the effect generally assumed to be the same
\2 In cases where it would change the effect of the current, this convention is not used
\1 Charge density is the charge per unit area in the cross section, as a vector with the same direction as the current, such that $I = \int \vec{J} \cdot d\vec{A}$, where $\vec{A}$ is the normal of the area
\2 Charge density can be represented by streamlines, such that the spacing between the lines/the density of the lines represents the charge density at the cross-section
\1 Drift velocity is the velocity of the electrons in the direction of the current, minute compared to the random motion of the charges in both conductors with and without current ($10^{-5}$ m/s, rather than $10^6$)
\2 This is also vastly slower than the speed of the current itself, often closer to $10^9$, the result of the slight random motion net movement in the direction of the current
\2 
\end{outline*}

\subsection{Resistance}
\begin{outline*}

\end{outline*}

\section{Chapter 27 - Circuits}
\begin{outline*}
\1 Circuits can be powered either by a charged capacitor or a charge pump, the former temporarily until it discharges, the latter an object which produces a steady current by maintaining a voltage
\2 Charge pumps are also called emf devices, producing an emf, or electromotive force, often in the form of a battery and electric generator, or less commonly in the form of solar, fuel cells, thermophiles, or biological generators
\2 EMF is represented as an arrow from the positive to negative terminals of the device, with internal chemistry in the device to create current unless a circuit joins the terminals
\2 The EMF device forces charges in the EMF direction within the battery to create a constant potential, converting some other form of energy to produce the electrical energy, such as mechanical energy in generators
\2 $EMF = \frac{dW}{dq}$, or the work that must be done on each charge by the device to move a unit positive charge to the high voltage terminal, measured in V
\3 This is due to the need for the battery to move an equal amount of charge that is traveling in any given amount of time through a cross-section of the circuit, through the battery
\2 These EMF devices produce electrical energy, providing the energy to allow the movement of charges as current, allowing it to be used in other devices and converted/used as other forms of energy
\1 Ideal EMF devices are those without any internal resistance on the circuit, such that the voltage of the battery is equal to the emf
\2 Real EMF devices are those with internal resistance in the battery, such that when not connected, the voltage is equal to the emf, but when connected to the circuit, is not equal
\1 Current in a single-loop circuit can be calculated either by conservation of energy, or by potential
\2 
\end{outline*}

\end{document}
