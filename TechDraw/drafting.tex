\documentclass[11 pt, twoside]{article}
\usepackage{textcomp}
\usepackage[margin=1in]{geometry}
\usepackage[utf8]{inputenc}
\usepackage{color}
\usepackage{indentfirst} %Comment out for no first paragraph indent
\usepackage[parfill]{parskip}
\usepackage{setspace}
\usepackage{tikz}
\usepackage{amsmath}
\usepackage{amsfonts}
\usepackage{amssymb}
\usepackage{enumitem}
\usepackage{outlines}

\usepackage{fancyhdr}
\pagestyle{fancy}
\cfoot{\hyperlink{content}{\thepage}}
\lhead{}
\chead{}
\rfoot{}
\lfoot{}
\rhead{}
\renewcommand{\headrulewidth}{0pt}
\renewcommand{\footrulewidth}{0pt}


\usepackage{hyperref}
\hypersetup {
	colorlinks,
	citecolor=black,
	filecolor=black,
	linkcolor=black,
	urlcolor=black
}

\newcommand{\sepitem}{0pt} %Added room between items on the list, not including a list and its sublist
\newcommand{\seppar}{1pt} %Between items and lists overall

\setenumerate[1]{itemsep=\sepitem, parsep=\seppar}
\setenumerate[2]{itemsep=\sepitem, parsep=\seppar}
\setenumerate[3]{itemsep=\sepitem, parsep=\seppar}
\setenumerate[4]{itemsep=\sepitem, parsep=\seppar}

\newenvironment{outline*}
{
	\begin{outline}[enumerate]
	}
	{\end{outline}
}

\newcommand{\foot}[1]{\hyperlink{#1}{$_#1$}}

\begin{document}

\title{Technical Drawing}
\author{Avery Karlin}
\date{Spring 2016}
%\newcommand{\textbook}{}
\newcommand{\teacher}{Griffith}

\maketitle
\newpage
\hypertarget{content}{\tableofcontents}
\vspace{11pt}
\noindent
%\underline{Primary Textbook}: \textbook\\
\underline{Teacher}: \teacher
\newpage

\section{Types of Lines}
\begin{outline*}
\1 Object lines are dark, thick lines, used to show the visible edges of the object in the drawing
\2 Object lines must be the darkest within the drawing
\2 Hidden lines, on the other hand, are dashed lines to show edges of the object that would not noramally be visible, drawn medium and dark
\1 Extension lines are drawn from the boundaries of an edge, perpendicular, without touching the actual edge, used with dimension lines to show measurements, drawn dark and thin
\2 Dimension lines are drawn between extension lines, with the measurement drawn in the center of the line, with arrows on both sides, pointing to the end of the extension lines, drawn dark and thin
\2 Leader lines are similarly just arrows, able to be bent, but still straight, and are used to state notes or other information, drawn dark and thin
\1 Center lines illustrate symmetry or circles/holes in an object, drawn for a sideview as alternating long and short lines, drawn within a circle as short lines forming a cross at the center, with long lines going outward, drawn dark and thin
\2 Break lines are jagged lines, used to seperate the general part of the object, with a more detailed section used to show the specific characteristics, versus the general shape, drawn dark and thin
\2 Cutting plane lines are drawn as long lines, with arrows on the end, both pointing in the same direction, showing how the object was cut for a sectional view, drawn dark and thin
\2 Section lines are a gradient of $45^o$ lines to show where an object was physically sliced within a section, drawn dark and thin
\1 Phantom lines are drawn as alternating long-short lines to show an alternate position of the object, used to display possible movement, in the place of object lines, drawn dark and medium
\1 Construction lines are used to aid with the drawing, drawn of medium thickness and grey, erased after the drawing
\end{outline*}

\section{Sketches}
\subsection{Definition}
\begin{outline*}
\1 Sketches are drawings created without any drawing tools, purely freehand, not drawn to scale, and used for the development of a design, but not as the final design
\2 They are intended only for personal use, rather than to actually build the object, and are not saved after
\1 Sketches must continue in a straight path, generally based with short, overlapping strokes, rather than a ridged and stiff, or curved line
\2 They are able to be drawn somewhat broken, due to small gaps between strokes, though this is not ideal
\end{outline*}
\subsection{Cross-Section Circles/Arcs}
\begin{outline*}
\1 Circles are drawn first by creating the square in which it will be inscribed in, and then marking the midpoints of each of the edges
\2 Then, diagonals of the square are drawn, and the approximate interceptions of the diagonals with the circles are marked
\2 Finally, the circle itself can be drawn, though it must be made sure that the circle is tangent to the midpoints of the sides, rather than moving too quickly toward the next midpoint
\1 Arcs are drawn by drawing a square with edges of the length of the radius, and drawing a quarter of a circle within that square similarly
\1 Ellipses are drawn similarly to a circle, except with a circumscribed rectangle instead of a square
\end{outline*}
\subsection{Isometric Sketches}
\subsubsection{Theory}
\begin{outline*}
\1 Pictorial drawings are those which show the 3 spatial dimensions within one drawing
\1 Axonometric drawings are pictorial sketches based on axes, which remain parallel throughout the drawing
\2 Isometric sketches are those with the X (width), Y (depth), Z (height) axes are $120^o$ angles from each other
\3 This provides an equally emphasized view of the right view between the Y and Z axes on the right side of the sketch, front between X and Z, and top between X and Y
\3 The Z axis is drawn downward from the corner closest to the viewer, while the X and Y are drawn upward from there
\2 Dimetric sketches have $150^o$ between X and Y and $105^o$ between the other sets of axes, to emphasize the front and right views
\2 Trimetric sketches have $105^o$ between X and Z, $120^o$ between X and Y, and $135^o$ between Y and Z, to provide a clear top view, long right view, and shorter front view
\1 Oblique drawings are pictorial sketches with the front view drawn flat and unangled, with the Y axis at any particular angle
\1 Perspective drawings are pictorial sketches drawn such that parallel lines converge at vanishing points, attempting to simulate human vision of objects
\2 These can have one vanishing point in the upper-right hand corner, two in the upper corners, or three (the third on the negative y-axis)
\2 All edges which go in the general direction of the vanishing point are made to converge at the point
\end{outline*}
\subsubsection{Methodology}
\begin{outline*}
\1 General drawings begin by drawing a rectangular prism, after which the edges of the object which touch the outside of the prism are drawn
\2 After, the inner sections and non-isometric sections of the drawing are drawn, where non-isometric sections means those not parallel to the isometries
\2 Unneeded lines are then drawn after, including those of the rectangular prism itself
\1 Curves are drawn by first constructing the rectangular prism around the curve, in which the cylinder is inscribed within
\2 After, the radius to some point on the curve other than the endpoints is drawn
\2 Finally, the curve itself is drawn, making sure to draw the curve tangential to the points, not curving at too quick a pace, before removing surplus lines
\end{outline*}

\subsection{Orthographic Sketches}
\begin{outline*}
\1 Orthographic projections, also called multiview drawings, are representations of three dimensional objects in two dimensions
\2 This allows the features of an object to be projected onto an imaginary plane
\1 The miter line method places the top face in the upper left, front on bottom left, right on bottom right, and the miter line on the top right
\2 The miter line is drawn at $45^o$, used to project features from the right side to the top side
\3 In most sketches, this is erased at the end, and an isometric drawing is often drawn in its place
\2 The sides of the drawing must be equally spaced with lengths of adjecent sides equal
\2 Hidden lines for each side must also be found and drawn as features that cannot be seen on the side, but are present on the projection
\end{outline*}
\section{Lettering}
\begin{outline*}
\1 Lower case and upper case letters must never be mixed, and all must be uniform in line thickness and space between
\1 Guidelines should be used at the top and bottom of the letters to keep the size constant, generally with a guideline in the middle as well for middle letter characteristics
\2 Guidelines are drawn in the same form as construction lines
\end{outline*}

\section{Technical Drawing Setup}
\subsection{Page Setup}
\begin{outline*}
\1 The page is originally taped at the corners by masking tape, such that it is parallel to the edges of the drafting table precisely 
\1 First, the border must be drawn 0.5 inches from the edge of the paper all around the page, followed by the production of the title block on the bottom of the page
\1 The object block is then drawn, first drawing the construction line diagonals from the edges of the upper border to the top edges of the title block, such that the center of the page is found
\2 A construction line is then drawn through the center, parallel to the edges of the page, equal to the length and width of the object, bisected by the center
\2 These lines are then used to construct the box of construction lines, such that the lines intersect the center of each side, in which the object is drawn
\2 Finally, the diagonals and parallel construction lines are erased
\end{outline*}

\subsection{Title Block Setup}
\begin{outline*}

\end{outline*}

\section{Drafting Tool Usage}
\begin{outline*}

\end{outline*}

\end{document}
