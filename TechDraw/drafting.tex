\documentclass[11 pt, twoside]{article}
\usepackage{textcomp}
\usepackage[margin=1in]{geometry}
\usepackage[utf8]{inputenc}
\usepackage{color}
\usepackage{indentfirst} %Comment out for no first paragraph indent
\usepackage[parfill]{parskip}
\usepackage{setspace}
\usepackage{tikz}
\usepackage{amsmath}
\usepackage{amsfonts}
\usepackage{amssymb}
\usepackage{enumitem}
\usepackage{outlines}

\usepackage{fancyhdr}
\pagestyle{fancy}
\cfoot{\hyperlink{content}{\thepage}}
\lhead{}
\chead{}
\rfoot{}
\lfoot{}
\rhead{}
\renewcommand{\headrulewidth}{0pt}
\renewcommand{\footrulewidth}{0pt}


\usepackage{hyperref}
\hypersetup {
	colorlinks,
	citecolor=black,
	filecolor=black,
	linkcolor=black,
	urlcolor=black
}

\newcommand{\sepitem}{0pt} %Added room between items on the list, not including a list and its sublist
\newcommand{\seppar}{1pt} %Between items and lists overall

\setenumerate[1]{itemsep=\sepitem, parsep=\seppar}
\setenumerate[2]{itemsep=\sepitem, parsep=\seppar}
\setenumerate[3]{itemsep=\sepitem, parsep=\seppar}
\setenumerate[4]{itemsep=\sepitem, parsep=\seppar}

\newenvironment{outline*}
{
	\begin{outline}[enumerate]
	}
	{\end{outline}
}

\newcommand{\foot}[1]{\hyperlink{#1}{$_#1$}}

\begin{document}

\title{Technical Drawing}
\author{Avery Karlin}
\date{Spring 2016}
%\newcommand{\textbook}{}
\newcommand{\teacher}{Griffith}

\maketitle
\newpage
\hypertarget{content}{\tableofcontents}
\vspace{11pt}
\noindent
%\underline{Primary Textbook}: \textbook\\
\underline{Teacher}: \teacher
\newpage

\section{Types of Lines}
\begin{outline*}
\1 Object lines are dark, thick lines, used to show the visible edges of the object in the drawing
\2 Object lines must be the darkest within the drawing
\2 Hidden lines, on the other hand, are dashed lines to show edges of the object that would not noramally be visible, drawn medium and dark
\1 Extension lines are drawn from the boundaries of an edge, perpendicular, without touching the actual edge, used with dimension lines to show measurements, drawn dark and thin
\2 Dimension lines are drawn between extension lines, with the measurement drawn in the center of the line, with arrows on both sides, pointing to the end of the extension lines, drawn dark and thin
\2 Leader lines are similarly just arrows, able to be bent, but still straight, and are used to state notes or other information, drawn dark and thin
\1 Center lines illustrate symmetry or circles/holes in an object, drawn for a sideview as alternating long and short lines, drawn within a circle as short lines forming a cross at the center, with long lines going outward, drawn dark and thin
\2 Break lines are jagged lines, used to seperate the general part of the object, with a more detailed section used to show the specific characteristics, versus the general shape, drawn dark and thin
\2 Cutting plane lines are drawn as long lines, with arrows on the end, both pointing in the same direction, showing how the object was cut for a sectional view, drawn dark and thin
\2 Section lines are a gradient of $45^o$ lines to show where an object was physically sliced within a section, drawn dark and thin
\1 Phantom lines are drawn as alternating long-short lines to show an alternate position of the object, used to display possible movement, in the place of object lines, drawn dark and medium
\1 Construction lines are used to aid with the drawing, drawn of medium thickness and grey, erased after the drawing
\end{outline*}

\section{Sketches}
\subsection{Definition}
\begin{outline*}
\1 Sketches are drawings created without any drawing tools, purely freehand, not drawn to scale, and used for the development of a design, but not as the final design
\2 They are intended only for personal use, rather than to actually build the object, and are not saved after
\1 Sketches must continue in a straight path, generally based with short, overlapping strokes, rather than a ridged and stiff, or curved line
\2 They are able to be drawn somewhat broken, due to small gaps between strokes, though this is not ideal
\end{outline*}
\subsection{Cross-Section Circles/Arcs}
\begin{outline*}
\1 Circles are drawn first by creating the square in which it will be inscribed in, and then marking the midpoints of each of the edges
\2 Then, diagonals of the square are drawn, and the approximate interceptions of the diagonals with the circles are marked
\2 Finally, the circle itself can be drawn, though it must be made sure that the circle is tangent to the midpoints of the sides, rather than moving too quickly toward the next midpoint
\1 Arcs are drawn
\1 Ellipses are drawn
\end{outline*}
\subsection{Isometric Sketches}
\begin{outline*}

\end{outline*}

\end{document}
