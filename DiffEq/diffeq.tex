\documentclass[11 pt, twoside]{article}
\usepackage{textcomp}
\usepackage[margin=1in]{geometry}
\usepackage[utf8]{inputenc}
\usepackage{color}
%\usepackage{indentfirst} %Comment out for no first paragraph indent
\usepackage[parfill]{parskip}
\usepackage{setspace}
\usepackage{tikz}
\usepackage{amsmath}
\usepackage{amsfonts}
\usepackage{amssymb}

%\usepackage{outlines}

\usepackage{hyperref}
\hypersetup {
	colorlinks,
	citecolor=black,
	filecolor=black,
	linkcolor=black,
	urlcolor=black
}

%\newenvironment{outline*}
%{
%	\begin{outline}[enumerate]
%	\setlength{\itemsep}{0pt}
%	\setlength{\parsep}{0pt}
%	}
%	{\end{outline}
%}

\begin{document}

\title{Differential Equations}
\author{Avery Karlin}
\date{Spring 2016}
%\newcommand{\textbook}{}
\newcommand{\teacher}{Stern}

\maketitle
\newpage
\tableofcontents
\vspace{11pt}
\noindent
%\underline{Primary Textbook}: \textbook\\
\underline{Teacher}: \teacher
\newpage

\section{Introduction}
Coupled systems are those defined as requiring being solved together, such as $$R' = aR + bJ, J' = cR + bJ,$$ rather than equations containing only their own function, which would be uncoupled.

Linear differential equations are defined as equations with constants as the multiplier of each function derviative.

First order differential equations means that the highest derivative in the equation is the first derivative, and is extended as such.

For some identically restricted, coupled functions, $$R' = aR + bJ, J' = bR + aJ,$$ where R(0) = $R_0$ and J(0) = $J_0$, where $a < 0, b > 0$, if $|a| > |b|$, we can graph the phase plane of R on the x-axis, J on the y-axis, such that for all possible functions, it moves toward the stable node (0, 0). These stable nodes don't need to be a point, but rather can be a curve of some sort. If $|a| < |b|$, if the initial points are above R = -J, it moves in a parabolic fashion until asymptotic to R = J on the positive side. If below R = -J, it moves similarly, except asymptotic to R = J on the negative side. Finally, if $|a| = |b|$, the function moves infinitely in a circle around the origin, with the radius determined by the initial point.

\section{Background on $\mathbb{R}$}

\end{document}
