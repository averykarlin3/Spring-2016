\documentclass[11 pt, twoside]{article}
\usepackage{textcomp}
\usepackage[margin=1in]{geometry}
\usepackage[utf8]{inputenc}
\usepackage{color}
%\usepackage{indentfirst} %Comment out for no first paragraph indent
\usepackage[parfill]{parskip}
\usepackage{setspace}
\usepackage{tikz}
\usepackage{amsmath}
\usepackage{amsfonts}
\usepackage{amssymb}

%\usepackage{outlines}

\usepackage{hyperref}
\hypersetup {
	colorlinks,
	citecolor=black,
	filecolor=black,
	linkcolor=black,
	urlcolor=black
}

%\newenvironment{outline*}
%{
%	\begin{outline}[enumerate]
%	\setlength{\itemsep}{0pt}
%	\setlength{\parsep}{0pt}
%	}
%	{\end{outline}
%}

\begin{document}

\title{Differential Equations}
\author{Avery Karlin}
\date{Spring 2016}
%\newcommand{\textbook}{}
\newcommand{\teacher}{Stern}

\maketitle
\newpage
\tableofcontents
\vspace{11pt}
\noindent
%\underline{Primary Textbook}: \textbook\\
\underline{Teacher}: \teacher
\newpage

\section{Introduction}

\subsection{Definitions}

\textbf{Differential equations} are relations between one or more unknown functions an a finate amount of their derivatives, along with a certain number of independent variables. Most are not solvable, and do not have approximations for more than a minute region of the function.

\textbf{Linear differential equations} are defined as equations with algebraic functions as the multiplier of each function derviative and each derivative with a degree/power of 1, such that all linear equations are able to be solved.

\textbf{First order differential equations} means that the highest derivative in the equation is the first derivative, and is extended as such.

\textbf{Coupled systems} are those defined as requiring being solved together, such as $$X' = aX + bY, Y' = cX + bY,$$ rather than equations containing only their own function, which would be uncoupled.

\textbf{Ordinary Differential Equations} are those where all unknown functions depend on the same, single independent variable, while \textbf{Partial Differential Equations} are those where the functions have multiple independent variables. In addition though, vector function differential equations can fall into either catagory, action more as a system of differential equations, rather than a single one.

Finally, in general, either an initial condition is set allowing it to be solved for a single solution, or a family of solution must be considered as the answer.

\subsection{Example}

For some identically restricted, coupled functions, $$R' = aR + bJ, J' = bR + aJ,$$ where R(0) = $R_0$ and J(0) = $J_0$, where $a < 0, b > 0$, if $|a| > |b|$, we can graph the phase plane of R on the x-axis, J on the y-axis, such that for all possible functions, it moves toward the stable node (0, 0). These stable nodes don't need to be a point, but rather can be a curve of some sort. If $|a| < |b|$, if the initial points are above R = -J, it moves in a parabolic fashion until asymptotic to R = J on the positive side. If below R = -J, it moves similarly, except asymptotic to R = J on the negative side. Finally, if $|a| = |b|$, the function moves infinitely in a circle around the origin, with the radius determined by the initial point.

\end{document}
