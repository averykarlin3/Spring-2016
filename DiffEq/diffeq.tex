\documentclass[11 pt, twoside]{article}
\usepackage{textcomp}
\usepackage[margin=1in]{geometry}
\usepackage[utf8]{inputenc}
\usepackage{color}
%\usepackage{indentfirst} %Comment out for no first paragraph indent
\usepackage[parfill]{parskip}
\usepackage{setspace}
\usepackage{tikz}
\usepackage{amsmath}
\usepackage{amsfonts}
\usepackage{amssymb}

%\usepackage{outlines}

\usepackage{fancyhdr}
\pagestyle{fancy}
\cfoot{\hyperlink{content}{\thepage}}
\lhead{}
\chead{}
\rfoot{}
\lfoot{}
\rhead{}
\renewcommand{\headrulewidth}{0pt}
\renewcommand{\footrulewidth}{0pt}


\usepackage{hyperref}
\hypersetup {
	colorlinks,
	citecolor=black,
	filecolor=black,
	linkcolor=black,
	urlcolor=black
}

%\newenvironment{outline*}
%{
%	\begin{outline}[enumerate]
%	\setlength{\itemsep}{0pt}
%	\setlength{\parsep}{0pt}
%	}
%	{\end{outline}
%}

\newcommand{\realn}{\mathbb{R}^n}

\begin{document}

\title{Differential Equations}
\author{Avery Karlin}
\date{Spring 2016}
%\newcommand{\textbook}{}
\newcommand{\teacher}{Stern}

\maketitle
\newpage
\hypertarget{content}{\tableofcontents}
\vspace{11pt}
\noindent
%\underline{Primary Textbook}: \textbook\\
\underline{Teacher}: \teacher
\newpage

\section{Introduction}

\subsection{Definitions}

\textbf{Differential equations} are relations between one or more unknown functions an a finate amount of their derivatives, along with a certain number of independent variables. Most are not solvable, and do not have approximations for more than a minute region of the function.

\textbf{Linear differential equations} are defined as equations with algebraic functions as the multiplier of each function derviative and each derivative with a degree/power of 1, such that all linear equations are able to be solved.

\textbf{First order differential equations} means that the highest derivative in the equation is the first derivative, and is extended as such.

\textbf{Coupled systems} are those defined as requiring being solved together, such as $$X' = aX + bY, Y' = cX + bY,$$ rather than equations containing only their own function, which would be uncoupled.

\textbf{Ordinary Differential Equations} are those where all unknown functions depend on the same, single independent variable, while \textbf{Partial Differential Equations} are those where the functions have multiple independent variables. In addition though, vector function differential equations can fall into either catagory, action more as a system of differential equations, rather than a single one.

Finally, in general, either an initial condition is set allowing it to be solved for a single solution, or a family of solution must be considered as the answer.

\subsection{Example}

For some identically restricted, coupled functions, $$R' = aR + bJ, J' = bR + aJ,$$ where R(0) = $R_0$ and J(0) = $J_0$, where $a < 0, b > 0$, if $|a| > |b|$, we can graph the phase plane of R on the x-axis, J on the y-axis, such that for all possible functions, it moves toward the stable node (0, 0). These stable nodes don't need to be a point, but rather can be a curve of some sort. If $|a| < |b|$, if the initial points are above R = -J, it moves in a parabolic fashion until asymptotic to R = J on the positive side. If below R = -J, it moves similarly, except asymptotic to R = J on the negative side. Finally, if $|a| = |b|$, the function moves infinitely in a circle around the origin, with the radius determined by the initial point.

\section{Basic Existance and Uniqueness Theorems}

\subsection{Flow Theorem}
\subsubsection{Theorem} 
Let $\vec{F}(\vec{x}) = (F_1(\vec{x}), F_2(\vec{x}), \dots, F_n(\vec{x}))$ be a vector field (the assignment of a vector to each point within some subset of space) defined on some closed, bounded region D in $\realn$. Also assume $\vec{F}$ is $C^1$ and let $\vec{p}$ be a specific point interior to D. Then $\exists$ a function $\vec{\sigma}(t)$ from some time interval t, $(-\epsilon, \epsilon)$ with $\epsilon > 0$ into D, such that $\vec{\sigma}(0) = \vec{p}$ and $\vec{\sigma}'(t) = \vec{F}(\vec{\sigma}(t)).$

Thus, for any point on the interval t, the velocity of $\vec{\sigma}$ is the vector field, such that we call $\vec{\sigma}$ the flow for $\vec{F}$ on that interval, starting at $\vec{p}$, which we call the initial condition.

The flow is unique, such that for any two flows for $\vec{F}$ starting at the same initial condition, they must agree whenever both are defined.

This can easily be extended to higher derivatives, such that it is written as a system of two differential equations, each going one derivative, though an initial condition is needed for both of the systems.

\subsubsection{Application}
For some nth-order differential equation, $f(t)x^{(n)} = F(t, x, x', x'', \dots, x^{(n-1)})$, written in \textbf{singular standard form}, assuming there is no singularity (value at which $f(t_0)$ = 0), and f(t) is continuous, such that it is of constant sign by the Intermediate Value Theorem on the interval, we divide by f(t) to put the equation explicitly equal to the nth derivative term, called the \textbf{regular standard form case}. It must also be an initial value problem, such that initial values for each derivative and the function, x(t), itself, are given for some initial time, $t_0$. It should be noted that t and x can be vectors/vector-valued functions as well, and this would still be valid.

\underline{Theorem:} There is a unique solution, $\sigma = x(t)$ defined in some time interval $(t_0 - \epsilon, t_0 + \epsilon)$, where $\epsilon > 0$.

\underline{Proof:}
Let $x_0 = t, x_1 = x_1(t) = x(t), x_2 = x(t), \dots, x_n = x^{(n-1)}(t)$, such that $x_0' = 1, x_1' = x_2, x_2' = x_3, \dots, x_n' = x^{(n)}$. It should also be noted that $x_0' = x_1 = F_0(x_0, x_1, \dots, x_n), x_1' = F_1(x_0, x_1, \dots, x_n)$, and so forth, removing the time dependence, such that each is purely in terms of $\vec{x} = (x_0, x_1, \dots, x_n)$. 

Thus, the flow theorem is able to apply, assuming that the $F_d(\vec{x})$ functions are all $C^1$. Since each of the functions are either constant or continuous, with the exception of $x_n'$, we must show that F is $C^1$ for there to be a unique solution.

\subsubsection{Proof}


\end{document}
