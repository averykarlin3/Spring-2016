\documentclass[11 pt, twoside]{article}
\usepackage{textcomp}
\usepackage[margin=1in]{geometry}
\usepackage[utf8]{inputenc}
\usepackage{color}
%\usepackage{indentfirst} %Comment out for no first paragraph indent
\usepackage[parfill]{parskip}
\usepackage{setspace}
\usepackage{tikz}
\usepackage{amsmath}
\usepackage{amsfonts}
\usepackage{amssymb}
\usepackage{outlines}
\usepackage{enumitem}

\usepackage{hyperref}
\hypersetup {
	colorlinks,
	citecolor=black,
	filecolor=black,
	linkcolor=black,
	urlcolor=black
}

\newcommand{\sepitem}{0pt} %Added room between items on the list, not including a list and its sublist
\newcommand{\seppar}{1pt} %Between items and lists overall

\setenumerate[1]{itemsep=\sepitem, parsep=\seppar}
\setenumerate[2]{itemsep=\sepitem, parsep=\seppar}
\setenumerate[3]{itemsep=\sepitem, parsep=\seppar}
\setenumerate[4]{itemsep=\sepitem, parsep=\seppar}

\newenvironment{outline*}
{
	\begin{outline}[enumerate]
	}
	{\end{outline}
}

\begin{document}

\title{Microeconomics}
\author{Avery Karlin}
\date{Spring 2016}
\newcommand{\textbook}{Krugman's Microeconomics for AP}
\newcommand{\teacher}{Schweitzer}

\maketitle
\newpage
\tableofcontents
\vspace{11pt}
\noindent
\underline{Primary Textbook}: \textbook\\
\underline{Teacher}: \teacher
\newpage

\section{Chapter 2 - Supply and Demand}

\subsection{Price Controls}
\begin{outline*}
\1 Governments can intervene in the market for the benefit of sellers or buyers, based on moral or political arguments, placing price controls in the form of a ceiling or floor
\2 In inefficient markets, price controls can often not hurt the market efficiency, but can rather improve it
\1 Price ceilings are generally used during major shortages, such as wars or natural disasters, that hurt the general public
\end{outline*}

\end{document}
