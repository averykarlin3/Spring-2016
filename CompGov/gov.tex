\documentclass[11 pt, twoside]{article}
\usepackage{textcomp}
\usepackage[margin=1in]{geometry}
\usepackage[utf8]{inputenc}
\usepackage{color}
\usepackage{indentfirst} %Comment out for no first paragraph indent
\usepackage[parfill]{parskip}
\usepackage{setspace}
\usepackage{tikz}
\usepackage{amsmath}
\usepackage{amsfonts}
\usepackage{amssymb}
\usepackage{enumitem}
\usepackage{outlines}

\usepackage{hyperref}
\hypersetup {
	colorlinks,
	citecolor=black,
	filecolor=black,
	linkcolor=black,
	urlcolor=black
}

\newcommand{\sepitem}{0pt} %Added room between items on the list, not including a list and its sublist
\newcommand{\seppar}{1pt} %Between items and lists overall

\setenumerate[1]{itemsep=\sepitem, parsep=\seppar}
\setenumerate[2]{itemsep=\sepitem, parsep=\seppar}
\setenumerate[3]{itemsep=\sepitem, parsep=\seppar}
\setenumerate[4]{itemsep=\sepitem, parsep=\seppar}

\newenvironment{outline*}
{
	\begin{outline}[enumerate]
	}
	{\end{outline}
}

\begin{document}

\title{AP Comparative Government}
\author{Avery Karlin}
\date{Spring 2016}
\newcommand{\textbook}{Ethel Wood's AP Comparative Government}
\newcommand{\teacher}{Trainor}

\maketitle
\newpage
\tableofcontents
\vspace{11pt}
\noindent
\underline{Primary Textbook}: \textbook\\
\underline{Teacher}: \teacher
\newpage

\section{Chapter 1 - Introduction to Comparative Government}

\subsection{Comparative Method}
\begin{outline*}
\1 Government is the leadership and institutions which make national policy decisions, while politics is the flow of power from different people and groups within a government
\1 Political science, as a social science, can either be done based on empirical data, or facts and statistics, or nrmative issues, which are based on value judgements
\2 Since it is a science, it requires a hypothesis of a relationship with variables, to find a causation from the independent to dependent variable, such that one causes the other
\2 Correlations are when the change in both variables are simultaneous, implying the possibility of a causation, but not acting as evidence
\1 The main comparative model is the three-world approach from the Cold War, dividing into the first world of the US and its allies, the second of the USSR and its allies, and the third world of economically underdeveloped, unaffiliated nations
\2 It is used in modern day, based on communist, post-communist, and capitalist nations, advanced, economically developed democracies, and developing nations
\2 Developing and communist nations are more likely to become authoritarian nations, rather than economically developed capitalist nations, which are likely to be democratic
\1 This model integrates political and economic systems, due to the economy having a strong factor in citizen interaction with the government, allows observation of the impact of political change since the fall of the USSR, and the impact of informal politics, or the interaction of citizens and the civil society, or the organization and defining of citizen activism, with politics
\end{outline*}

\subsection{Sovereignty, Authority, and Power}

\subsubsection{Nation-States}
\begin{outline*}
\1 States are organizations that define the use of violence within a specific territory, through military and weapon restrictions, with institutions to create policy and promote general welfare
\2 States thus have soverignty, or the ability to create their own policies without influence
\2 States without soverignty are subject to corruption, used by internal and external organizations for their own ends, often in undeveloped nations
\1 Nations are groups of people with a common political identity, such that nationalism is the send of belonging, often resulting in patriotism for the nation
\1 Nation-states are the main form of a state, such that borders are drawn around a specific nation, providing the identity for those in the main nationality
\2 Bi/multi-national states are those containing multiple nations, such as the USSR, such that minority groups began protests for independence, decaying into nationstates, though the same issue has applied to the multinational Russia since
\2 Stateless nations are those without a state, such as the Kurds, often causing fierce nationalism
\end{outline*}

\end{document}
