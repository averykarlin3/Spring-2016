\documentclass[11 pt, twoside]{article}
\usepackage{textcomp}
\usepackage[margin=1in]{geometry}
\usepackage[utf8]{inputenc}
\usepackage{color}
\usepackage{indentfirst} %Comment out for no first paragraph indent
\usepackage[parfill]{parskip}
\usepackage{setspace}
\usepackage{tikz}
\usepackage{amsmath}
\usepackage{amsfonts}
\usepackage{amssymb}
\usepackage{enumitem}
\usepackage{outlines}

\usepackage{hyperref}
\hypersetup {
	colorlinks,
	citecolor=black,
	filecolor=black,
	linkcolor=black,
	urlcolor=black
}
\newcommand{\sepitem}{0pt} %Added room between items on the list, not including a list and its sublist
\newcommand{\seppar}{0pt} %Between items and lists overall

\setenumerate[1]{itemsep=\sepitem, parsep=\seppar}
\setenumerate[2]{itemsep=\sepitem, parsep=\seppar}
\setenumerate[3]{itemsep=\sepitem, parsep=\seppar}
\setenumerate[4]{itemsep=\sepitem, parsep=\seppar}

\newenvironment{outline*}
{
	\begin{outline}[enumerate]
	}
	{\end{outline}
}

\begin{document}

\title{AP Comparative Government}
\author{Avery Karlin}
\date{Spring 2016}
\newcommand{\textbook}{Ethel Wood's AP Comparative Government}
\newcommand{\teacher}{Trainor}

\maketitle
\newpage
\tableofcontents
\vspace{11pt}
\noindent
\underline{Primary Textbook}: \textbook\\
\underline{Teacher}: \teacher
\newpage

\section{Chapter 1 - Introduction to Comparative Government}

\subsection{Comparative Method}
\begin{outline*}
\1 Government is the leadership and institutions which make national policy decisions, while comparative government is the study of the flow of power from different people and groups within a government
\2 Politics are the activities associated with the governance of a country or area, especially the debate or conflict among individuals or parties with or hoping to gain power
\1 Political science, as a social science, can either be done based on empirical data, or facts and statistics, or normative issues, which are based on value judgements
\2 Since it is a science, it requires a hypothesis of a relationship with variables, to find a causation from the independent to dependent variable, such that one causes the other
\2 Correlations are when the change in both variables are simultaneous, implying the possibility of a causation, but not acting as evidence
\1 The main comparative model is the three-world approach from the Cold War, dividing into the first world of the US and its allies, the second of the USSR and its allies, and the third world of economically underdeveloped, unaffiliated nations
\2 It is used in modern day, based on communist, post-communist, and capitalist nations, advanced, economically developed democracies, and developing nations
\2 Developing and communist nations are more likely to become authoritarian nations, rather than economically developed capitalist nations, which are likely to be democratic
\1 This model integrates political and economic systems, due to the economy having a strong factor in citizen interaction with the government, allows observation of the impact of political change since the fall of the USSR, and the impact of informal politics, or the interaction of citizens and the civil society, or the organization and defining of citizen activism, with politics
\end{outline*}

\subsection{Sovereignty, Authority, and Power}

\subsubsection{Nation-States}
\begin{outline*}
\1 States are organizations that define the use of violence within a specific territory, through military and weapon restrictions, with institutions to create policy and promote general welfare
\2 States thus have soverignty, or the ability to create their own policies without influence
\2 States without soverignty are subject to corruption, used by internal and external organizations for their own ends, often in undeveloped nations
\1 Nations are groups of people with a common political identity, such that nationalism is the send of belonging, often resulting in patriotism for the nation
\1 Nation-states are the main form of a state, such that borders are drawn around a specific nation, providing the identity for those in the main nationality
\2 Bi/multi-national states are those containing multiple nations, such as the USSR, such that minority groups began protests for independence, decaying into nationstates, though the same issue has applied to the multinational Russia since
\2 Stateless nations are those without a state, such as the Kurds, often causing fierce nationalism
\2 Nations generally expanded from core areas, until they reached another nation-state, creating boundaries, with periphery areas around the core areas, with more open land and fewer towns
\2 Multicore states often have inner-conflict as the result of multiple groups having competing interests, and can hurt stability, such as in Nigeria, though often not, like the US
\end{outline*}

\subsubsection{Governmental Regimes}
\begin{outline*}
\1 Regimes are the sets of rules that states set and follow, generally divided into authoritarian and democratic regimes
\1 Democratic regimes can either be indirect, electing representatives for the people, or direct, with individuals directly having a say in government, generally only direct with small populations
\2 Parliamentary systems are those where the legislature is elected, and those officials determine the executive, while presidential systems have both elected, with seperation of powers between
\2 While there are different levels of economic regulation, democracies have independent corporations from the government
\2 Most democracies are divided into a legislative, executive, and judicial branches
\2 Semi-presidential systems can also exist, such as in the 1993 Russian constitution, with both a parliamentary prime minister and a president sharing power, though the president has taken far more under Putin
\1 Authoritarian regimes have power held by the political elites without citizen input, either by a dictator, hereditary monarch, aristocrats, or single political party, controlling both the government and economy
\2 In these societies, there is no limits on the power of the leaders, responsibility to the public, or restriction of civil rights
\2 Communist countries are controlled by the party, controlling all aspects of life, following Marx or Zedong economic philosophy
\2 Corporatism is the supervision of government policies by some labor or business group, though it may be some other patron-client system
\3  
\2 Totalitarian regimes are a subset of authoritarian, whcih attempt to control all aspects of political and economic systems, often based on a strong ideological goal, such as communism
\3 Totalitarian governments especially use violence to remove opposition, and are more illegitimate, in that they are not accepted by the people, which authoritarian governments may be
\2 Military rule is a common form of authoritarian, often taking power in a forced takeover/coup d'etat during unrest, generally with public support, restricting civil liberties to preserve order, joining with the bureaucracy
\3 This can lead to a weak state, forcing other coup d'etats, in a series of weak regimes
\end{outline*}

\section{Articles}
\begin{enumerate}
\item ``The Clash of Civilizations?'' by Samuel Huntington
\end{enumerate}

\end{document}
