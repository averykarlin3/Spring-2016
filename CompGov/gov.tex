\documentclass[11 pt, twoside]{article}
\usepackage{textcomp}
\usepackage[margin=1in]{geometry}
\usepackage[utf8]{inputenc}
\usepackage{color}
\usepackage{indentfirst} %Comment out for no first paragraph indent
\usepackage[parfill]{parskip}
\usepackage{setspace}
\usepackage{tikz}
\usepackage{amsmath}
\usepackage{amsfonts}
\usepackage{amssymb}
\usepackage{enumitem}
\usepackage{outlines}

\usepackage{fancyhdr}
\pagestyle{fancy}
\cfoot{\hyperlink{content}{\thepage}}
\lhead{}
\chead{}
\rfoot{}
\lfoot{}
\rhead{}
\renewcommand{\headrulewidth}{0pt}
\renewcommand{\footrulewidth}{0pt}

\newcommand{\foot}[1]{\hyperlink{#1}{$_#1$}}

\usepackage{hyperref}
\hypersetup {
	colorlinks,
	citecolor=black,
	filecolor=black,
	linkcolor=black,
	urlcolor=black
}
\newcommand{\sepitem}{0pt} %Added room between items on the list, not including a list and its sublist
\newcommand{\seppar}{0pt} %Between items and lists overall

\setenumerate[1]{itemsep=\sepitem, parsep=\seppar}
\setenumerate[2]{itemsep=\sepitem, parsep=\seppar}
\setenumerate[3]{itemsep=\sepitem, parsep=\seppar}
\setenumerate[4]{itemsep=\sepitem, parsep=\seppar}

\newenvironment{outline*}
{
	\begin{outline}[enumerate]
	}
	{\end{outline}
}

\begin{document}

\title{AP Comparative Government}
\author{Avery Karlin}
\date{Spring 2016}
\newcommand{\textbook}{Ethel Wood's AP Comparative Government}
\newcommand{\teacher}{Trainor}

\maketitle
\newpage
\hypertarget{content}{\tableofcontents}
\vspace{11pt}
\noindent
\underline{Primary Textbook}: \textbook\\
\underline{Teacher}: \teacher
\newpage

\section{Chapter 1 - Introduction to Comparative Government}

\subsection{Comparative Method}
\begin{outline*}
\1 Government is the leadership and institutions which make national policy decisions, while comparative government is the study of the flow of power from different people and groups within a government
\2 Politics are the activities associated with the governance of a country or area, especially the debate or conflict among individuals or parties with or hoping to gain power
\1 Political science, as a social science, can either be done based on empirical data, or facts and statistics, or normative issues, which are based on value judgements
\2 Since it is a science, it requires a hypothesis of a relationship with variables, to find a causation from the independent to dependent variable, such that one causes the other
\2 Correlations are when the change in both variables are simultaneous, implying the possibility of a causation, but not acting as evidence
\1 The main comparative model is the three-world approach from the Cold War, dividing into the first world of the US and its allies, the second of the USSR and its allies, and the third world of economically underdeveloped, unaffiliated nations
\2 It is used in modern day, based on communist, post-communist, and capitalist nations, advanced, economically developed democracies, and developing nations
\2 Developing and communist nations are more likely to become authoritarian nations, rather than economically developed capitalist nations, which are likely to be democratic
\2 This model integrates political and economic systems, due to the economy having a strong factor in citizen interaction with the government, allows observation of the impact of political change since the fall of the USSR, and the impact of informal politics, or the interaction of citizens and the civil society, or the organization and defining of citizen activism, with politics
\1 The Huntington model\foot{1} states the next great conflict will be between different civilizations (the broadest level of identity/cultural similarites), having moved from monarch territories to nations (from the 1790s to 1910s) to ideologies
\2 After the Cold War between Western civilizations, non-Western groups began to move away from being puppets of the West, responding to Western ideas of appropriate policies in the UN and IMF, to going back to traditional ideas, and due to increased interaction between civilizations
\3 Educated foreigners are now being educated in their own culture, while Western culture spreads through globalization, reversing past trends
\3 Finally, trade has begun moving backward toward mainly regional in recent years
\3 Most particularly the Western idea of a universal civilization contrasts the Eastern idea of particularism and differences between civilizations, forcing civilization to bandwagon with the west, isolate themselves, or modernize to create a balance of power (though only Japan has done it without moving towards Western)
\2 The civilizations include Western, Confucian, Japanese, Slavic-Orthodox, Hindu, Islamic, Latin American, and African civilizations, due different views on relationships, rights, hierarchy, and religion, such that intellectual debate is not possible
\2 Economic progress and social change has also weakened the power of nations, resulting in religious divisions taking over, leading to fundementalism
\2 This is also the movement back to the original, unchanging institutions, rather than ideological divisions, spreading both influence between civilization and local territory along fault lines
\2 This has manifested most commonly in economic rivalries such as US-Japan or US-China and civilization support for minor conflicts, but also in ethnic clensing, Islamic fundementalism, movement from democratic institutions, and military conflict
\3 In addition, the double standard of exempting similar nations from human rights regulations, while condemning others, leads to conflict
\3 This leads to Western attempts to ban the production of non-Western weaponry, in an attempt to not hurt their interests, while other nations define it as equal protection, leading to middle eastern and eastern Weapons States
\2 Torn countries between multiple civilizations must have the poltical and economic elite, the general public, and the majority of the new civilization agree to be able to take a new identity, found within Mexico
\3 On the other hand, in the case of Russia, none are present with the Post-Cold War attempts to join the West
\end{outline*}

\subsection{Sovereignty, Authority, and Power}
\subsubsection{Nation-States}
\begin{outline*}
\1 States are organizations that define the use of violence within a specific territory, through military and weapon restrictions, with institutions to create policy and promote general welfare
\2 States thus have soverignty, or the ability to create their own policies without influence
\2 States without soverignty are subject to corruption, used by internal and external organizations for their own ends, often in undeveloped nations
\1 Nations are groups of people with a common political identity, such that nationalism is the send of belonging, often resulting in patriotism for the nation
\1 Nation-states are the main form of a state, such that borders are drawn around a specific nation, providing the identity for those in the main nationality
\2 Bi/multi-national states are those containing multiple nations, such as the USSR, such that minority groups began protests for independence, decaying into nationstates, though the same issue has applied to the multinational Russia since
\2 Stateless nations are those without a state, such as the Kurds, often causing fierce nationalism
\2 Nations generally expanded from core areas, until they reached another nation-state, creating boundaries, with periphery areas around the core areas, with more open land and fewer towns
\2 Multicore states often have inner-conflict as the result of multiple groups having competing interests, and can hurt stability, such as in Nigeria, though often not, like the US
\end{outline*}

\subsubsection{Governmental Regimes}
\begin{outline*}
\1 Regimes are the sets of rules that states set and follow, generally divided into authoritarian and democratic regimes
\1 Democratic regimes can either be indirect, electing representatives for the people, or direct, with individuals directly having a say in government, generally only direct with small populations
\2 Parliamentary systems are those where the legislature is elected, and those officials determine the executive, while presidential systems have both elected, with seperation of powers between
\3 Thus, in a parliamentary system, the prime minister doesn't have the same monopoly on power, rather ``first among equals'', such that it is harder to lobby, due to requiring a majority
\3 Cabinet members are also taken from the legislature, with a shadow cabinet of members of the largest opposition party
\3 In addition, since there is often more than two parties, they require the support of a third party to create a coalition government to get the majority needed, offering positions and policies
\2 While there are different levels of economic regulation, democracies have independent corporations from the government
\2 Most democracies are divided into a legislative, executive, and judicial branches
\2 Semi-presidential systems can also exist, such as in the 1993 Russian constitution, with both a parliamentary prime minister and a president sharing power, though the president has taken far more under Putin
\2 Democratic regimes rely on pluralism, or power split among many interest groups attempting to influence
\1 Authoritarian regimes have power held by the political elites without citizen input, either by a dictator, hereditary monarch, aristocrats, or single political party, controlling both the government and economy
\2 In these societies, there is no limits on the power of the leaders, responsibility to the public, or restriction of civil rights
\2 Communist countries are controlled by the party, controlling all aspects of life, following Marx or Zedong economic philosophy
\2 Corporatism is the supervision of government policies by some labor or business group, though it may be some other patron-client system
\3 Corporatism often results from authoritarian regimes trying to provide the appearence of citizen involvement to gain co-optation, or citizen support, while banning other groups 
\3 Patron-clientelism is the system of benefits provided to a specific group in exchange for vocal support
\3 It can also result from economic regulation or nationalization of industries resulting in close ties between government and industry
\3 Democratic corporatism can be shown by recognition of specific groups by the state, while forcing others to require recognition, legally bound to the state, working on behalf of the state
\2 Totalitarian regimes are a subset of authoritarian, whcih attempt to control all aspects of political and economic systems, often based on a strong ideological goal, such as communism
\3 Totalitarian governments especially use violence to remove opposition, and are more illegitimate, in that they are not accepted by the people, which authoritarian governments may be
\2 Military rule is a common form of authoritarian, often taking power in a forced takeover/coup d'etat during unrest, generally with public support, restricting civil liberties to preserve order, joining with the bureaucracy
\3 This can lead to a weak state, forcing other coup d'etats, in a series of weak regimes
\1 The Democratic Index was published by the Economist since 2007, ranking countries based on the electoral process and pluralism, civil liberties, government functioning, political participation, and political culture
\2 It also catagorizes into democracies (like the UK), flawed democracies, authoritarian (like Nigeria, Russia, China, and Iran), and hybrid regimes (like Mexico)
\end{outline*}

\subsubsection{Legitimacy}
\begin{outline*}
\1 Legitimacy, or the right to rule, is determined by the citizens, and is catagorized as either traditional, charismatic, or rational-legal
\2 Legitimacy is easier to maintain in economic prosperity and with high government performance approval
\1 Transitional is based on tradition, such as hereditary rulers, often based on myths, legends, religion, or divine right, with ceremonies, symbols, and artifacts to encourage the idea of legitimacy
\1 Charismatic is often based on personality or military talent, such as Napoleon, though when he lost militarily, it faded, generally a shortlived form of legitimacy, unable to be passed on after death
\1 Rational-legal is based on institutional laws and procedures, preserved through belief in the rule of law and acceptance of the authority of the state, such that shared political culture is important
\2 It can be based on common law, or legal tradition and precedents
\2 Legitimacy of rational-legal in democratic governments can be the result of the loss of the legitimacy of the electoral system
\1 In modern states, the main form of legitimacy is from rational-legal, though traditional and charismatic allow easier gain of power, or influencing politics easier within interest groups
\2 Many states also preserve some form of traditional legitimacy, to add legitimacy to the legal-rational democratic form of government
\end{outline*}

\subsubsection{Political Culture and Ideologies}
\begin{outline*}
\1 Political culture is the collection of political beliefs, values, practices, and institutions which a country is based on, such that for a government to remain, it must be based on that culture
\2 Social capital, or reciprocity and trust between citizens and the state or other citizens of all levels, can be used to measure how democratic it is, such that more democratic makes it greater
\3 On the other hand, social capital theory, which predicts difficulty in Islam or Confucian regions, has been critiqued for ignoring countries such as Turkey or India
\2 Consensual political culture is agreement on what issues should be solved and the process by which decisions are made, such that legitimacy of the government is accepted
\2 Conflictual political culture by fundemental economic, religious, or polticial differneces often leads to conflict, and prevents effective rule
\1 Political ideologies are sets of political values of the basic goal of government, held by the individual
\2 Liberalism values political and economic freedom, maximizing rights and freedoms, and allowing citizens to disagree with the state and attempt to influence decisions
\2 Communism values equality, believing freedom won't create general prosperity, believing eventually a wealthy class will form and take control of the government, advocating state control of all resources to protect economic equality
\2 Socialism values a combination of freedom and equality, believing in the free market and private owenership, but believes the state have heavy control of the economy to provide benefits and preserve equality
\2 Fascism values strength, believing some groups are inherantly inferior, attempting to create the strongest possible state, such that rights must be taken away by the authoritarian state to preserve it
\2 Religious ideologies also play a large role in many nations, often having an official state religion, or having special interest groups influencing it
\end{outline*}
\subsection{Political and Economic Change}
\begin{outline*}
\1 Change generally happens both politically and economically simultaneously, and most countries experience it over time, but when happening seperately, creates tensions
\1 Change can occur through reform, attempting to use standard political and economic institutions to create change
\2 Revolution attempts to change the political and economic institutions through the overthrow or revision of the institutions, generally impacted economic, political, and social systems, regardless of the intent
\2 Coup d'etats replace the government with new leaders by force, often carried out by the military, but can cause instability
\1 The strongest attitude toward change is radicalism, or the belief in rapid, dramatic changes, often believing the institutions cannot be fixed, leading to revolutions
\2 Liberalism as an attitude is the belief in reform and gradual change, beliving in repairing and improving existing systems, with the goal often of leading to a complete transformation over time
\2 Conservatism believes change is disruptive and causes unexpected negative outcomes, believing in the need to preserve legitimacy of government, basic societal values, and law and order
\2 Reactionaries believe that the current state has already move too far from basic societal values, wanting to use revolutionary means to return to old institutions
\1 The first major trend of modern change is democratization, based on the idea of competitive elections, with many countries moving further to liberal/substantive democracies, instead of illiberal/procedural
\2 Liberal democracies have belief in neutrality of the judiciary, checks and balances on power, civil liberties, rule of law, civilian control of the military, and open civil society
\3 Illiberal democracies often have an unchecked executive and restricted citizen groups, preventing truly free elections, but are necissary before a society can become a liberal democracy
\2 Huntington believes there are three waves of democratization, the first gradual until WWII, the second after WWII involving de-colonization, and the third involving the defeat of totalitarians after the Cold War
\2 Democratization is due to the legitimacy of authoritarians, expansion of urban middle class, human rights emphasis, and international snowball effect
\2 Democratization happens after a trigger event taking place, after a revolution of rising expectations of high living standards, causing democratic consolidation of the elites and public willing to share power, spreading throughout society called political liberalization
\1 The second trend is economic liberalism, moving to market economies, such that it is under debate, due to influence such as China, if democracy and market economies inherantly move together
\2 19th century European reformists were generally middle-class bourgeoisie, who wanted their views represented in government, and economic goals unrestricted to allow economic mobility
\2 Radicals, on the other hand, believed that freedom clashed with equality, and thus a free market was not the ideal, including Marx, beliving instead of a command economy of government owned businesses
\3 These economies, in the USSR and China, had a state planning committee, with economic production blueprints and quotas in 5 year plans
\3 On the other hand, these generally, while creating economic growth, did not lead to higher living standards
\2 In recent years, most command economies have moved toward market ecnomies with less government regulation, with the current debate between a mixed (with significant regulation and control from government) or market economy
\2 Economic liberalization is based on the failure of many command economies and the belief that government is too large
\3 Thus, many command economies have had marketization toward market economies and privatization toward private ownership
\3 The main downside, the business cycle, has led most to adopt a mixed economy to lower the dangers of the business cycle
\1 The third trend is fragmentation, or divisions based on culture/ethnicity, moving away from prior globalization toward nationalism, especially found in the politicization of religion and increase in fundementalism
\2 It is argued that those who believe in the clash of civilizations underestimate this factor of cultural differences
\end{outline*}
\subsection{Citizens, Society, and the State}
\begin{outline*}
\1 Social cleavages are divisions in society that are outside politics, but impact political policymaking based on causing deep political identification, including social, ethnic, religious, and regional cleavages
\2 Coinciding cleavages are those which divides the same groups against each other on issues, while cross-cutting cleavages are those which divide groups that agree on some issues, but disagree with others
\2 Regional conflicts are often the results of different levels of economic development, or religious divisions between regions
\2 The depth of the cleavages in the social structure of a society, the level of political party alignment with cleavages, and specific cleavages not involved in the political process determine the importance
\1 Government-citizen relationships are based first on political efficacy, or the citizen ability to understand and influence political events, determining if they feel the government cares about their opinion
\2 This works to create active voting behavior, and political participation, rather than interacting purely through subject activities (obeying laws), creating political attitudes
\2 Attitudes in other respects, such as trust or the ability of the government to impact their lives also play a role
\2 Government transparency also changes interactions, preventing corruption, as well as the methods of learning about political actions to create immediate views
\1 Social movements are organized, collective activities to create desired policies, using nontraditional reform methods and bring non-mainstream positions to mainstream society
\1 Civil society are voluntary organizations outside the government to aid identification and advancement of personal interests, encouraged in liberal democracies
\2 They can be either political or apolitical (not politically active), rather just to promote goals and interests, preventing tyranny of the majority
\2 It has been argued that globalization has led to cosmopolitanism, or a universal political order and civil society based on worldwide identity and values, found within international, political, nongovernmental organizations (NGOs)
\2 Authoritarian nations are against civil society, dividing purely based on social clevages
\3 Civil societies are often formed later through civil education of democratic rights, and through NGO involvement
\end{outline*}
\subsection{Political Institutions}
\subsubsection{National Forces}
\begin{outline*}
\1 Political institutions are structures which carry out governing, though they cannot be assumed to have the same powers in each nation
\1 Unitary systems have all policymaking centralized in one location, while federal systems divides between central government and sub-units, and confederal has power almost purely in sub-units, with a weak central
\2 Federal and confederal systems are criticized for inefficiency, due to local governments with possibly competing interests, such that very few governments are confederal for that reason
\1 Supernational organizations affecting national policies are a result of integration, or the loss of soverignty to gain international influence, based on shared policies and rules, such as the UN
\2 This results in a relationship between domestic policies and international relations, creating additional international trade, banking, assets, and foreign direct investment
\2 It also creates a ripple effect of international events
\2 On the other hand, integration ironically causes fragmentation, by iviing the world in regional international organzations, such as the EU
\1 Centripetal forces are those which bind states together, such as nationalism, or identities based on nationalism, encouraging belief in laws and patriotism, using schools, symbols and holidays to promote it
\2 It is also encouraged by transportation systems and technology, uniting the parts of the country with each other and its government
\1 Centrifugal forces include organizations rivaling the government for  influence, such as the church in the USSR, or nationalism if leading to seperatist movements or devolution
\2 Centifugal forces can often lead to devolution, or decentralized decision-making to regional governments, moving toward federalism, even in long-established states
\3 Devolutionary forces can include ethnonationalism, or the feeling of an ethnic group as a seperate nation, with the right to autonomy, especially if the ethnic group is concentrated in a specific region, due to ethnic groups having a shared culture, language, customs, and religons
\3 Regional economic inequality or peripheral location, especially if cut off geographically, can also be a strong devolutionary force
\2 Seperatist movements to fully break into a seperate nation, are generally ethnic, based in nationalism, and can be encouraged by peripheral location, socioeconomic inequality
\end{outline*}
\subsubsection{Government Branches}
\begin{outline*}
\1 The executive carries out laws, split into the head of state (symbolizing/representing the people at home and abroad, often a figurehead, often called the president) and the head of government (often the prime minister)
\2 The chief executive begins policy initiatives, often given veto power in a presidential system, makes foreign policy crisis decisions, and oversees the execution of laws
\2 The cabinet in a parlimentary system is led by the ``first among equals'' prime minister, taken from the legislature to run debartments, formed from a cabinet coalition in a multi-party system with no majority
\3 Cabinet coalitions can often lead to instability
\3 In presidential systems, the president chooses the cabinet, approved often by the legislature, often more independent than in a parlimentary system due to not being major political figures
\3 On the other hand, in a presidential system, the president can remove them from the cabinet if they disobey his wishes
\2 Bureaucracies are agencies to implement government policy, viewed by Weber as necissary to respond to a changing society, growing as the role of government grew
\3 He stated all bureaucracies must be complete meritocracies with clear goals, extensive, well-established rules, task specialization/division of labor, a hierarchy
\3 On the other hand, he feared lack of meritocracy, found in the US by the patronage system until reformed after the Garfield assassination, and discretionary power to make small decisions, against democratic beliefs, but acknoledged they provided stability due to being unelected
\3 In authoritarian regimes, the head of government has complete power,and uses bureaucracy to directly control many aspects of life, with large amounts of patronage
\3 In many Latin American countries, the military regime formed a technocrat of civilian and military bureaucracy coalition, controlling government in the name of rapid modernization
\3 Realistically, bureaucracies have characteristics of non-elected positions, efficient/partially meritocratic structures, job qualifications, hierarchy, and inefficient red tape (especially in large bureaucracies)
\1 Legislatures are governing bodies, popularly elected, although in authoritarian regimes, are controlled by the executive
\2 Legislatures are either bicameral (two houses) or unicameral, often the former to allow an upper house for regional governments and a lower house, directly for the population, balancing regional powers
\3 It can also be used in non-federal systems to allow a house further from the people, and thus less impulsive, to moderate decisions
\2 Legislatures often create, debate, and vote on policies, have taxing and spending power, appoint officials to other branches, serve as appeals courts, impeachment courts, and act as elite recruitment for future government leaders
\1 Judiciaries serve in authoritarian regimes under the control of the executive, simply as legal courts or rubber-stamps
\2 Constitutional courts, defending the democratic principles of a country, often have judicial review to rule on constitutionality of government actions, and have the power to protect against other citizens infringing on rights
\2 Courts have been critisized for being unelected though, and are often weaker than the other branches 
\end{outline*}
\subsubsection{Linkage Institutions}
\begin{outline*}
\1 Linkage institutions are groups which connect government to citizens, such as parties, media, and interest groups, larger in nations with a larger government and population
\1 Political parties provide labels for voting, hold politicians more accountable, and bring different people and ideas to a united group
\2 Two-party systems provide a plurality electoral system, while multi-party provide proportional generally, the latter being more standard
\1 Electoral systems determine how votes are cast and counted
\2 In the US, UK, and India, the first-part-the-post system is used/plurality/winner-take-all system is used, competing for a single seat
\2 Many nations use a proportional system, creating multi-member districts, voting for a party instead of a candidate, and others use a mixed system combining the two
\3 Proportional systems also encourage coalitions to form a majority to get legislation passed
\2 Elections fall into either an election of public officials, referendums to vote on a policy issue (called by the government), or an initiative to vote on policy (called by a petition of citizens)
\3 Plebiscites are non-binding referendums to gage public opinion
\1 Interest groups are organizations attempt to influence public policy, existing independent from the government based on a common interest
\2 Nonpolitical groups can also be interest groups, seeking to advance a private or corporate interest
\2 Interest groups, unlike political parties, do not run candidates to influence the process, but may support candidates
\2 In a multi-party system, they are more similar, in that single issue parties often form, rather than broad platform parties
\2 In authoritarian states, interest groups must be government-sanctioned, acting as transmission belts to extend party influence to members of the group and the general public
\2 Interest group pluralism has complete independence, getting their own funds and leaders
\2 Corporatism has fewer groups compete, with each group generally having a monopoly over the sector, sanctioned and often protected by the government, in between the other two systems
\3 State corporatism has the states decide which groups take control of each sector, while societal corporatism has the interest group form the monopoly, and gain its own power within the state
\1 Political elites, or leaders with high amounts of political influence, are found in every system, such that there must be methods of recruitment to find future elites
\2 Further, all nations must have a system of political succession, or replacing ineffective or resigning leaders
\end{outline*}
\subsection{Public Policy}
\begin{outline*}
\1 Policy is created by the three branches of government, interest groups, and political parties, to solve general issues
\1 Economic performance is one of the most important issues, affected by both international and domestic trade
\2 This is measured by GDP, Gross National Product (GDP + Income Earned by Citizens Outside the Country), GNP per capita, or Purchasing Power Parity
\1 Environmental issues are also a major modern problem, especially in Europe, leading to green, environmental parties and international interest groups
\1 Social welfare, such as providing health, employment, education, and family services, are important factors
\2 These are measured by literacy rates, income distribution, education levels, life expectancy, Gini Index for economic inequality, and the Human Development Index (measuring the well-being of citizens by a variety of social welfare factors and GDP)
\1 Civil liberties, or the promotion of freedom, and political rights, or the promotion of equality, involve government protection
\2 These range based on the amount of rights preserved in addition to levels of government involvement, and are often guaranteed by constitutions in liberal democracies
\2 Freedom House ranks nations from 1 (most free) to 7 as a measure
\end{outline*}
\section{Chapter 2 - United Kingdom}
\subsection{Advanced Democracies}
\begin{outline*}
\1 Advanced democracies are those with a long history of stable democracy, measured by the political characteristics of liberal democracy and economically
\2 They have legitimacy from the history and large social capital, or trust between citizens and with the state
\2 In most nations, citizenship is required to vote, though in Scandinavia, permanent residents can vote, and in most states except the US and France, the state automatically registers voters
\1 Economically, they are post-modernist, with values of environmentalism, education, and health care, as well as post-industrial, due to the large amount of tertiary/service sector, such as tech, legal, business, or financial services
\2 Modernist civilizations have values of secularism, rationalism, materialism, technology, bureaucracy, and freedom over equality, with a large amount in the industrial/secondary sector, due to being industrial
\2 Pre-industrial have a majority of workers in the primary/agricultural sector
\1 In advanced democracies, there has been a move toward supranational integration organizations, especially in Europe by the EU, but also by the North American Free Trade Agreement
\2 This worked to promote international cooperation over nationalism, with the EU adopting a common currency
\1 The UK began democratic ideals of limited monarchy, started industrialization, democratic institutions, and spread them through the world by colonialism
\2 On the other hand, it's economy has slowed since WWII, and lost its colonies, but maintained a powerful worldwide influence, currently transforming into a more European nation
\end{outline*}
\subsection{Soverignty, Authority, and Power}
\begin{outline*}
\1 Legitimacy grew over time based in tradition, originally based on the idea of a traditional, hereditary monarch, but was limited over time, until Parliment was in charge in the 1600s
\2 Except in Northern Ireland religious conflicts, seperation of church and state are engrained
\2 The Constitution of the Crown of documents, common law, and customs take the place of a standard constitution
\3 The Magna Carta was signed in 1215 by King John, limiting the monarch to consulting nobles before major policies
\4 Although it was mainly ignored by the King, and failed to prevent war between the king and the nobles, and was condemned by both the Pope and the King immediately, it created a tradition of democracy\foot{2}
\4 It gave rights to fair, speedy trials, no taxation without representation, and stated the king wasn't above the law\foot{2} (rule of law over arbitrary rule\foot{3})
\4 It also protected the church and confirmed local privileges, still protected in the UK today, although most of the remaining sections are archaic\foot{2} and ignored local peasent representation, and providing rights mainly only to free men, but all groups rallied around it by the 1300s, supported later by the Church due to their protection\foot{3}
\4 In 1216, France invaded to support the Lords, such that it was brought back by loyal lords to protect his son, Henry, as king\foot{2}
\4 After, it was revised several times and not enforced properly, but worked to symbolically preserve a limited monarchy\foot{2}
\4 The ideas within it were not new, found throughout Europe, but were the most detailed restrictions forced on a ruler, due to absurd taxes, complete lack of due process and rule of law, and arbitrary executions\foot{3}
\3 The Bill of Rights was signed by William and Mary in 1688 after the monarchy was restored following the Glorious Revolution after the English Civil War executed Charles I, giving major policy powers, including tax and spending, to Parliament
\3 Common law rule is based on public officials and courts creating precedents instead of a legal code, used limitedly in the US
\1 The development of political culture began with the shaping of the monarchy, followed by the rise of the Parliament (the execution of Charles I, and the installation of the Roundhead/Pro-Parliament leader, Cromwell
\2 Charles II was installed shortly after, until the Glorious Revolution forced the signing of the Bill of Rights to protect Parliament, creating the Prime Minister as head of government by PM Walpole in the 1700s
\2 In the 1700s, the Industrial Revolution replaced the feudal system with merchantalism, creating a middle class of merchants, forcing their inclusion in politics
\2 In the 1700s and later, industrialization and nationalism led to colonialism gaining raw materials abroad to produce goods at home, and allowing communication and transport between distant locations
\2 In the 20th century, world wars and a welfare state hurt the economy, leading to a urban and infrastructure decay, and a neo-conservative backlash of Thatcherism, followed by Blair's Liberal Third Way, attempting to find a balance between welfare and capitalism and acclimating to less world influence
\1 UK Political culture is based on a large amount of nationalism and insularity, or feelings of seperation from Europe, leading to fear of the EU, refusing the Euro even after joining
\2 It also includes noblesse oblige, or duty of the nobles to help the lower classes, dating to feudalism, which led to the welfare state and colonialism to help the lower races overseas
\2 Multi-nationalism is also found, due to the multiple nations united, even with the cultural homogeneity within each, except for religious differences in Northern Ireland
\1 The UK has fewer political-related crimes and a smaller police force, due to respect for the law, except in Northern Ireland for independence and recent international terrorist issues
\end{outline*}
\subsection{Political and Economic Change}
\begin{outline*}
\1 Political change was based in gradualism, or slowly over time, which served to create strong traditions, such as the transition from 1066, requiring noble support by William the Conqueror to become king, until the Bill of Rights
\1 During the Industrial Revolution, the middle class and laborer class were created, and over time, noblesse oblige led to incorporation into the political system
\2 The Great Reform Act of 1832 gave more power to the House of Common and 300k men suffrage, Reform Act of 1867 gave 3M, Representation of the People Act of 1884 further, and Representation of the People Act of 1918 gave all men, and women over 30 suffrage, extended in 1928 to over 21
\2 In the late 1800s, unions and social services, such as mandatory elementary school and pensions were made, further preventing class anger and the rise of Marxism
\2 By the 1900s, the House of Lords only had the power to delay laws, with the House of Common in power
\1 In 1906, the Labour Party represented the working class, with the Conservatives taking middle-class merchants, leading to welfare, housing, public education, and healthcare reform, supported by Labour
\2 Labour took power as the main liberal at this time, pushing the moderate Liberal Party as a 3rd party
\2 Labour then began to act in favor of social democratic ideology and militant trade unionism, supported since by the Trade Union Council coalition, but never reached Marxist revolutionary
\1 WWII led to widespread infrasructure destruction and war debt, and while the Marshall Plan allowed economic recovery, the UK began to prepare colonies for independence, keeping ties, but ending imperialism
\1 The war also led to the collective consensus of the parties joining together in favor of a welfare system, accepting the Beverage Report to provide unemployment, health, pensions, and income to every citizen
\2 In 1948, the National Health Service was made by the Labours, and while the classes were split by party, all supported reforms
\2 During ecnomic crises of the 1970s, Labour moved to the far-left, supporting socialism, and Conservatives went to the far-right, supporting a free market and denationalization of all industry
\3 Industry downturn, loss of colonies, and OPEC oil embargo led to the economic downturn, leading to higher wage demands, and strikes, such as the Coal Strke of 1972
\3 Many voters left the Labour Party, blaming the welfare state and union power, and elected Thatcher, the Iron Lady, strengthening defense, privatizing, moving toward neoliberal theory of low government regulation
\3 On the other hand, many felt she hurt and divided the nation, and disliked her firm personality, challenged in 1990, but still lowered the power of the welfare state
\2 Major, Thatcher's chosen successor, followed her policies, but removed the poll tax, joined the EU, and slowed privatization, but the Cosnervatives still lost the margin
\3 Blair won in 1997, promising a move to the Third Way, more centrist, but after Iraq War support, lost popularity, and Brown took his place, but the recession hurt the Labour Party
\3 In 2010, Cameron formed a coalition with Liberal Democrats and took power, creating the Big Society of grassroots and private organizations over big government
\end{outline*}
\subsection{Citizens, Society, and the State}
\begin{outline*}
\1 The UK is mainly homogeneous, with only 7\% as a minority, but has major social cleavages between Protestant/Catholic, social classes, and multi-national identities
\2 The main region of Great Britian is England, historically holding the majority of the population and political power
\2 Wales is the Western region, became under the English king in the 1500s, with Welsh pride in the Plaid Cymru flag and the language, still taught in schools, though there is some resentment
\2 Scotland is the Northern region, joining by intermarriage in the 1600s, still resisting rule, not agreeing to a single parliament until the 1700s, but have a lot of nationalism and their own flag
\3 They also recently remade their independent Parliament, with the idea of independence under much debate with recent referendums
\2 Ireland was fought over, from Cromwell's attempt in the 1600s to push Protestantism on Ireland, eventually giving home rule in WWI to the all except the NE (Northern Ireland)
\3 This was mainly due to the Irish Republican Army using guerilla warfare, and the religious conflict is still present today
\1 There is a strong divide between the lower and middle class, based on the idea of solidarity, staying within your own class, found also among the upper class
\2 The division was furthered by expensive public boarding schools, training for public life of civil/military/political service, training the elite, before going to Oxbridge (Oxford or Cambridge)
\3 A majority of the government officials attended Oxbridge, though since WWII, scholarships have been given more to lower classes
\2 Middle class students go to private non-boarding grammar schools, but until 2008's Education and Skills Act, had compulory education only until 16 years old
\3 Other universities were available to the middle class, but in 2012, the price was raised for these by Parliament, making it less accessible
\1 Minorities mainly come from former colonies, growing by 53\% in the 1990s, but still low, mainly Indian, then Pakistani, Afro-Caribbean, and then Black African
\2 Past immigration restrictions, especially under Thatcher, make it so most minorities are young, half below 25, working in recent years to half immigration, though open EU borders make it only apply to non-Europeans
\2 There has been tension with minorities by police racism and verbal harassment from citizens, as well as events like the May 2001 race riots
\3 This was caused by the murder of a black man by the police, and white flight has been seen in London, creating segregation, though mixed-race population has been growing
\2 Muslims especially are seperate, as distinct immigrant minority, leading to to several terrorist attacks since 9/11
\3 This is also caused by strong support of the government to the Iraq War against Islam, and the fact that the majority are Pakistani, with stronger links to Bin Laden than other groups
\3 It is also due to lack of general minority integration into mainstream culture, even with more religious rights than other nations, and far more common poverty and illiteracy among Muslims
\2 Eastern Europeans also began immigrating due to the EU open borders (or lessened restrictions until 2014), such that Polish are now the majority of foreign nationals, though the financial collapse had some return
\3 Many are migrant workers, doing low-paying rural work, creating fears of increasing unemployment among the lower classes
\3 The recent refugee crisis along with EU policies guaranteeing social services to refugees, has led to a move from the EU, petitioning for stronger immigration, and limits on services to deincentivize immigrations among both the Conservatives and Labour (working class)
\3 The EU ideal of a political single-entity has also created resentment among the British government for the political aspects of the EU, for fear of losing soverignty
\1 Until the 1970s, political/civic culture was based on trust, pragmatism, harmony (acceptance for other points of view and the legal sstructure), deference to competent authority, and political participation
\2 Since, the idea of a collective consensus has been replaced by politics of protest, with less tolerance, more open disagreement and violent protest against governmental policy
\2 This is due to loss of union support during the 1970s, when many strikes hurt the economy, and the idea of Thatcherism of free market economy, individualism, and competition 
\3 In addition, Northern Ireland violence, especially the murder by military of 13 Catholics in Bloody Sunday in January 1972, and IRA/protestant miliants, has hurt these ideals
\3 The Iraq War also led to protests among members of parties, hurting complete party support, even among cabinet members, leading to Blair's resignation in 2007
\2 On the other hand, New Labour, with looser ties to unions and a looser grasp on Northern Ireland Good by the Good Friday agreement, preserved older values
\1 The UK has more than 70\% of voters vote in parliamentary elections, but with less party loyalty than previously, voting along social and regional lines
\2 The Labour party has power in industrial regions, Scotland, Wales, and London, while Conservatives are generally in rural and suburban areas
\2 Working class is generally Labour, while middle is Conservative, though with increased mobility, many middle-class vote Labour because their parents were lower, but many working-class have been anti-immigrant and anti-taxation, voting Conservative
\2 Labour also takes the disadvantaged, or Scots and Welsh, but recent weakening of both parties due to becoming moderate has allowed third parties to gain a foothold
\end{outline*}
\subsection{Political Institutions}
\begin{outline*}
\1 The political system of the UK is parliamentary with the prime minister chosen from the legislature
\1 Political parties originated in the 1700s, beginning as caucuses, or meetings of people with similar ideologies, where the Tories and Whigs originated under Charles II, the former supporting the king, the latter not
\2 The Whigs were used as slang for Scottish bandits, Tories for Irish bandits, eventually becoming the Liberal and Conservatives respectively
\2 The Labour party would later form during the early 20th century, based on the new industrial worker demands during the Industrial Revolution
\3 It began in 1906 as an allience of trade unions and socialist groups, though Blair weakened union ties during power from 1997 to the 2010 loss to the Conservative coalition
\3 Clause 4 called for nationalizing the commanding heights of industry, such that the third way was shown by the removal of the clause
\3 The loss under Kinnock in 1992, in power since the 1980s, led to his reignation and appointment of the moderate Smith, a Scotsman to get support of Scottish nationalist, dying in 1994
\3 Blair was an educated upper-class, to bring intellectuals and middle-class to the party, in power until 2007, winning during that time, but with a smaller margin in 2005, leading to his resignation in 2007
\3 Brown then took power, but lost further ground, until he resigned in 2010, and Milliband took over, moving far to the left, but losing support to the Conservative-Liberal coalition
\2 Conservatives were the majority mostly from WWII to 1997, based on pragmatism rather than ideology, characterized as London-based and elitist with MPs choosing leadership, only recently adding elections
\3 The traditional (one-nation) wing is based on noblesse oblige, with the elite ruling for the good of all, supporting the EU
\3 The Thatcherite wing is based on free market and lack of welfare, generally Euroskeptics against the EU
\3 Cameron, leader since 2005, has taken power back as Blair became weaker, taking back the majority recently, losing ground temporarily after the takeover of Brown, but gaining it back, acting as a one-nation Tory
\2 Third parties also have strong power, such as the Liberal Democratic Party in the 1980s, though the single-member plurality system creates little representation in Parliament
\3 On the other hand, in certain times such as 2010, a hung parliament of a coalition government rather than a majority has occured
\end{outline*}
\section{Articles}
\begin{enumerate}
\item \hypertarget{1}{``The Clash of Civilizations?'' by Samuel Huntington}
\item \hypertarget{2}{``BBC News, Just how important is the Magna Carta 800 years on?'' by Nick Higham}
\item \hypertarget{3}{``The Guardian, Magna Carta - 800 years on'' by David Carpenter}
\end{enumerate}

\end{document}
