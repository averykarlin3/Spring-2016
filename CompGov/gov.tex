\documentclass[11 pt, twoside]{article}
\usepackage{textcomp}
\usepackage[margin=1in]{geometry}
\usepackage[utf8]{inputenc}
\usepackage{color}
\usepackage{indentfirst} %Comment out for no first paragraph indent
\usepackage[parfill]{parskip}
\usepackage{setspace}
\usepackage{tikz}
\usepackage{amsmath}
\usepackage{amsfonts}
\usepackage{amssymb}
\usepackage{enumitem}
\usepackage{outlines}

\usepackage{fancyhdr}
\pagestyle{fancy}
\cfoot{\hyperlink{content}{\thepage}}
\lhead{}
\chead{}
\rfoot{}
\lfoot{}
\rhead{}
\renewcommand{\headrulewidth}{0pt}
\renewcommand{\footrulewidth}{0pt}

\newcommand{\foot}[1]{\hyperlink{#1}{$_#1$}}

\usepackage{hyperref}
\hypersetup {
	colorlinks,
	citecolor=black,
	filecolor=black,
	linkcolor=black,
	urlcolor=black
}
\newcommand{\sepitem}{0pt} %Added room between items on the list, not including a list and its sublist
\newcommand{\seppar}{0pt} %Between items and lists overall

\setenumerate[1]{itemsep=\sepitem, parsep=\seppar}
\setenumerate[2]{itemsep=\sepitem, parsep=\seppar}
\setenumerate[3]{itemsep=\sepitem, parsep=\seppar}
\setenumerate[4]{itemsep=\sepitem, parsep=\seppar}

\newenvironment{outline*}
{
	\begin{outline}[enumerate]
	}
	{\end{outline}
}

\begin{document}

\title{AP Comparative Government}
\author{Avery Karlin}
\date{Spring 2016}
\newcommand{\textbook}{Ethel Wood's AP Comparative Government}
\newcommand{\teacher}{Trainor}

\maketitle
\newpage
\hypertarget{content}{\tableofcontents}
\vspace{11pt}
\noindent
\underline{Primary Textbook}: \textbook\\
\underline{Teacher}: \teacher
\newpage

\section{Chapter 1 - Introduction to Comparative Government}

\subsection{Comparative Method}
\begin{outline*}
\1 Government is the leadership and institutions which make national policy decisions, while comparative government is the study of the flow of power from different people and groups within a government
\2 Politics are the activities associated with the governance of a country or area, especially the debate or conflict among individuals or parties with or hoping to gain power
\1 Political science, as a social science, can either be done based on empirical data, or facts and statistics, or normative issues, which are based on value judgements
\2 Since it is a science, it requires a hypothesis of a relationship with variables, to find a causation from the independent to dependent variable, such that one causes the other
\2 Correlations are when the change in both variables are simultaneous, implying the possibility of a causation, but not acting as evidence
\1 The main comparative model is the three-world approach from the Cold War, dividing into the first world of the US and its allies, the second of the USSR and its allies, and the third world of economically underdeveloped, unaffiliated nations
\2 It is used in modern day, based on communist, post-communist, and capitalist nations, advanced, economically developed democracies, and developing nations
\2 Developing and communist nations are more likely to become authoritarian nations, rather than economically developed capitalist nations, which are likely to be democratic
\2 This model integrates political and economic systems, due to the economy having a strong factor in citizen interaction with the government, allows observation of the impact of political change since the fall of the USSR, and the impact of informal politics, or the interaction of citizens and the civil society, or the organization and defining of citizen activism, with politics
\1 The Huntington model\foot{1} states the next great conflict will be between different civilizations (the broadest level of identity/cultural similarites), having moved from monarch territories to nations (from the 1790s to 1910s) to ideologies
\2 After the Cold War between Western civilizations, non-Western groups began to move away from being puppets of the West, responding to Western ideas of appropriate policies in the UN and IMF, to going back to traditional ideas, and due to increased interaction between civilizations
\3 Educated foreigners are now being educated in their own culture, while Western culture spreads through globalization, reversing past trends
\3 Finally, trade has begun moving backward toward mainly regional in recent years
\3 Most particularly the Western idea of a universal civilization contrasts the Eastern idea of particularism and differences between civilizations, forcing civilization to bandwagon with the west, isolate themselves, or modernize to create a balance of power (though only Japan has done it without moving towards Western)
\2 The civilizations include Western, Confucian, Japanese, Slavic-Orthodox, Hindu, Islamic, Latin American, and African civilizations, due different views on relationships, rights, hierarchy, and religion, such that intellectual debate is not possible
\2 Economic progress and social change has also weakened the power of nations, resulting in religious divisions taking over, leading to fundementalism
\2 This is also the movement back to the original, unchanging institutions, rather than ideological divisions, spreading both influence between civilization and local territory along fault lines
\2 This has manifested most commonly in economic rivalries such as US-Japan or US-China and civilization support for minor conflicts, but also in ethnic clensing, Islamic fundementalism, movement from democratic institutions, and military conflict
\3 In addition, the double standard of exempting similar nations from human rights regulations, while condemning others, leads to conflict
\3 This leads to Western attempts to ban the production of non-Western weaponry, in an attempt to not hurt their interests, while other nations define it as equal protection, leading to middle eastern and eastern Weapons States
\2 Torn countries between multiple civilizations must have the poltical and economic elite, the general public, and the majority of the new civilization agree to be able to take a new identity, found within Mexico
\3 On the other hand, in the case of Russia, none are present with the Post-Cold War attempts to join the West
\end{outline*}

\subsection{Sovereignty, Authority, and Power}
\subsubsection{Nation-States}
\begin{outline*}
\1 States are organizations that define the use of violence within a specific territory, through military and weapon restrictions, with institutions to create policy and promote general welfare
\2 States thus have soverignty, or the ability to create their own policies without influence
\2 States without soverignty are subject to corruption, used by internal and external organizations for their own ends, often in undeveloped nations
\1 Nations are groups of people with a common political identity, such that nationalism is the send of belonging, often resulting in patriotism for the nation
\1 Nation-states are the main form of a state, such that borders are drawn around a specific nation, providing the identity for those in the main nationality
\2 Bi/multi-national states are those containing multiple nations, such as the USSR, such that minority groups began protests for independence, decaying into nationstates, though the same issue has applied to the multinational Russia since
\2 Stateless nations are those without a state, such as the Kurds, often causing fierce nationalism
\2 Nations generally expanded from core areas, until they reached another nation-state, creating boundaries, with periphery areas around the core areas, with more open land and fewer towns
\2 Multicore states often have inner-conflict as the result of multiple groups having competing interests, and can hurt stability, such as in Nigeria, though often not, like the US
\end{outline*}

\subsubsection{Governmental Regimes}
\begin{outline*}
\1 Regimes are the sets of rules that states set and follow, generally divided into authoritarian and democratic regimes
\1 Democratic regimes can either be indirect, electing representatives for the people, or direct, with individuals directly having a say in government, generally only direct with small populations
\2 Parliamentary systems are those where the legislature is elected, and those officials determine the executive, while presidential systems have both elected, with seperation of powers between
\3 Thus, in a parliamentary system, the prime minister doesn't have the same monopoly on power, rather ``first among equals'', such that it is harder to lobby, due to requiring a majority
\3 Cabinet members are also taken from the legislature, with a shadow cabinet of members of the largest opposition party
\3 In addition, since there is often more than two parties, they require the support of a third party to create a coalition government to get the majority needed, offering positions and policies
\2 While there are different levels of economic regulation, democracies have independent corporations from the government
\2 Most democracies are divided into a legislative, executive, and judicial branches
\2 Semi-presidential systems can also exist, such as in the 1993 Russian constitution, with both a parliamentary prime minister and a president sharing power, though the president has taken far more under Putin
\2 Democratic regimes rely on pluralism, or power split among many interest groups attempting to influence
\1 Authoritarian regimes have power held by the political elites without citizen input, either by a dictator, hereditary monarch, aristocrats, or single political party, controlling both the government and economy
\2 In these societies, there is no limits on the power of the leaders, responsibility to the public, or restriction of civil rights
\2 Communist countries are controlled by the party, controlling all aspects of life, following Marx or Zedong economic philosophy
\2 Corporatism is the supervision of government policies by some labor or business group, though it may be some other patron-client system
\3 Corporatism often results from authoritarian regimes trying to provide the appearence of citizen involvement to gain co-optation, or citizen support, while banning other groups 
\3 Patron-clientelism is the system of benefits provided to a specific group in exchange for vocal support
\3 It can also result from economic regulation or nationalization of industries resulting in close ties between government and industry
\3 Democratic corporatism can be shown by recognition of specific groups by the state, while forcing others to require recognition, legally bound to the state, working on behalf of the state
\2 Totalitarian regimes are a subset of authoritarian, whcih attempt to control all aspects of political and economic systems, often based on a strong ideological goal, such as communism
\3 Totalitarian governments especially use violence to remove opposition, and are more illegitimate, in that they are not accepted by the people, which authoritarian governments may be
\2 Military rule is a common form of authoritarian, often taking power in a forced takeover/coup d'etat during unrest, generally with public support, restricting civil liberties to preserve order, joining with the bureaucracy
\3 This can lead to a weak state, forcing other coup d'etats, in a series of weak regimes
\1 The Democratic Index was published by the Economist since 2007, ranking countries based on the electoral process and pluralism, civil liberties, government functioning, political participation, and political culture
\2 It also catagorizes into democracies (like the UK), flawed democracies, authoritarian (like Nigeria, Russia, China, and Iran), and hybrid regimes (like Mexico)
\end{outline*}

\subsubsection{Legitimacy}
\begin{outline*}
\1 Legitimacy, or the right to rule, is determined by the citizens, and is catagorized as either traditional, charismatic, or rational-legal
\2 Legitimacy is easier to maintain in economic prosperity and with high government performance approval
\1 Transitional is based on tradition, such as hereditary rulers, often based on myths, legends, religion, or divine right, with ceremonies, symbols, and artifacts to encourage the idea of legitimacy
\1 Charismatic is often based on personality or military talent, such as Napoleon, though when he lost militarily, it faded, generally a shortlived form of legitimacy, unable to be passed on after death
\1 Rational-legal is based on institutional laws and procedures, preserved through belief in the rule of law and acceptance of the authority of the state, such that shared political culture is important
\2 It can be based on common law, or legal tradition and precedents
\2 Legitimacy of rational-legal in democratic governments can be the result of the loss of the legitimacy of the electoral system
\1 In modern states, the main form of legitimacy is from rational-legal, though traditional and charismatic allow easier gain of power, or influencing politics easier within interest groups
\2 Many states also preserve some form of traditional legitimacy, to add legitimacy to the legal-rational democratic form of government
\end{outline*}

\subsubsection{Political Culture and Ideologies}
\begin{outline*}
\1 Political culture is the collection of political beliefs, values, practices, and institutions which a country is based on, such that for a government to remain, it must be based on that culture
\2 Social capital, or reciprocity and trust between citizens and the state or other citizens of all levels, can be used to measure how democratic it is, such that more democratic makes it greater
\3 On the other hand, social capital theory, which predicts difficulty in Islam or Confucian regions, has been critiqued for ignoring countries such as Turkey or India
\2 Consensual political culture is agreement on what issues should be solved and the process by which decisions are made, such that legitimacy of the government is accepted
\2 Conflictual political culture by fundemental economic, religious, or polticial differneces often leads to conflict, and prevents effective rule
\1 Political ideologies are sets of political values of the basic goal of government, held by the individual
\2 Liberalism values political and economic freedom, maximizing rights and freedoms, and allowing citizens to disagree with the state and attempt to influence decisions
\2 Communism values equality, believing freedom won't create general prosperity, believing eventually a wealthy class will form and take control of the government, advocating state control of all resources to protect economic equality
\2 Socialism values a combination of freedom and equality, believing in the free market and private owenership, but believes the state have heavy control of the economy to provide benefits and preserve equality
\2 Fascism values strength, believing some groups are inherantly inferior, attempting to create the strongest possible state, such that rights must be taken away by the authoritarian state to preserve it
\2 Religious ideologies also play a large role in many nations, often having an official state religion, or having special interest groups influencing it
\end{outline*}
\subsection{Political and Economic Change}
\begin{outline*}
\1 Change generally happens both politically and economically simultaneously, and most countries experience it over time, but when happening seperately, creates tensions
\1 Change can occur through reform, attempting to use standard political and economic institutions to create change
\2 Revolution attempts to change the political and economic institutions through the overthrow or revision of the institutions, generally impacted economic, political, and social systems, regardless of the intent
\2 Coup d'etats replace the government with new leaders by force, often carried out by the military, but can cause instability
\1 The strongest attitude toward change is radicalism, or the belief in rapid, dramatic changes, often believing the institutions cannot be fixed, leading to revolutions
\2 Liberalism as an attitude is the belief in reform and gradual change, beliving in repairing and improving existing systems, with the goal often of leading to a complete transformation over time
\2 Conservatism believes change is disruptive and causes unexpected negative outcomes, believing in the need to preserve legitimacy of government, basic societal values, and law and order
\2 Reactionaries believe that the current state has already move too far from basic societal values, wanting to use revolutionary means to return to old institutions
\1 The first major trend of modern change is democratization, based on the idea of competitive elections, with many countries moving further to liberal/substantive democracies, instead of illiberal/procedural
\2 Liberal democracies have belief in neutrality of the judiciary, checks and balances on power, civil liberties, rule of law, civilian control of the military, and open civil society
\3 Illiberal democracies often have an unchecked executive and restricted citizen groups, preventing truly free elections, but are necissary before a society can become a liberal democracy
\2 Huntington believes there are three waves of democratization, the first gradual until WWII, the second after WWII involving de-colonization, and the third involving the defeat of totalitarians after the Cold War
\2 Democratization is due to the legitimacy of authoritarians, expansion of urban middle class, human rights emphasis, and international snowball effect
\2 Democratization happens after a trigger event taking place, after a revolution of rising expectations of high living standards, causing democratic consolidation of the elites and public willing to share power, spreading throughout society called political liberalization
\1 The second trend is economic liberalism, moving to market economies, such that it is under debate, due to influence such as China, if democracy and market economies inherantly move together
\2 19th century European reformists were generally middle-class bourgeoisie, who wanted their views represented in government, and economic goals unrestricted to allow economic mobility
\2 Radicals, on the other hand, believed that freedom clashed with equality, and thus a free market was not the ideal, including Marx, beliving instead of a command economy of government owned businesses
\3 These economies, in the USSR and China, had a state planning committee, with economic production blueprints and quotas in 5 year plans
\3 On the other hand, these generally, while creating economic growth, did not lead to higher living standards
\2 In recent years, most command economies have moved toward market ecnomies with less government regulation, with the current debate between a mixed (with significant regulation and control from government) or market economy
\2 Economic liberalization is based on the failure of many command economies and the belief that government is too large
\3 Thus, many command economies have had marketization toward market economies and privatization toward private ownership
\3 The main downside, the business cycle, has led most to adopt a mixed economy to lower the dangers of the business cycle
\1 The third trend is fragmentation, or divisions based on culture/ethnicity, moving away from prior globalization toward nationalism, especially found in the politicization of religion and increase in fundementalism
\2 It is argued that those who believe in the clash of civilizations underestimate this factor of cultural differences
\end{outline*}
\subsection{Citizens, Society, and the State}
\begin{outline*}
\1 Social cleavages are divisions in society that are outside politics, but impact political policymaking based on causing deep political identification, including social, ethnic, religious, and regional cleavages
\2 Coinciding cleavages are those which divides the same groups against each other on issues, while cross-cutting cleavages are those which divide groups that agree on some issues, but disagree with others
\2 Regional conflicts are often the results of different levels of economic development, or religious divisions between regions
\2 The depth of the cleavages in the social structure of a society, the level of political party alignment with cleavages, and specific cleavages not involved in the political process determine the importance
\1 Government-citizen relationships are based first on political efficacy, or the citizen ability to understand and influence political events, determining if they feel the government cares about their opinion
\2 This works to create active voting behavior, and political participation, rather than interacting purely through subject activities (obeying laws), creating political attitudes
\2 Attitudes in other respects, such as trust or the ability of the government to impact their lives also play a role
\2 Government transparency also changes interactions, preventing corruption, as well as the methods of learning about political actions to create immediate views
\1 Social movements are organized, collective activities to create desired policies, using nontraditional reform methods and bring non-mainstream positions to mainstream society
\1 Civil society are voluntary organizations outside the government to aid identification and advancement of personal interests, encouraged in liberal democracies
\2 They can be either political or apolitical (not politically active), rather just to promote goals and interests, preventing tyranny of the majority
\2 It has been argued that globalization has led to cosmopolitanism, or a universal political order and civil society based on worldwide identity and values, found within international, political, nongovernmental organizations (NGOs)
\2 Authoritarian nations are against civil society, dividing purely based on social clevages
\3 Civil societies are often formed later through civil education of democratic rights, and through NGO involvement
\end{outline*}
\subsection{Political Institutions}
\subsubsection{National Forces}
\begin{outline*}
\1 Political institutions are structures which carry out governing, though they cannot be assumed to have the same powers in each nation
\1 Unitary systems have all policymaking centralized in one location, while federal systems divides between central government and sub-units, and confederal has power almost purely in sub-units, with a weak central
\2 Federal and confederal systems are criticized for inefficiency, due to local governments with possibly competing interests, such that very few governments are confederal for that reason
\1 Supernational organizations affecting national policies are a result of integration, or the loss of soverignty to gain international influence, based on shared policies and rules, such as the UN
\2 This results in a relationship between domestic policies and international relations, creating additional international trade, banking, assets, and foreign direct investment
\2 It also creates a ripple effect of international events
\2 On the other hand, integration ironically causes fragmentation, by iviing the world in regional international organzations, such as the EU
\1 Centripetal forces are those which bind states together, such as nationalism, or identities based on nationalism, encouraging belief in laws and patriotism, using schools, symbols and holidays to promote it
\2 It is also encouraged by transportation systems and technology, uniting the parts of the country with each other and its government
\1 Centrifugal forces include organizations rivaling the government for  influence, such as the church in the USSR, or nationalism if leading to seperatist movements or devolution
\2 Centifugal forces can often lead to devolution, or decentralized decision-making to regional governments, moving toward federalism, even in long-established states
\3 Devolutionary forces can include ethnonationalism, or the feeling of an ethnic group as a seperate nation, with the right to autonomy, especially if the ethnic group is concentrated in a specific region, due to ethnic groups having a shared culture, language, customs, and religons
\3 Regional economic inequality or peripheral location, especially if cut off geographically, can also be a strong devolutionary force
\2 Seperatist movements to fully break into a seperate nation, are generally ethnic, based in nationalism, and can be encouraged by peripheral location, socioeconomic inequality
\end{outline*}
\subsubsection{Government Branches}
\begin{outline*}
\1 The executive carries out laws, split into the head of state (symbolizing/representing the people at home and abroad, often a figurehead, often called the president) and the head of government (often the prime minister)
\2 The chief executive begins policy initiatives, often given veto power in a presidential system, makes foreign policy crisis decisions, and oversees the execution of laws
\2 The cabinet in a parlimentary system is led by the ``first among equals'' prime minister, taken from the legislature to run debartments, formed from a cabinet coalition in a multi-party system with no majority
\3 Cabinet coalitions can often lead to instability
\3 In presidential systems, the president chooses the cabinet, approved often by the legislature, often more independent than in a parlimentary system due to not being major political figures
\3 On the other hand, in a presidential system, the president can remove them from the cabinet if they disobey his wishes
\2 Bureaucracies are agencies to implement government policy, viewed by Weber as necissary to respond to a changing society, growing as the role of government grew
\3 He stated all bureaucracies must be complete meritocracies with clear goals, extensive, well-established rules, task specialization/division of labor, a hierarchy
\3 On the other hand, he feared lack of meritocracy, found in the US by the patronage system until reformed after the Garfield assassination, and discretionary power to make small decisions, against democratic beliefs, but acknoledged they provided stability due to being unelected
\3 In authoritarian regimes, the head of government has complete power,and uses bureaucracy to directly control many aspects of life, with large amounts of patronage
\3 In many Latin American countries, the military regime formed a technocrat of civilian and military bureaucracy coalition, controlling government in the name of rapid modernization
\3 Realistically, bureaucracies have characteristics of non-elected positions, efficient/partially meritocratic structures, job qualifications, hierarchy, and inefficient red tape (especially in large bureaucracies)
\1 Legislatures are governing bodies, popularly elected, although in authoritarian regimes, are controlled by the executive
\2 Legislatures are either bicameral (two houses) or unicameral, often the former to allow an upper house for regional governments and a lower house, directly for the population, balancing regional powers
\3 It can also be used in non-federal systems to allow a house further from the people, and thus less impulsive, to moderate decisions
\2 Legislatures often create, debate, and vote on policies, have taxing and spending power, appoint officials to other branches, serve as appeals courts, impeachment courts, and act as elite recruitment for future government leaders
\1 Judiciaries serve in authoritarian regimes under the control of the executive, simply as legal courts or rubber-stamps
\2 Constitutional courts, defending the democratic principles of a country, often have judicial review to rule on constitutionality of government actions, and have the power to protect against other citizens infringing on rights
\2 Courts have been critisized for being unelected though, and are often weaker than the other branches 
\end{outline*}
\subsubsection{Linkage Institutions}
\begin{outline*}
\1 Linkage institutions are groups which connect government to citizens, such as parties, media, and interest groups, larger in nations with a larger government and population
\1 Political parties provide labels for voting, hold politicians more accountable, and bring different people and ideas to a united group
\2 Two-party systems provide a plurality electoral system, while multi-party provide proportional generally, the latter being more standard
\1 Electoral systems determine how votes are cast and counted
\2 In the US, UK, and India, the first-part-the-post system is used/plurality/winner-take-all system is used, competing for a single seat
\2 Many nations use a proportional system, creating multi-member districts, voting for a party instead of a candidate, and others use a mixed system combining the two
\3 Proportional systems also encourage coalitions to form a majority to get legislation passed
\2 Elections fall into either an election of public officials, referendums to vote on a policy issue (called by the government), or an initiative to vote on policy (called by a petition of citizens)
\3 Plebiscites are non-binding referendums to gage public opinion
\1 Interest groups are organizations attempt to influence public policy, existing independent from the government based on a common interest
\2 Nonpolitical groups can also be interest groups, seeking to advance a private or corporate interest
\2 Interest groups, unlike political parties, do not run candidates to influence the process, but may support candidates
\2 In a multi-party system, they are more similar, in that single issue parties often form, rather than broad platform parties
\2 In authoritarian states, interest groups must be government-sanctioned, acting as transmission belts to extend party influence to members of the group and the general public
\2 Interest group pluralism has complete independence, getting their own funds and leaders
\2 Corporatism has fewer groups compete, with each group generally having a monopoly over the sector, sanctioned and often protected by the government, in between the other two systems
\3 State corporatism has the states decide which groups take control of each sector, while societal corporatism has the interest group form the monopoly, and gain its own power within the state
\1 Political elites, or leaders with high amounts of political influence, are found in every system, such that there must be methods of recruitment to find future elites
\2 Further, all nations must have a system of political succession, or replacing ineffective or resigning leaders
\end{outline*}
\subsection{Public Policy}
\begin{outline*}
\1 Policy is created by the three branches of government, interest groups, and political parties, to solve general issues
\1 Economic performance is one of the most important issues, affected by both international and domestic trade
\2 This is measured by GDP, Gross National Product (GDP + Income Earned by Citizens Outside the Country), GNP per capita, or Purchasing Power Parity
\1 Environmental issues are also a major modern problem, especially in Europe, leading to green, environmental parties and international interest groups
\1 Social welfare, such as providing health, employment, education, and family services, are important factors
\2 These are measured by literacy rates, income distribution, education levels, life expectancy, Gini Index for economic inequality, and the Human Development Index (measuring the well-being of citizens by a variety of social welfare factors and GDP)
\1 Civil liberties, or the promotion of freedom, and political rights, or the promotion of equality, involve government protection
\2 These range based on the amount of rights preserved in addition to levels of government involvement, and are often guaranteed by constitutions in liberal democracies
\2 Freedom House ranks nations from 1 (most free) to 7 as a measure
\end{outline*}
\section{Chapter 2 - United Kingdom}
\subsection{Advanced Democracies}
\begin{outline*}
\1 Advanced democracies are those with a long history of stable democracy, measured by the political characteristics of liberal democracy and economically
\2 They have legitimacy from the history and large social capital, or trust between citizens and with the state
\2 In most nations, citizenship is required to vote, though in Scandinavia, permanent residents can vote, and in most states except the US and France, the state automatically registers voters
\1 Economically, they are post-modernist, with values of environmentalism, education, and health care, as well as post-industrial, due to the large amount of tertiary/service sector, such as tech, legal, business, or financial services
\2 Modernist civilizations have values of secularism, rationalism, materialism, technology, bureaucracy, and freedom over equality, with a large amount in the industrial/secondary sector, due to being industrial
\2 Pre-industrial have a majority of workers in the primary/agricultural sector
\1 In advanced democracies, there has been a move toward supranational integration organizations, especially in Europe by the EU, but also by the North American Free Trade Agreement
\2 This worked to promote international cooperation over nationalism, with the EU adopting a common currency
\1 The UK began democratic ideals of limited monarchy, started industrialization, democratic institutions, and spread them through the world by colonialism
\2 On the other hand, it's economy has slowed since WWII, and lost its colonies, but maintained a powerful worldwide influence, currently transforming into a more European nation
\end{outline*}
\subsection{Soverignty, Authority, and Power}
\begin{outline*}
\1 Legitimacy grew over time based in tradition, originally based on the idea of a traditional, hereditary monarch, but was limited over time, until Parliment was in charge in the 1600s
\2 Except in Northern Ireland religious conflicts, seperation of church and state are engrained
\2 The Constitution of the Crown of documents, common law, and customs take the place of a standard constitution
\3 The Magna Carta was signed in 1215 by King John, limiting the monarch to consulting nobles before major policies
\4 Although it was mainly ignored by the King, and failed to prevent war between the king and the nobles, and was condemned by both the Pope and the King immediately, it created a tradition of democracy\foot{2}
\4 It gave rights to fair, speedy trials, no taxation without representation, and stated the king wasn't above the law\foot{2} (rule of law over arbitrary rule\foot{3})
\4 It also protected the church and confirmed local privileges, still protected in the UK today, although most of the remaining sections are archaic\foot{2} and ignored local peasent representation, and providing rights mainly only to free men, but all groups rallied around it by the 1300s, supported later by the Church due to their protection\foot{3}
\4 In 1216, France invaded to support the Lords, such that it was brought back by loyal lords to protect his son, Henry, as king\foot{2}
\4 After, it was revised several times and not enforced properly, but worked to symbolically preserve a limited monarchy\foot{2}
\4 The ideas within it were not new, found throughout Europe, but were the most detailed restrictions forced on a ruler, due to absurd taxes, complete lack of due process and rule of law, and arbitrary executions\foot{3}
\3 The Bill of Rights was signed by William and Mary in 1688 after the monarchy was restored following the Glorious Revolution after the English Civil War executed Charles I, giving major policy powers, including tax and spending, to Parliament
\3 Common law rule is based on public officials and courts creating precedents instead of a legal code, used limitedly in the US
\1 The development of political culture began with the shaping of the monarchy, followed by the rise of the Parliament (the execution of Charles I, and the installation of the Roundhead/Pro-Parliament leader, Cromwell
\2 Charles II was installed shortly after, until the Glorious Revolution forced the signing of the Bill of Rights to protect Parliament, creating the Prime Minister as head of government by PM Walpole in the 1700s
\2 In the 1700s, the Industrial Revolution replaced the feudal system with merchantalism, creating a middle class of merchants, forcing their inclusion in politics
\2 In the 1700s and later, industrialization and nationalism led to colonialism gaining raw materials abroad to produce goods at home, and allowing communication and transport between distant locations
\2 In the 20th century, world wars and a welfare state hurt the economy, leading to a urban and infrastructure decay, and a neo-conservative backlash of Thatcherism, followed by Blair's Liberal Third Way, attempting to find a balance between welfare and capitalism and acclimating to less world influence
\1 UK Political culture is based on a large amount of nationalism and insularity, or feelings of seperation from Europe, leading to fear of the EU, refusing the Euro even after joining
\2 It also includes noblesse oblige, or duty of the nobles to help the lower classes, dating to feudalism, which led to the welfare state and colonialism to help the lower races overseas
\2 Multi-nationalism is also found, due to the multiple nations united, even with the cultural homogeneity within each, except for religious differences in Northern Ireland
\1 The UK has fewer political-related crimes and a smaller police force, due to respect for the law, except in Northern Ireland for independence and recent international terrorist issues
\end{outline*}
\subsection{Political and Economic Change}
\begin{outline*}
\1 Political change was based in gradualism, or slowly over time, which served to create strong traditions, such as the transition from 1066, requiring noble support by William the Conqueror to become king, until the Bill of Rights
\1 During the Industrial Revolution, the middle class and laborer class were created, and over time, noblesse oblige led to incorporation into the political system
\2 The Great Reform Act of 1832 gave more power to the House of Common and 300k men suffrage, Reform Act of 1867 gave 3M, Representation of the People Act of 1884 further, and Representation of the People Act of 1918 gave all men, and women over 30 suffrage, extended in 1928 to over 21
\2 In the late 1800s, unions and social services, such as mandatory elementary school and pensions were made, further preventing class anger and the rise of Marxism
\2 By the 1900s, the House of Lords only had the power to delay laws, with the House of Common in power
\1 In 1906, the Labour Party represented the working class, with the Conservatives taking middle-class merchants, leading to welfare, housing, public education, and healthcare reform, supported by Labour
\2 Labour took power as the main liberal at this time, pushing the moderate Liberal Party as a 3rd party
\2 Labour then began to act in favor of social democratic ideology and militant trade unionism, supported since by the Trade Union Council coalition, but never reached Marxist revolutionary
\1 WWII led to widespread infrasructure destruction and war debt, and while the Marshall Plan allowed economic recovery, the UK began to prepare colonies for independence, keeping ties, but ending imperialism
\1 The war also led to the collective consensus of the parties joining together in favor of a welfare system, accepting the Beverage Report to provide unemployment, health, pensions, and income to every citizen
\2 In 1948, the National Health Service was made by the Labours, and while the classes were split by party, all supported reforms
\2 During ecnomic crises of the 1970s, Labour moved to the far-left, supporting socialism, and Conservatives went to the far-right, supporting a free market and denationalization of all industry
\3 Industry downturn, loss of colonies, and OPEC oil embargo led to the economic downturn, leading to higher wage demands, and strikes, such as the Coal Strke of 1972
\3 Many voters left the Labour Party, blaming the welfare state and union power, and elected Thatcher, the Iron Lady, strengthening defense, privatizing, moving toward neoliberal theory of low government regulation
\3 On the other hand, many felt she hurt and divided the nation, and disliked her firm personality, challenged in 1990, but still lowered the power of the welfare state
\2 Major, Thatcher's chosen successor, followed her policies, but removed the poll tax, joined the EU, and slowed privatization, but the Cosnervatives still lost the margin
\3 Blair won in 1997, promising a move to the Third Way, more centrist, but after Iraq War support, lost popularity, and Brown took his place, but the recession hurt the Labour Party
\3 In 2010, Cameron formed a coalition with Liberal Democrats and took power, creating the Big Society of grassroots and private organizations over big government
\end{outline*}
\subsection{Citizens, Society, and the State}
\begin{outline*}
\1 The UK is mainly homogeneous, with only 7\% as a minority, but has major social cleavages between Protestant/Catholic, social classes, and multi-national identities
\2 The main region of Great Britian is England, historically holding the majority of the population and political power
\2 Wales is the Western region, became under the English king in the 1500s, with Welsh pride in the Plaid Cymru flag and the language, still taught in schools, though there is some resentment
\2 Scotland is the Northern region, joining by intermarriage in the 1600s, still resisting rule, not agreeing to a single parliament until the 1700s, but have a lot of nationalism and their own flag
\3 They also recently remade their independent Parliament, with the idea of independence under much debate with recent referendums
\2 Ireland was fought over, from Cromwell's attempt in the 1600s to push Protestantism on Ireland, eventually giving home rule in WWI to the all except the NE (Northern Ireland)
\3 This was mainly due to the Irish Republican Army using guerilla warfare, and the religious conflict is still present today
\1 There is a strong divide between the lower and middle class, based on the idea of solidarity, staying within your own class, found also among the upper class
\2 The division was furthered by expensive public boarding schools, training for public life of civil/military/political service, training the elite, before going to Oxbridge (Oxford or Cambridge)
\3 A majority of the government officials attended Oxbridge, though since WWII, scholarships have been given more to lower classes
\2 Middle class students go to private non-boarding grammar schools, but until 2008's Education and Skills Act, had compulory education only until 16 years old
\3 Other universities were available to the middle class, but in 2012, the price was raised for these by Parliament, making it less accessible
\1 Minorities mainly come from former colonies, growing by 53\% in the 1990s, but still low, mainly Indian, then Pakistani, Afro-Caribbean, and then Black African
\2 Past immigration restrictions, especially under Thatcher, make it so most minorities are young, half below 25, working in recent years to half immigration, though open EU borders make it only apply to non-Europeans
\2 There has been tension with minorities by police racism and verbal harassment from citizens, as well as events like the May 2001 race riots
\3 This was caused by the murder of a black man by the police, and white flight has been seen in London, creating segregation, though mixed-race population has been growing
\2 Muslims especially are seperate, as distinct immigrant minority, leading to to several terrorist attacks since 9/11
\3 This is also caused by strong support of the government to the Iraq War against Islam, and the fact that the majority are Pakistani, with stronger links to Bin Laden than other groups
\3 It is also due to lack of general minority integration into mainstream culture, even with more religious rights than other nations, and far more common poverty and illiteracy among Muslims
\2 Eastern Europeans also began immigrating due to the EU open borders (or lessened restrictions until 2014), such that Polish are now the majority of foreign nationals, though the financial collapse had some return
\3 Many are migrant workers, doing low-paying rural work, creating fears of increasing unemployment among the lower classes
\3 The recent refugee crisis along with EU policies guaranteeing social services to refugees, has led to a move from the EU, petitioning for stronger immigration, and limits on services to deincentivize immigrations among both the Conservatives and Labour (working class)
\3 The EU ideal of a political single-entity has also created resentment among the British government for the political aspects of the EU, for fear of losing soverignty
\1 Until the 1970s, political/civic culture was based on trust, pragmatism, harmony (acceptance for other points of view and the legal sstructure), deference to competent authority, and political participation
\2 Since, the idea of a collective consensus has been replaced by politics of protest, with less tolerance, more open disagreement and violent protest against governmental policy
\2 This is due to loss of union support during the 1970s, when many strikes hurt the economy, and the idea of Thatcherism of free market economy, individualism, and competition 
\3 In addition, Northern Ireland violence, especially the murder by military of 13 Catholics in Bloody Sunday in January 1972, and IRA/protestant miliants, has hurt these ideals
\3 The Iraq War also led to protests among members of parties, hurting complete party support, even among cabinet members, leading to Blair's resignation in 2007
\2 On the other hand, New Labour, with looser ties to unions and a looser grasp on Northern Ireland Good by the Good Friday agreement, preserved older values
\1 The UK has more than 70\% of voters vote in parliamentary elections, but with less party loyalty than previously, voting along social and regional lines
\2 The Labour party has power in industrial regions, Scotland, Wales, and London, while Conservatives are generally in rural and suburban areas
\2 Working class is generally Labour, while middle is Conservative, though with increased mobility, many middle-class vote Labour because their parents were lower, but many working-class have been anti-immigrant and anti-taxation, voting Conservative
\2 Labour also takes the disadvantaged, or Scots and Welsh, but recent weakening of both parties due to becoming moderate has allowed third parties to gain a foothold
\end{outline*}
\subsection{Political Institutions}
\subsubsection{Linkage Institutions}
\begin{outline*}
\1 Political parties originated in the 1700s, beginning as caucuses, or meetings of people with similar ideologies, where the Tories and Whigs originated under Charles II, the former supporting the king, the latter not
\2 The Whigs were used as slang for Scottish bandits, Tories for Irish bandits, eventually becoming the Liberal and Conservatives respectively
\2 The Labour party would later form during the early 20th century, based on the new industrial worker demands during the Industrial Revolution
\3 It began in 1906 as an allience of trade unions and socialist groups, though Blair weakened union ties during power from 1997 to the 2010 loss to the Conservative coalition
\3 Clause 4 called for nationalizing the commanding heights of industry, such that the third way was shown by the removal of the clause
\3 The loss under Kinnock in 1992, in power since the 1980s, led to his reignation and appointment of the moderate Smith, a Scotsman to get support of Scottish nationalist, dying in 1994
\3 Blair was an educated upper-class, to bring intellectuals and middle-class to the party, in power until 2007, winning during that time, but with a smaller margin in 2005, leading to his resignation in 2007
\3 Brown then took power, but lost further ground, until he resigned in 2010, and Milliband took over, moving far to the left, but losing support to the Conservative-Liberal coalition
\2 Conservatives were the majority mostly from WWII to 1997, based on pragmatism rather than ideology, characterized as London-based and elitist with MPs choosing leadership, only recently adding elections
\3 The traditional (one-nation) wing is based on noblesse oblige, with the elite ruling for the good of all, supporting the EU
\3 The Thatcherite wing is based on free market and lack of welfare, generally Euroskeptics against the EU
\3 Cameron, leader since 2005, has taken power back as Blair became weaker, taking back the majority recently, losing ground temporarily after the takeover of Brown, but gaining it back, acting as a one-nation Tory
\2 Third parties also have strong power, such as the Liberal Democratic Party in the 1980s, though the single-member plurality system creates little representation in Parliament
\3 On the other hand, in certain times such as 2010, a hung parliament of a coalition government rather than a majority has occured
\3 Cymru's Welsh National Party and the Scottish National Party also has gained seats in the regions, though most went to Labour, but after Blair made regional assemblies, they gained large numbers of seats within those
\3 Northern Ireland has especially powerful regional parties, such as the Protestant clergymen Democratic Unionist Party and the IRA Sinn Fein
\3 The British National Party and UK Independence Party, the former antisemetic and anti-Muslim radical right, the latter radical right anti-EU, have gained seats since 2009, after the weakening of Labour, the former never having seats before, the latter becoming the 4th largest
\2 The Liberal Democrats formed from the merge of the Liberals and Social Democrats in 1989 after 6 years of alliances, with the goal of making a moderate party
\3 Due to the plurality system, with 26\% of votes in 1983, they won 3.5\% of seats, and have campaigned for a proper bill of rights and proportional representation
\3 Ashdown held the party together as the other parties moved moderate with strong education, environment, and health stances, and under Kennedy, gained a few seats as Blair lost support after the Iraq War
\3 In 2007, Clegg took power, critisizing the Labour Party for hurting individual civil liberties, gaining 10\% of seats with 24\% of votes
\3 In 2010, they formed a coalition with Cameron, and Clegg became Deputy Prime Minister as a result, but the coalition has issues due to Cameron being further right, especially on the House of Lords reform and EU
\1 Elections are held every five years, though the Prime Minister is able to call them earlier, officially after the Queen dissolves Parliament, realistically at the order of the PM to keep the majority in power
\2 On the other hand, recently, the Fixed-Term Parliaments Act requires a 2/3rd vote to dissolve, such that the minority typically must agree as well \foot{6}
\2 There is a strong civil society culture, with 70\% of eligible voters voting, unlike America which averages 45\%
\2 Elections are plurality with no run-offs, each party having the option to select a candidate to run in the district, winning even without a majority
\3 This systems tends to hurt minor parties and exaggerate the majority party, but allows regional parties to gain a foothold
\3 It is argued that broad parties provide a more stable government than minor issue party coalitions \foot{6}
\4 On the other hand, as issues become less one-dimensional like the economy, and more cultural, this becomes less sustainable
\4 It is still possible that a charismatic leader will reshape the parties to gain a majority, similar to Thatcher or Blair, people will move their votes back to the majority, or the majority party will change if the system is not reformed
\4 It is also especially likely as Labour gains seats due to the system, that Conservatives, who have been the main group to oppose change, will shift
\3 On the other hand, in recent years, the move toward third parties by a larger percentage of the population, discontented with the major parties and stagnant economy, leading to overall less loyalty, has shown the need for change \foot{6}
\4 This has also made seats less fixed, as third parties take away from the majority
\4 In stable seats, previously there has been almost no campaigning, leading to no loyalty to the politicians or parties
\3 As a result, in 2011, the Liberals called for an alternate vote referendum, with ranking candidates, and requiring a simple majority, or otherwise least popular votes would be redistributed to the second choice and so forth
\4 Labour's leader, Miliband, supported the alternative vote, but most MPs went against it, and it was defeated
\2 MPs are not required to live in the district, such that they are chosen by the party, with leaders running in safe districts, new politicians in unlikely districts to gain experience
\2 The April 1998 Good Friday Agreement allowed Northern Ireland a proportional-based regional government, later given to Scotland and Wales similarly, resulting in no majorities generally
\3 Similarly, local levels have become more democratic as well, such as the mayor of London now elected
\2 European Parliament elections are also done every 5 years, with 72 representatives from the UK, notably in 2009 with the UK Independence tied for 2nd place in seats with Labour
\2 UK campaigns are far cheaper than US campaigns, though investigations took place in 2006 for breaking a 1925 law offering peerages (House of Lords seats for money) and the 2000 law requiring disclosing personal loans
\3 Scandals involving the investigation were among the reason which caused Blair to step down as PM
\1 The UK has interest group pluralism, competing for influence independently from government, but also has neo-corporatism, occasionally gaining enough power to control the state
\2 Quangos, or quasi-autonomous nongovernmental organizations, in the form of government appointed policy advisory boards, work with government officials to develop policy and provide services
\3 These were weakened under Thatcher and Conservative leadership to reduce the deficit
\2 The Trade Unions Congress union coalition also has a lot of power, consulted by government before decisions, after the merging of unions, though there is a degree of competition
\2 The Confederation of Business Industries also has a lot of influence with government circles, but is a less complete business coalition
\1 British newspapers are divided between mass circulation tabloids and upper middle class quality news, while radio and TV were originally monopolized by the BBC to educate citizens and create a civil society
\2 In 2003, the BBC went against the government claim of WMDs in Iraq, leading to Kelly, the inspector/source, committing suicide, leading Blair to appoint an appeals judge, who blamed the BBC, leading the chairman to resign
\2 The BBC is still supported by a license fee by all households, though Conservatives have advocated for making it a more public, transparent corporation
\2 In 2011, it was discovered that tabloids, such as Murdock's News of the World, had been hacking phones, leading to Cameron's communications chief arrested, having recently left as EIC of the paper and included the London Metropolitan Police, who had ignored hacking for years \foot{4}
\3 This led to questioning of Cameron, even while he called an investigation, and a fight with the Sun and a Labour MP over topless women
\3 It was also questioned that the chief was exempted by Cameron from the background check, who believed his story of a lone wolf
\3 After the scandal, employees and ex-employees had legal costs paid by Murdoch, and recieved politician support by those close to Murdoch, especially Blair
\3 The police had previously threatened to investigate and surveil Davies after he brought proof of the hacking, and had a history of the police selling confidential conversations with officials to the newspapers
\2 It could be argued that newspapers have decided elections, and due to being under Murdoch, allowed corruption, but it could also be argued that they are influenced by circumstances, rather than vice versa \foot{4}
\3 On the other hand, deals such as the EU constitution referendum by Blair, likely to be voted against the EU, in exchange for support from Murdoch's Sun, show clear corruption, moving back after the election
\end{outline*}
\subsubsection{Government Branches}
\begin{outline*}
\1 The political system of the UK is parliamentary with the prime minister chosen from the legislature, with a judicial and bureaucracy, all within a unitary system
\1 The Cabinet Executive is made of the PM and other ministers, running the bureaucracy, chosen by the PM from Parliament
\2 The cabinet doesn't vote on decisions, but rather have collective responsibility, supporting the PM's decisions at all times
\2 The PM is the first among equals, but in coalition governments, must appoint minority parties as deputy PMs and other roles, and combine the platforms, while the other parties work as loyal opposition
\2 The job of the prime minister is to speak for Parliament, appoint cabinet ministers and bureaucrat heads, make cabinet decisions, and represent and head the party
\2 The parties are able to vote at any time to change the party leader, and as a result, the PM if the majority party
\1 Parliament (legislature) is bicameral, with the lower House of Commons, which is based on the idea of two parties/coalitions, the loyal opposition and the majority
\2 Commons is set up as two facing benches for each coalition, the third parties sitting on the side of the minority, the PM at the front of the majority side
\3 The minority party also has a shadow cabinet of those who would take power if the majority switched, sitting at the front of that side, while backbenchers are less important party members
\3 During weekly question time, the Cabinet defends themselves from opposition questions, presided over by the Speaker of the House, whose job is to be objective and keep calm, often not from the majority even
\3 Debate is used to gain prevalence by backbenchers and as a check on the power of the majority party
\3 Due to the majority party acting as the government in executive and legislative respects, there must be discipline to preserve legitimacy, especially fearing a vote of no confidence on a major bill, resulting if lost in the resigning of the Cabinet and new elections for all MPs
\4 Most laws aren't voted on by Parliament, but simply debated in the House and ratified by the Cabinet
\4 In recent years, there have been more backbencher rebellions on controversial issues, such as calling for Brown's resignation, leading to the complete changing of the Cabinet and removal of 5 ministers
\4 In addition, recent reveals of personal expenses charged by government officials threatened the idea of Parliament being the ultimate ruling body (parliamentary sovereignty)
\2 Lords is hereditary, with little power since the 1600s, able to ammend legislation, but a majority vote by Commons can remove it
\3 They are also able to debate bills and delay legislation, and until 2009, had 5 law lords who were the highest court of appeals, but without judicial review
\3 Until 1999, half were hereditary peers, half life peers given for service to the UK, but under Blair, most were given to life peers, such that only 92 are hereditary
\4 In 2001, there were plans for a purely life peer Lords house, and in 2007, there was a vote for an elected upper house, but was rejected by the Lords,
\4 In 2008, Straw announced an 80-100\% elected upper house, with one third elected each election, with 12 or 15 year terms, but still with little power
\2 Bureaucracy is under the majority party, based on career service, working to provide expertise to the elected, but non-expert Ministers, such that they have large discretionary power, but rarely enter politics
\1 The Judicial is limited by the idea of parliamentary soverignty, preventing judicial review, such that they can only determine if government decisions violate Parliamentary decision or common law
\2 The legal system is based on common law, or precedent and interpretation, and are divided by appelate and original jurisdiction
\3 Recent EU treaties, binding the UK to international law, have created the conflict between Parliament and EU law, which has given the court the power to determine the outcome, and which may cause future conflict
\2 In 2009, the law lords were replaced as the highest authority by the Supreme Court of a president and 11 justices, similarly limited
\2 Scotland has a seperate judicial system from the UK one, on the other hand, and has its own appelate courts
\2 Judges are generally independent and non-partisan or political, expected to retire at 75, though technically appointed for life with good behavior, generally well-educated
\end{outline*}
\subsection{Public Policy}
\begin{outline*}
\1 While the UK is based on liberalism of freedoms, the collective consensus was based in social democratic values, including nationalization of  major industry and Keynesianism
\2 In the 1980s, Thatcher emphasized neo-liberalism free market revivial, since establishing a middle ground
\2 Under Blair and Brown (as chairman of treasury), lowered the misery index (inflation and unemployment) to a new low, funding welfare without raising taxes
\2 During the Great Recession, public spending on health care and education was cut, and Cameron made the Big Society of private corporations and employee-owned cooperatives, funded by the state but independent, instead of central planning
\3 On the other hand, raising tuition by 300\% to \$14k and privitizing the NHS to respond to baby boomers and long wait-times, creating competition and lowering government oversight of practitioners, has been protested
\1 The Parliament was thought to be transparent until the 2009 scandal, especially about the second home allowance of London and district homes, spending money on improving primary homes, temporarily calling them secondary
\2 This led to calls for reforms on member benefit policy made by members themselves, and more accountability, with calls for oversight/investigation committees, primary elections, and fewer MPs
\3 After Cameron took officie, websites were made to give access to policy information and schedules for accountability
\2 This also greatly hurt trust in government, which the recession already had hurt, with 1/3 stating all politicians always lie
\1 The UK has always been insular, agreeing to join the Common Market several years late, vetoed by France for 20 years more
\2 Thatcher refused adoption of the Euro and under Major, the Maastricht Treaty entered the EU partially, without the Euro and full constitution adoption, with Blair enthusiastic for support
\2 Brown was less strong, but fewer wanted to keep the pound, and Cameron is caught between the two sides on the issue, with as support waned, promising a popular vote on to stay after 2015 if Conservatives have a majority
\1 After the 2005 London transit Muslim terrorist attacks and several other plots in the UK, terrorism has changed from IRA to Muslim, watching mosques
\2 The government also began non-violent Muslim religious classes, but was critisized for not differentiating between extremists, seperating Muslims from society
\2 M15 now spends 75\% of the time on muslim terrorism, especially during the 2011 riots, but the recession and 2012 London Olympics made it more difficult to get funding and manpower, especially after G4S quit for providing Olympic security, such that the military had to help
\2 On the other hand, little effort is going into fixing the socioeconomic gap or integrating Muslims into society
\1 Blair intended to preserve the close US-UK relationship, while working with the EU, acting as a bridge, but supporting Iraq recieved condemnation from the EU, hurting Labour
\2 While the US-UK relationship remained strong, the recession hurt the ability to provide the US actual military support
\1 Recent independence efforts have led to devolution of power to regional bodies, began by Labour until Thatcher slowed it, failing by referendum in 1977
\2 After violence in 2002, the Northern Ireland Assembly was shut down for five years, and was at risk similarly in 2009
\2 Others have called for reforms such as a Bill of Rihts, written constitution, freedom of information, or proportional representation, but traditions cause change to occur difficultly
\2 The 2015 Scottish Independence referendum came extremely close to passing, with many fearing it would take away British culture created by the union, or hurt national chemistry \foot{5}
\3 While there is dangers of economic collapse after losing the pound, many British dislike their far-left, deficit spending, gaining full representation in Parliament along with their own regional government, with general anger toward the British, creating resentment
\3 While it could change British culture theoretically, they are being given the choice alone without any penelty as well
\3 In addition, the Parliament able to adjust taxation by Blair, instead of slowing independence, has caused an upsurge of the movement, beginning in response to end of imperialism and cultural removal of London from the rest of the UK
\3 In addition, recent English pride over British pride has further contributed to the likelihood of independence
\end{outline*}
\section{Chapter 3 - The European Union}
\subsection{Supranationalism}
\begin{outline*}
\1 One of the greatest examples of integration and supranationalism is the United Nations, originally only involving 49 nations in 1945, now global
\2 The Security Council, with the 5 permanent members able to veto, is able to send peacekeeping forces to conflict and request military forces from states
\2 They are generally restricted in terms of weapons usage, and are supposed to remain neutral except in specific circumstances
\2 The UN also contains many other organizations, such as the International Court of Justice and the World Bank, charging dues to pay
\1 The World Trade Organization is another organization to establish trade agreements and world trade rules
\2 Most major nations are members, with Russia joining in 2012, and the terms of membership based on economic development and trade regime, taking approximately five years
\2 Membership is withheld based on political or human rights issues, though long term stability overrules these factors
\1 The World Bank was made in 1944 to aid rebuilding, now providing economically weak countries loans at low interest rates to prevent poverty, corruption, and create investment and infrastructure
\2 They also provide health and environmental initiatives, such as vaccines
\2 Economically advanced countries such as the US control most policy, such that their interests can override weaker countries occasionally
\1 Regional organizations, generally military, such as NATO or the Warsaw Pact, as well as the Organization of American States, Arab League, or the Organization for African Unity has worked for the mutual improvement of nations within
\end{outline*}
\subsection{Creation and Organization}
\begin{outline*}
\1 After WWII in 1949, the Council of Europe formed to promote economic rebuilding among nations after the war, with no power, but promoting cooperation between nations, as well as create a mixed French-German economy to prevent another world war
\2 In 1950, the European Coal and Steel Community formed to coordinate those industries to allow rebuilding
\2 In 1957, the European Economic Community/Common Market was established by the Treaty of Rome, forcing the abolishment of all tariffs between European nations
\2 In 1965, the European Community was made, to promote peaceful use of atomic nations and economic growth, under debate about the power given without losing national soverignty
\1 In 1991, the EU was made by the Maastricht Treaty to aid in the transition to capitalism, with influence in all policy, the common currency of the Euro and European Central Bank
\2 The treaty created the three pillars of authority of trade/economic matters (European Community), justice and home affairs (including asylum, borders, and immigration), and common/joint foreign/security policy
\2 On the other hand, many feared it lost the need for an EU as the Soviet Union fell and security coordination became irrelevant \foot{7}
\1 It has expanded over the years since 1957 from 6 to 28, though it has slowed recently due fears of economic or political instability
\2 Turkey is feared that being Muslim could hurt integration, as well as the history of non-democratic government, low GDP, and geographic split between Asia and Europe
\2 The expansion created fears of organizational issues, integration of weak post-communist nations, and immigrants, causing enlargement fatigue, or disapproval of growth, as well as heterogenity of economies \foot{7}
\3 This has also served to make citizens feel disenfranchized, and create resentment between nations \foot{7}
\2 New nations must have a stable democracy, market economy, and acceptance of EU laws
\2 On the other hand, it is based on the idea of an ever-closer union, moving toward complete integration over time
\3 Conversely, nationalism has still shown to be an issue, especially with economic stagnation as boomers retire and political dissatisfaction in individual countries, creating Euroskeptic sentiment \foot{7}
\3 This creates the possibility of a charismatic leader uniting the EU and finding solutions to reform institutions involved, minor fixes will be used by economic stagnation will continue, or economic collapse will lead to economic collapse, authoritarianism, social unrest, or Russian alliences \foot{7}
\1 The EU is made up of the Commission, Council of Ministers, Court of Justice, and Parliament
\2 The Commission has one represenative from each state, with a large bureaucracy, each with an area of policy as Directorate General, all undera  president, working purely for the EU, not their nation
\2 The Council is made up of national executives, diplomatic, and financial leaders, meeting regularly for different types, required to pass laws, given votes based on population
\3 Every six months, the European Council of heads of state meets
\3 The president of the council is elected every 2.5 years, with two maximum terms, originally rotating every 6 months
\2 The Parliament is directly elected by citizens, with indepedence from national governments, able to propose legislations or veto Council proposals, though it can overrode unanimously
\3 Representatives are not based on population, but rather with smaller states having more votes proportionally
\2 The Court is the supreme court with judicial review, able to limit national soverignty, and can rule between states, officials, companies, and individuals, with each state having a judge, voting by simple majority
\end{outline*}
\subsection{Policy}
\begin{outline*}
\1 The EU is weakest in defense and social policy, but is most active in maintaining a single, internal market by eliminating tariffs and making most professional licenses accepted throughout, except law
\1 It also created European monetary policy by the European Monetary Union and central bank, except the UK and Sweden
\2 Most nations found the Euro more stable than national currency during the recession, but it forced economic coordination of stimulus money, but had issues with agreeing to provide or accept stimulus and how to spend it
\2 On the other hand, more recently, it has been shown that financial collapse in one nation can now spread, or force a nation out, and have hurt Europe's ability to deal with political issues during financial crises, as well as creating resentment \foot{7}
\2 The Euro also created the ability for countries to question the currency, such as threatening to leave the currency \foot{7}
\1 It also created common agricultural policy, spending half the budget on agricultural programs, modernizing farms by farm subsidies, though these have been found to be ineffective
\2 This has led to transitions toward direct payments for development projects recently
\1 Common defense policy has been done less, but in 1999 created crisis management as the main aspect, either humanitarian, rescue, and peacemaking, able to deploy 60k troops in 2 months, sustainable for 1 year
\2 It allowed states to determine their own sending of troops, rather than create an army
\2 Terrorism also became a major issue since 9/11, fearing issues with open borders, discussing and coordinating security with the US
\1 The 1997 Treaty of Amsterdam created judicial initiatives, such as the free movement between states by visas, immigration, and asylum
\2 It also attempted to create cooperation between police forces, but did not require states to participate in these policies
\1 In 2004, the European Constitution was signed to steamline the individual nation treaties, intended to take effect by the end of 2006, but Netherlands and France refusing led to the UK to postpone
\2 Instead, the Lisbon Treaty was signed in 2007 to consolidate the treaties to compensate, including allowing states to leave the EU
\2 It also created the Council president, the Charter of Fundemental Rights (civil, political, economic, and social), and a structure of policies under EU vs national purview
\2 It also gave Parliament additional powers over farm subsidies, border control, asylum, budget and integration, giving power similar to the European Council, but gave national governments more mechanisms to be involved and take action instead of the EU unless necissary
\2 France and Netherlands rejected the treaty due to a dislike of expansion and fear of a democratic deficit, or loss of direct democratic control as the organization grows
\2 While it made the Parliament more powerful, it is still the weakest, and the only directly elected, such that many feared lack of accountability
\2 The Lisbon Treaty worked to fix this, and while rejected at first by Ireland, was ratified
\1 The EU is based on economic liberalism of free markets over economic nationalism of tariffs, shown by the sovereign debt crisis of Greece starting in 2010
\2 Greece originally barely met the economic criteria, and with high soverign/national debt after the 2008 recession, it took a large loan from the EU and IMF, with harsh restrictions
\2 The bailout was thought necissary to protect the Euro, but forced other nations to pay for Greece, angering many, such as Germany, especially when a second bailout was needed in 2011
\2 Many also wanted economic structural adjustment, or restructuring, to forgive the debt, forcing other nations in Europe to pay for it to allow recovery
\2 The crisis also led to the rise of interest rates for European deficit spending, causing massive debt from strong countries like Germany as well
\end{outline*}
\section{Chapter 4 - Russia}
\subsection{Communist and Post-Communist States}
\begin{outline*}
\1 Communist nations are based on the idea that capitalism protects freedom, while ignoring and allowing inequality to be created, found most prominently in the USSR, PRC, DPRK, and Cuba
\1 In 1848, Marx wrote the Communist Manifesto in 1848, stating that the free market involved the exploitation of the poor and creating inequality, believing eventually there would be a revolution of the proletariat against the bourgeoisie
\2 He believed after, there would be no social classes or private property, and without property, government would become unnecessary
\1 In 1917, Lenin created Leninism in a revolution, but believed that a vanguard of the revolution of leaders were necessary to lead the revolution, rather than beginning spontaneously, and overthrew the Czar
\2 The government was democratic centralist, electing leaders from below, discussing after, but not allowing dissent after the fact, with a centralized government and control economy, emphasizing creating industrialization
\3 The legitimacy of the state in theses cases, is based on the ideology, but democratic centralism easily led to authoritarianism
\2 They use force combined with cooptation, or allocation of power through political, social, and economic institutions, recruiting elite by nomenklatura, or filling major positions (political, economic, military, educational, and press) with those approved by the party
\3 The party approval is based on joining the party and active membership, such that social mobility is present, where nomenklatura acted as a specific pathway
\1 In 1949, Mao took over China, based on the idea of Maoism, preserving the peasant based society over industrialization
\2 After his death in 1976, Deng implemented market-based socialism as more of a mixed economy, allowing a gradual move toward a capitalist society
\1 Communism assumes gender roles are due to inequality created by capitalism, such that it will be replaced, but realistically in communist nations, while there is generally more women in the workplace, there is not complete equality
\1 Communism is based on central planning to create equal distribution, though this has led to logistic issues in larger economies and lack of worker incentives for efficiency and innovation
\2 In recent years, the move away from this has led to rapid industrialization in China and Russia, creating the BRIC group along with Brazil and India, becoming new world powers, though Russia has had issues with falling oil even after the recession, adding South Africa in 2010
\1 After the fall of the USSR in 1991 ended the Cold War, Yeltsin took over the remaining largest republic, the Russian Federation, advocating for shock therapy reforms to push to democracy and free market
\2 He was sick and an alcoholic, often becoming authoritarian, controlled by friends and advisers as a corrupt oligarchy, but a constitution was written in 1993 with elections
\2 In 2000, Putin was elected, preserving the oligarchy and centralizing power, stepping down after two terms, but remaining Prime Minister under Medvedev, controlling it, then running again in 2012
\end{outline*}
\subsection{Sovereignty, Authority, and Power}
\begin{outline*}
\1 Political legitimacy was traditionally based on strong, autocratic tsars from the 1300s, through the Romanov family after the 1600s, transitioning previously through assassination and violence, followed quickly by Stalinism, as a form of totalitarianism
\2 Only for several years was Marxism-Leninism present, which had the ideology provide legitimacy for democratic centralism, and still relied on a strong oligarchic governing body
\2 While the Constitution of 1993 attempted to divide power democratically among the president and the Duma, with the lower house by popular election, approved by referendum, attempted coups and conflict prevented legitimacy
\2 Under Putin, while it appeared to show the legitimacy of the Constitution, it was more of a move back to autocracy under himself, with the Duma losing power
\2 Many of the small republic states after the fall of the USSR also began moving back toward authoritarianism since
\1 Political tradition is based on absolute, centralized rule, due to the geography having Huns, Vikings, and Mongols attacking and rebelling at the outskirts, forcing strong rule to keep order
\2 Regular invasions, both outward and in, before the 1600s, and rapid expansion afterward to the current size, led to a tradition of cultural homogeneity, leading to the constant changing of borders over the last hundred years
\2 The conflict between Slavophile leaders, resisting outside influence, and Westernizer leaders, starting from Peter the Great in the late 1600s has characterized
\3 Peter created a new bureaucracy, strong army and navy, infrastructure, and a Window on the West as St. Petersburg, ending isolationism, built on further by Catherine the Great in the 1700s
\3 The removal of old social classes under the Bolsheviks, relied on blending westernization (economic, industrial, and technology growth) and Slavophile customs
\1 Political culture in Russia is based on the geographic location, touching many other nations with different cultures, multiple bodies of water on each side, but almost no warm ports to allow trade, conquering nations that block the sea
\2 There are many resources, such as oil, wood, or natural gas, difficult to develop, which have allowed rapid industrialization, leading to the idea of pragmatic internal development from Stalinism
\2 Russian egalitarianism by the Communists is based on the idea of equality of result over opportunity from Marxism, hurting capitalist ideas
\3 The Marxist ideology of collapse of the Capitalist nations in the West also helped during the USSR to increase Russian nationalism
\3 Communism also created a wide range of economic beliefs, such as a market transition, rapid reform, or moving back to Communist ideology
\3 Economic decline in the 1990s was also blamed on Yeltsin's shock therapy, creating doubt about the idea of rapid market reform
\3 This also created a divide by opinion on the USSR and Communism, with many wanting to move back, partially due to high employment, creating a generational split
\2 Eastern Orthodoxy, working closely with Constantinople originally over Western Europe, worked to separate its ideology from that of European Enlightenment ideology
\3 This created the desire for statism, or a strong state, over a civil society, or freedom from state control, and prevented the separation of church and state, though under the USSR, religion was banned
\2 Ironically, there is skepticism and hostility towards those in power, unexpected when glasnost was implemented, leading to the fall of the USSR, with little popularity of nongovernmental business, finance, and media leaders as well as government, except Putin
\3 Many interest groups also have low particiption, due to a lack of faith in government to produce a democratic society
\2 In addition, nationalism has a strong hold, creating divisions and dislikes for certain groups, such as the Turks or Jews, though they like Baltic peoples
\end{outline*}
\subsection{Political and Economic Change}
\begin{outline*}
\1 Russia was based on resistance to change, creating chaos and revolution when force was not used to created reform, or when contradictory forces took place
\2 Attempts by czars in the 1800s to industrialize prevented and by Alexander II to free serfs led to chaos and assassination, such that it took revolution to stabilize Russian class struggles and Stalin to industrialize Russia properly
\2 Gorbachev's attempts to reform the USSR failed, led to coup d'etats and the collapse of the USSR
\2 The transition from czars to communism to democracy was also shown by revolutionary, violent changes
\1 The tsarist period began with rule by the princes of Moscow, aiding Mongols in exchange for power and land, eventually taking power as the Mongols lost, and  heading the Russian Orthodox Church, far from influence of Western cities
\2 Peter the Great introduced Western culture and technology, having visited Holland, Germany, and the UK, bringing engineers and architects 
\2 Catherine the Great was born in Germany later, and gained warm water access to the Black Sea, and read Enlightenment writings, acting as an enlightened despot, or an absolute ruler with clear, positive goals for the nation
\3 Both also made sure to preserve Slavic culture within the nation, while westernizing it
\2 After Napoleon's invasion in 1812, Western culture came into contact, with many intellectuals feeling Western democracy could not grow under tzars, leading to the Decemberist Revolt of 1825, destroyed by Nicholas I
\3 The defeat in the Crimean War later in the 1800s showed the need to many for reforms, leading to growth of the secret police and executions
\3 Alexander II later created regional zemstvas assemblies and freed the serfs, but many middle class intellectuals didn't believe it was enough, and his son, Alexander III undid much after his assassination
\1 The Revolution of 1917 was due to the losses in the Russo-Japanese War and WWI, due to built-up protests over 12 years, lack of supplies on the front such as guns, and Nicholas II's weakness
\2 Lenin wrote ``What Is To Be Done'' in 1905, arguing the revolution could occur in an agricultural/non-capitalist nation if conditions were bad enough, and creating Leninism
\3 His followers, Bolsheviks, took power in 1917, and a civil war of Russian military backed by the Allies (White Army) and the Bolshevik Red Army took place until 1920
\3 After, Lenin created the New Economic Policy combining centralized government and economy with private ownership, but died in 1924
\2 Stalinism made the party the only party, and more exclusive, admitting 7\%, with the Central Committee of 300 members meeting anually, and the Politburo of 12 men who ran the nation, the general secretary, Stalin, at the top with full power
\3 He removed the NEP and made state-run collective farms to increase efficiency, though many kulaks (wealthy peasents) resisted, and were sent to labor camps
\3 He also implemented large scale industrialization, with a Five Year Plan for production of heavy industry materials, with more specific plans for each factory made by Gosplan, or the State Planning Commission
\3 He also made purges of those deemed disloyal, even within the Politburo and heading the army, executing 1 million members, though denounced after his death
\3 His foreign policy was based in socialism in one nation before international revolution to focus on rapid industrialization, ignoring fascism until attacked, joining the Allies after, but with tension
\1 In 1953, after a power struggle, Khrushchev took power, giving the secret speech about a letter written by Lenin denouncing Stalin and the purges, leading to deStalinization
\2 This included less censorship, economic decentralization, and restructuring of collective farms, as well as a peaceful coexistence with the US, but was critisized for the reforms and lost power after the Cuban Missile Crisis
\1 He was replaced by Brezhnev, who ended the reforms, followed by the younger Gorbachev, who acted more Western, implementing Western reforms to prevent economic collapse
\2 Glasnost, or openness, allowed policial, social, and economic discussion, but led to dissent spreading and revolt, especially in the small republics
\2 Democratization created a Congress of People's Deputies, directly elected, and a President selected by the Congress to add some democracy, but many elected were hostile to his regime
\2 Perestroika made a mixed market economy, allowing privately owned companies, stricter requirements for state-run factories, leasing farm land, and joint ventures with foreign companies
\1 In 1991, Conservatives who wanted to move back from reforms, including the KGB and military led a coup, but were stopped by protests led by Yeltsin, the President
\2 Yeltsin advocated extreme reforms, having been removed from the Politburo, and restored Gorbachev, though in 1991, 11 republics had left, and the USSR ended and the Party breaking down
\1 The Constitution of 1993 made the lower legislative Duma, the Constitutional Court, President, and Prime Minister
\2 Shock therapy to a market economy immediately didn't cause an immediate response, and erratic behavior and firing of staff led to hit resigning before 2000, putting the Prime Minister Putin in power
\end{outline*}
\subsection{Citizens, Society, and the State}
\subsubsection{Cleavages}
\begin{outline*}
\1 The most important cleavage in Russia is nationality, such that while a majority of the nation is ethnic Russian, the country is organized into a federation of autonomous regions and republics based on ethnicity
\2 Most gain trade benefits to be forced to stay, except Chechnya of Muslims, where terrorism for independence has been seen, such as the 2004 school seizing
\2 This has led to fears of other regions desiring independence if Chechnya breaks away, such that a referendum on a regional constitution was done in 2007, approved even though it made them part of Russia
\2 On the other hand, murders, kidnappings, and terrorism has continued since
\2 The invasion of Georgia in 2008, and the Caucasus near Chechyna has led to sucide bombings, such as a president of Ingushetia assassination attempt in 2009, leading to fears of violence during the Sochi Olympics near the Caucasus
\2 Russian nationalists have also done several attacks and kidnappings, such as a 2006 market bombing, to remove immigrants from the nation, leading to the Red Square riot in 2010 after the shooting of a Russian by a Caucasus man
\1 Religion, with the reestablishment and encouragment of the Church by Yeltsin, led to a large amount identifying as Russian Orthodox, though few practicing
\2 In 2007, the Russian Church Abroad rejoined with the Russian Orthodox, having left after the Bolshevik revolution, after Putin convinced them he was a believer and the old regime was gone
\3 On the other hand, due to governmental control over appointments within the Church, many fear it is still restricted too much
\2 Religious, nonreligious, Roman Catholic, Jewish, Muslim, and Protestant groups appear to be existing fairly well, under a combination of flexibility and authoritarianism, without too much conflict
\3 Muslim growth, on the other hand, such that 20M are present, especially in Moscow, the Caucasus, Bashkortostan, and Tatarstan, the final two being fairly peaceful compared to the Caucasus, is very rapid
\3 In 2005, to appease Muslims, Putin worked with President Shaimiev of Tatarstan in the Middle East to make Russia appear pro-Islam
\3 On the other hand, before Sochi, there were several hundred arrests of Muslim extremists found with Islamic literature
\1 Social classes were created in the USSR by Communist Party members and others, with economic and political gains for those involved, though nomenklatura allowed made it slightly egalitarian
\2 After the fall of the USSR, a new rich class formed from opportunities found in the breakup, though several times, first in the 1997 business bust, were arrested
\2 The new wealthy also was favored by Yeltsin and allowed Putin's rise, though he later arrested or exiled many for supposed tax evasion, including Khodorkovsky, the wealthiest man in the nation to send a message to the oligarchs
\1 After Stalin's industrialization 73\% live in cities, creating a large economic, educational, and cultural divide between urban and rural citizens, though the effects of this are unknown
\end{outline*}
\subsubsection{Political Participation}
\begin{outline*}
\1 In the USSR, elections were non-competitive until Gorbachev, with voting mandatory, though after, with only one party, it was purely for party challengers, eventually voting out Gorbachev
\2 Voter turnout since has been fairly low comparatively to Europe, though higher than the US, showing alienation to the Duma, with 50.3\% after the 1993 attempt to take over the country, rising to 64\% by 2008, falling to 60\% after 2012
\2 Presidential election turnout has purely dropped from 75\%  in 1991 to 65\% in 2004, rising in 2008 to 70\%, then falling again in 2012 to 65\%
\1 In the recession of 2008, protests broke out in Russia, especially in the east, Vladivostok, with the Communist Party having a rally and calling for a central planned economy
\2 The rally was allowed with police protection, none of the protests becoming violent
\2 Protests also occurred after Putin ran again in 2012, after the 2011 Parliament elections stating that United Russia rigged them, and during the May 2012 election for Putin
\3 When violence occurred in several cities, the opposition leaders were arrested and the government/Putin did not respond, but protests were not allowed after
\1 On the other hand, there is almost no civil society of private organizations, and only 1\% are in a political party, though a majority do keep informed on politics
\2 Before 1917, low economic development prevented it, while under the USSR, state corporatist systems were created such that the party was the only representation of the people, with state-sponsored organizations
\2 Civil society began to grow during glasnost, though under Putin, there has been restrictions, especially on those of the opposition, with legal, bureaucratic, and police harassment
\3 In addition, since 2012, nonprofits were also restricted to prevent foreign interests and financing, such that assisting foreign organizations is high treason
\2 Putin also created pro-government youth groups, such as Nashi, Youth Guard, and Locals, to create patriotism and prevent opposition, with Nashi creating marches in favor of Putin in 2008 and 2011-12
\3 They also created anti-corruption protests against the opposition in Downtown Moscow in May 2011, and held violent protests against foreign nation embassies over policy and relation issues, such as the moving of an Estonian USSR memorial
\3 It has been often compared to the Komsomol youth Communist Party of the USSR wing, getting grants and support from the government
\end{outline*}
\subsection{Political Institutions}
\subsubsection{Governmental Structures}
\begin{outline*}
\1 The USSR and Russian Federation have a federal government, with 89 regions, 21 of which are non-ethnically Russian, generally called republics due to ruling fairly independently, though not all agreed to be a section
\2 Due to some districts being more powerful, it is called asymmetric federalism
\1 Putin has worked to centralize it, creating seven super-districts of Russia, each with an appointee to supervise the regions within
\2 He also created a law to remove governors who refuse to follow the national constitution, and ended direct elections, rather nominated by the president, confirmed by regional legislatures
\2 The upper house, the Federation Council, was originally governors and regional Duma heads, in 2002 prevented from serving themselves, replaced by those they appoint
\2 He removed the half-plurality, half-proportional rule in the Duma that allowed many minor parties popular in specific regions to gain seats, in favor of pure proportional nationally
\1 Russia has a semi-presidential government, to allow for a strong president, but with a check on executive power, with a seperate head of state (president) and head of government (PM)
\2 Starting in 2012, the presidential term is 6 years, with a 2 term limit, with 1M signature petition as the only requirement
\2 The president has the power to appoint the PM and cabinet, approved by the Duma, though if rejected 3 times, he can dissolve the Duma, almost occuring when they rejected Chernomyrdin twice
\3 The president can also decree by executive order with the force of law, controlling the Cabinet, with the Duma to have no power over the Cabinet
\3 The president can also dissolve the Duma, only done in 1993 when the old Russian Parliament was dissolved, and certain members refused to leave, leading to the army firing on the building
\2 If the president dies, the PM becomes acting president, occuring only when Yeltsin resigned in 1999, putting Putin in power
\3 Prime ministers are generally career bureaucrats, chosen for knowledge and loyalty, rather than party leadership
\1 The Russian legislature is bicameral, with the lower house of the Duma with 450 deputies (representatives) within, able to pass bills, approve the budget, confirm appointments, and theoretically impeach (though it failed when tried on Yeltsin due to difficulty)
\2 The Duma is also controlled by the president's party, such that their legislative power generally is used by the president
\3 Duma laws can also be overruled by more recent presidential decrees, such that it is fairly meaningless
\2 The Federation Council is the upper house, representing the regions, able only to delay legislation and reject legislation, though a 2/3 vote of the Duma can override that
\3 Theoretically, it can also change boundaries of regions, allow armed forces outside the country, and appoint and remove judges, though it has done none
\1 In the USSR, there was no judiciary independent of the Party, while the Constitution made a Constitutional Court of 19 members, appointed by the president, confirmed by the Federation Council, independent of the executive
\2 Putin moved it from St. Petersburg away from Moscow to hurt its influence, and it doesn't question him generally, even with judicial review
\1 The Supreme Court acts as the highest appeals court, independent of the executive, but both courts are under Putin's control, not questioning his arrests of the oligarchs
\2 Most legal experts were trained under the USSR, rather than in a constitutional system, and are unqualified as a result though, but Putin tried to modernize the judicial system in all except Chechnya, with codes of criminal and civil rights
\2 He also created juries, but not the idea of innocent before proven guilty
\1 The domestic security organization, the KGB under the USSR, has been broken up, mainly into the Federal Security Service, largely acting independently of the government
\2 There are also high levels of corruption in the majority of the population, hurting rule of law, needed to get any services, hurting economic development, by diverting it for bribes
\2 No members of the former KGB have been prosecuted for human rights abuses during the USSR, as well
\2 Putin has tried to fight corruption, firing the former defense minister, but it is engrained within society overall
\1 The military contained 4M during the USSR at times, but followed the government, not involved in politics, with huge amounts of money spent, with the majority remaining loyal even during the coup in 1991
\2 The military is currently not involved, with the exception of Lebed in 1996, paid little with little food provided, disgraced after a series of defeats in Afghanistan in 1988, and later in Chechnya in 1996
\2 Under Putin, the army has grown again drastically, nuclear-capable bombers began patrolling regularly again, and soldiers in the invasion of Georgia in 2008 were better trained, though recruitment is low and technology is archaic
\end{outline*}
\subsubsection{Linkage Institutions}
\begin{outline*}
\1 Linkage institutions are fairly weak, hurting democracy, with weak parties originally, followed by United Russia gaining a majority of power, weak interest groups due to the lack of civil society and free media
\1 Parties were created almost immediately after the 1991 Revolution, based around small factional parties centered around single leaders or interest groups
\2 These parties later fluctuated, preventing strong ideological parties and party loyalty, though the proportional representation reform required 7\% of the national vote to gain seats, such that only 4 have seats
\2 United Russia was founded from the Unity Party of oligarchs supporting Putin and the Fatherland All-Russia Party, winning slightly below half, with many minority parties in support, with the platform of pro-Putin
\3 In 2004, Putin was reelected virtually unopposed, and headed the Duma election in 2007 to ensure he would be PM, due to the proportionality rule
\3 In 2012, they lost seats down to slightly above half, and Putin only won with 64\% due to signs of corruption
\2 The Communist Party is second strongest party, losing seats from 1995 onward down to below 100, but rising to 92 during 2012, led by Zyuaganov, who was 2nd in the presidential election in 1996 and 2000, but dropping from 40\% to 29\%
\3 In 2004, he didn't run, and the party was weakened further by a revolt within by Tikhonov, gaining 18\% in 2008, rising a bit in 2012, but barely
\3 The main ideology is the stability of the old regime, unsupportive of Gorbachev reforms, in favor of central planning, nationalism, and recreating the USSR
\2 Liberal Democrats, led by Zhirinovsky, is an extreme nationalist far-right party, attacking Yeltsin's reforms, making anti-semetic and sexist comments, and discussing nuking Japan
\3 They have about 40 seats and 2.7\% in the 2000 presidential election, not running after, but after Putin running in 2012, they gained seats up to 56 seats
\2 A Just Russia was formed from the Life Party, Pensioner Party, and the Motherland People's Patriotic Union, led by the Federation Council speaker, Mironov
\3 Motherland formed in 2003 from the merge of 30 factions, leading to debates within on whether to challenge Putin in 2004, breaking into Fair Russia in 2004
\3 In 2007, they held about 40 seats, gaining up to 64 seats in 2011
\2 Patriots of Russia came in third in the 2011 elections, based on statism and nationalism, supposedly created by the Kremlin to hurt non-United Russia parties
\3 While it did win 13\% of regional legislative seats, they did not gain any Duma seats
\2 Thus, all parties are parties of power, backed by economic or political elites, remaining in power as long as those elites are in power
\1 Elections are either referendums, Duma elections, or Presidential elections
\2 Referendums are used occasionally, such as the Constitution of 1993, Yeltsin's job performance in 1992, in which he barely got approval, and the regional referendum to approve the constitution in Chechnya
\2 In 1993, 1995, then every 4 years after, the Duma was elected, until the 2007 reforms allowing an option to reject all candidates
\3 The stated goal of the reforms was to reduce the number of parties in the Duma, to streamline the legislative process
\2 Presidential elections have a run-off election between the top two candidates if no candidate recieves a majority of the vote, though a run-off has never happened, causing many to question the honesty of the elections
\1 Interest groups during USSR were only the state corporatism/state-sanctioned ones, only allowing Communist Party members to influence policy
\2 After the state industries were made private, they were bought for almost nothing by high ranking party members and those tied to Yeltsin's family, called oligarchs, such as Berezovsky who owned much of the media, getting Yeltsin reelected
\3 These oligarchs later put Putin in power, but Putin resisted, arresting those who opposed Putin, sending many to exile, and arresting the richest man, Khodorvsky, and destroying his Yukos Oil Company to prevent others from reaching
\3 Most left politics after, and Putin forced many to place money in his name as a result, but many were also hurt by the recession, relying on government loans, further losing power over it
\3 On the other hand, Medvedev was Chairman of Gazprom before President and succeeded there by the former PM under Putin, Zubkov
\2 State corporatism thus has large power still, with the Russian government creating major conglomerate companies, using legal tactics to keep them weak enough to rely on the government, such as with Yukos
\3 All there are controlled by the government or those loyal to Putin, called insider privatization
\3 This angers the idea of equality of result that is common among Russian citizens
\2 The Mafia also gained power during the 1991 Revolution, controlling banks, businesses, and natural resources, using laundering, protection money, and deals with government officials and former KGB
\3 They have also murdered prominent members of the Duma, business, and media, threatening to create lawlessness which may result in an authoritarian government
\2 The media under the USSR was the Pravda, functioning as a propaganda tool, now existing far more independently of government censorship, later becoming a tabloid under Putin
\3 While there is apparently press freedom, press such as NTV was taken over by government corporations, and competing stations were shut down, while certain critics of the administration mysteriously died
\3 It took over an hour to report the Beslan 2004 school seizure, though NTV eventually covered it for several hours
\3 The Russian Journalist Union was evicted from their headquarters in 2007 in favor of the state controlled Russia Today TV station, and the Novaya Gazeta, critical of Putin, has had a series of deaths, forcing immediate publishing
\3 Navalny used social media to lead protests during Putin's 2011 run, gaining a large following, but was arrested for embezzlement shortly after
\end{outline*}
\subsection{Public Policy}
\begin{outline*}
\1 The economy is a major issue, with the shock therapy having created the oligarchs are chaos, and leading to the 1997 government default, threatening economic collapse, as unemployment became an issue unlike in the USSR
\2 The ruble decreased by 2002 from more than a dollar to a 30000th of a dollar, and the overall standard of living dropped
\2 In 2008, lowering oil prices stopped several years of economic growth and rising standard of living, and the economic crisis made many question the government
\2 State-run oil and gas are a majority of the economy, such that prices greatly effect it, forcing a government Stabilization Fund in case of price falls, which hasn't fully fixed it
\2 In 2009, Medvedev planned to sell large amounts of state assets, reform taxing and banking, and improving infrastructure and innovation, though under Putin, state-owned companies have monopolized again
\1 Development of civil society is important, creating a private life independent, with trust and respect toward fellow citizens, but citizens don't have the belief in life, liberty, and private property, but rather in statism
\1 Terrorism, such as the Beslan school seige, resulting in 360 deaths, led to Putin's meeting of regional and national governments in 2004, resulting in centralizing government to prevent breaking up of the nation
\2 This led to the indirect selection of regional governors and security measures, slowing attacks, but they restarted in 2009 in the Caucasus
\1 Centralization of power under Putin, with a majority of seats in the Duma, control of natural gas Gazprom, and most media, may be a reaction to terrorism similar to in the West, or a move away from democratic society
\1 Population dropping due to a low birth rate and poor health, predicts an 18\% decline by 2050, due to high alcoholism, especially among males, and poverty and a culture of abortion leading to small families
\2 On the other hand, in 2012, population grew slightly, and Russia is paying ethnic Russians to immigrate
\1 After the USSR's breakup in 1991 forced Russia to accept loans from the US to prevent global economic collapse, and its military power was weakened overall, it had to redefine relations, creating the Confederation of Independent States of 15 former Soviet republica
\2 It contains trade agreements, but almost no political power and is watched closely by the West, while nationalism creates tensions and may cause breakdown
\2 In 2004, Putin financed and advised PM Yanukovich's campaign in Ukraine, leading to protests and a reelection, which he again won, creating tensions
\2 In 2007, a Soviet statue was removed in Tallinn, leading to protests by ethnic Russians, until computers shut down, possibly by Russian attack, and protesters attacked the Estonian embassy in Moscow
\2 In 2008, the invasion of Georgia's South Ossetia, who wanted independence from Georgia, showed tension between the president, Saakashvili, and Putin, as well as Russian militarism, after Georgia allied with the US
\3 The EU eventually brokered a ceasefire, though Russia recognizes that and Abkhazia as independent regions
\2 After the USSR collapsed, Russia was allowed in the G-7 (now G-8), and Russia along with France vetoed the Security Council's support of the Iraq War
\3 Russia also negotiated into the WTO until 2012, attempting to create better trade relations and more foreign investment
\2 Gazprom has led to fights with nearby nations in which pipelines are for gas rights, leading to protests over the rise in prices in 2006, until Gazprom shut down gas to a majority of Europe and Ukraine, forcing pressure on the government
\3 EU nations often act in favor of their own interests with Gazprom, allowing Russia to create tensions in their favor
\3 Russia has also shown support for BRIC fast economic growth nations over the EU, though a large amount of trade is with the EU
\2 After 9/11, US-Russia relations improved, but in 2003, Putin declared the Iraq War to be infringing on Iraqi rights, to which Bush ignored, insulting Russia, and the South Ossetia invasion further increased tensions
\3 America has threatened sanctions many times for human rights abuses, leading to banning Americans from adopting Russians, and the Kremlin blamed the US for protests over Putin's 2012 elections
\3 Ignoring Russia during the Bush and Obama administration in favor of other allies has also insulted the Russians
\end{outline*}
\section{Chapter 5 - China}
\subsection{Sovereignty, Authority, and Power}
\begin{outline*}
\1 Before the 20th century, China had dynastic cycles, or rule of families who eventually lost power when challenged, with legitimacy from the mandate of heaven
\2 The mandate gave the right to rule to the dynasty by the approval of ancestor wisdom in heaven
\2 Power was largely held in a centralized bureaucracy under the emperor during this era
\1 Legitimacy is derived from the mandate of heaven, such that when problems occurred, mandate was lost and the dynasty was challenged, with unrest as a sign of loss of legitimacy
\2 After the 1911 Revolution, Yat-sen became the first democratically elected president, based on his western education, but he was quickly challenged by regional warlords
\2 Mao became prominent at this time, forming Maoism (based on democratic centralism, communism, and the mass line of communication between ordinary people and rulers, emphasizing rural peasants) and establishing the PRC in 1949
\2 Military has also been a source of legitimacy due to the mandate, such as Deng being head of the Central Military Commission, rather than the CPC
\1 Traditions are based on authoritarian power, due to the centralized power controlling a vast empire, but with a constant pull toward decentralization by the warlords
\2 Confucianism is the political philosophy invented in 500s BCE, based on the idea of unequal relationships of families and government with respect between, fulfilling duties
\2 The idea of a meritocratic bureaucracy of scholars, originally based on Confucian knowledge, created a governmental elite
\2 The Middle Kingdom, meaning the center of civilization, created the idea of not needing anything from other nations, all of which without civility and high quality of life
\3 This was challenged, but not removed, by the century of humiliation in the 1800s
\2 Communism and Mao's ideal of moralism of Confucianism without hierarchy, and later Deng Xiaoping Theory based on economic privatization with authoritarian government also have a strong pull
\1 Political culture is based first on geography, with mountains, deserts, and oceans to isolate it, as well as easy access to rivers mainly in the east, condensing population there
\2 There is also a large climate/terrain difference between the north and south regions of the nation, creating a cultural split
\2 Dynastic rule gave a legacy of Confucian values of order, harmony, and hierarchy, with scholarship to establish social structure
\3 There was also a strong culture of ethnocentrism, based on the idea of being the superior culture, creating tensions with groups under their control
\3 This has created a sensitivity to the Western view of China as a 3rd world nation, especially under Deng, who moved away from Mao's isolationism, such as the 2008 Olympics to showcase economic strength
\4 This created anger at pro-Tibetan sentiment in the Western media and protests about treatment of minorities, considered inferior to Han
\3 In addition, the 2008 financial crisis allowed their economy to gain power in relation to the US, sitting front and center of G-20 largest economies in 2009, and many calling G-2 between the US and China more important
\2 Resistance to imperialism created a hatred of foreign capitalist nations, as well as suspicion about their motives within the culture
\3 During Macartnery's arrival in the 1700s, he was rejected believing the China had nothing to offer
\3 During the 1900s, there was conflict over the idea of self-sufficiency versus western modernization, eventually rejecting Western democracy and human rights by Mao and closing borders until Deng took over
\3 There are still fears of Western cultural imperialism, exporting democracy and human rights to China, though some are in favor, creating divide
\2 Maoism based his philosophy on the idea of the strength of collectives of peasants rather than Lenin's vanguard or individualism
\3 He also emphasized struggle and activism to preserve socialism, self-reliance of the peasant class, egalitarianism, and mass line between leaders and peasants, the former giving direction, the latter wisdom and guidance
\2 Deng Xiaoping Theory created the idea of pragmatism, improving the Chinese economy no matter what type of policy, leading to a combination of capitalism free market and socialist planning
\3 He kept Party social and political views, on the other hand, not allowing democratic or individual freedoms to interfere
\2 There was also the importance of informal relationships, both from the dynastic era family alliances and the Long March, with factions of leaders forming and competing for power
\3 This creates a form of patron-clientelism, based on support for the members of the faction, creating connections between different leaders and citizens, as well as making it hard to predict policies of leaders due to supporting factions rather than ideologies
\end{outline*}
\subsection{Political and Economic Change}
\begin{outline*}
\1 China has an ancient history of regional hegemony/control of areas around it, such that dynastic cycles caused change until the 1800s, interrupted by the Mongols in the 1200s, until taken back by the Han
\2 In the 1600s, the Manchus took over, establishing the Qing dynasty, which broke up in the early 1900s due to European influence
\1 In the 1800s, European imperialism interrupted the long-term stability resulting from the dynastic rule, dividing China into spheres of influence, creating resentment and the term ``foreign devils''
\1 The revolutionary upheaval era of 1911-1949 was the result of nationalism, taking power back from the West, establishing a new political community due to the fall of dynasties and imperialism, and socioeconomic development
\2 The new political structure was either going to be democratic, by the Nationalist Party, or Guomindang, led by Kai-shek, or Communist, led by Zedong
\2 Socioeconomic development was a major goal due to the loss of development under imperialistic rule, based on USSR policy during the 1920s, though the Nationalists severed ties in 1928
\2 During this time, the Communist Party was outlawed by Kai-shek, leading to the Long March in 1934-36 of Mao's army by Kai-shek, failing due to the invasion of Japan, and making Mao into a national hero
\1 After WWII, China entered Civil War, leading to Kai-shek fleeing to Taiwan, and the establishment of the PRC based on democratic centralism, while Kai-Shek claimed the government was in exile, forming the state of Taiwan, creating Two Chinas
\2 The PRC was not recognized as the official Chinese state by the UN until 1972
\2 The first phase of the PRC was the Soviet Model from 1949-1957, creating land reform by property redistribution, civil reform by ending drug addiction and establishing womens rights, and five year plans to nationalize industry and collectivize agriculture
\2 The second was the Great Leap Forward until 1966 to end Soviet control, attempting to radically change it into an equal society by rapid development, both industrial and agriculture, and decentralization
\3 It also included political fervor and unity, working as hard as possible, with government run by party workers (cadres), pushing intense productivity, as well as mass mobilization of workers to lower unemployment and increase productivity
\3 It generally failed due to unskilled workers and a history of bureaucratic centralism
\2 The third was until 1976, the Cultural Revolution, as cultural purification of the country and party, based on mass line, collectivism, egalitarianism, service to duty, and struggle of work
\3 It also attempted to remove the history of bureaucracy and all history of China, ending universities and libraries, but emphasizing elementary education
\3 After Mao's death, his followers were divided into Radicals such as the Gang of Four and his wife, Jiang Qing, Military led by Biao until his death in 1971, and Moderates led by Enlai, who wanted economic modernization and contact with the West
\2 After Mao's death, Guofeng arrested the Gang of Four, putting Deng in power from the moderates and military, based on the Four Modernizations invented by Enlai, who had just died
\3 These were industry, agriculture, science, and the military, leading to an open door trade policy to boost the economy, reforming, reinstating, and expanding higher education, and restoring the bureaucratic and legal system
\3 He also merged capitalism with the economy, and decentralized the government, but prevented political liberalization of the nation
\end{outline*}
\subsection{Citizens, Society, and the State}
\begin{outline*}
\1 Due to the loss of Communist ideology as the central aspect of civil life, economic power and the recreation of the ancient Chinese empire are now emphasized by the CPC instead
\1 The primary cleavage is ethnic, with the population mainly Han, but with 55 other recognized minorities making up 8\% of China with a history of resistance to Han rule, most at lightly populated borders
\2 Most are within the five autonomous regions of Guangxi, Inner Mongolia, Ningxia, Tibet, and Xinjiang, taking up 60\% of the territory, but political authority of the regions is limited
\2 Due to existing on the borders, there is a fear of independence movements or allying with neighboring nations
\2 Tibet was conquered early in the PRC, still unrecognized by the old government in exile around the Dalai Lama spiritual/political leader who fled to India in 1959, with Chinese rule contraversial within Tibet, leading to Bejing protests in 2008 during the Olympics
\3 The Dalai Lama moved aside for an elected PM in 2011, but there was a major confication in 2013 of illegal publications in favor of the Dalai Lama
\2 Uyghurs are Muslims Turks in Xinjiang, near Pakistan, the former USSR, and Afghanistan, using terrorism to promote independence, leading to the Urumqi riots in 2009 due to ethnic tension, caused by the death of two Uyghur workers
\2 This has led to language differences, leading to attempts to make Mandarin the official language, censoring Shanghainese in 2006, requiring public sector workers and media to use Manderin
\3 In 2008, Cantonese requirements in schools from 1998 were removed in favor of English combined with Manderin, to allow easier access to foreign schools
\1 The secondary cleavage is urban-rural, due to most economic growth located in cities, with 47\% living outside down from almost in the 80s
\2 This resulted in protests from many rural people who felt ignored by the government, leading to PM Jiabao's 2006 new socialist countryside to aid the rural economy
\1 Before the PRC, citizens were subjects, rather than participants, after which the CCP allowed participation to those committed and willing to devote time, 6\% of the population, with the Youth League to recruit members
\2 During Mao, cadres, based on ideological purity and party loyalty, mainly peasants and factory workers, made up the party, changed by Deng to technocrats
\3 Over time on the other hand, nepotism and family ties have begun to influence the process, though all leaders are highly educated, with fewer party members as peasants, more as intellectuals and professionals
\2 Women are hardly involved, with 22.6\% of the National Congress, 4.9\% of the Central Committee, 2 of the Politburo, and none in the Standing Committee
\2 Under Zemin in 2001, capitalists were allowed to join the party, such that business interests would be represented by the party as well, changing the balance
\1 Civil society has been forced to grow as a result of Western service industries spreading, as well as technology making it difficult to monitor citizens or hide Western culture
\2 Social activist organizations, fighting against governmental projects, such as dam projects or train expansion onto people's property, or lack of transparency have begun to grow
\3 This is due to the growing middle class, as well as the allowing of NGOs to register in the 90s with the government
\2 The growth of Christianity and Buddhism has also shown the growth of civil society, though there is still high control, such as the 2000 crackdown on the Falon Gong religious movement
\1 Protests have also been used against policies, as labor strikes, or by religious groups, though none to the level of Tienanmen Square, due to the censorship and massacre signaling the lack of allowance for that
\2 In Tibet, there were protests in 2008, Tibet held protests along the torch relay, inspiring protests in many other cities for Tibet, as well as human rights records, trade policies, or Taiwan, though Chinese nationals also held large counter-protests
\3 There were also protests that year in Tibet for the 49th anniversary of failed uprisings, calling for the release of 300 monks, leading to riots and tensions between the Dalai Lama and Chinese government over if the former was at fault
\2 In Xinjiang in 2009, riots over the death of several Uyghur workers broke out, forcing Jintao to leave the J-8, using curfews, shut down internet and cell service, and military force
\2 The government also arrested many protesters regularly, executing those found to have used cruel protest tactics
\2 Rural and urban protests are also common, often due to hukou household registration, making it difficult to migrate due to overcrowded cities, though recent programs have attempted to divide workers by skill, allowing some migration
\end{outline*}
\subsection{Political Institutions}
\subsubsection{Linkage Institutions}
\begin{outline*}
\1 The government is based on authoritarian rule by political elites, recruited from the party both due to personal relationships and skill, without input from citizens
\2 Due to the large territory, there has been a move in recent years due to the market economy toward decentralization, with many provinces implementing policies without governmental approval
\2 There is a complete integration of the CPC, military, and government throughout all areas of society
\1 The CPC legitimacy is based on the idea of a vanguard leading for the best interests of the people, organized in a hierarchy, led by the General Secretary
\2 The National Party Congress is a body of 2000 delegates, chosen from lower levels, meeting every 5 years, mainly as a rubber stamp organization, but electing members of the Central Committee
\2 The Central Committee has 350 members, meeting annually for plenums, electing the Politburo, with little explicit power
\2 The Politburo has 25 members, dictating governmental policy, with the 7 member Standing Committee with ultimate power, meeting in secret
\1 There are eight other democratic parties, each based on a specific group of society, controlled by the CPC, acting as advisors to the CPC and gaining roles in government, with prison sentences for those who work for independent parties
\1 Elections are held to give legitimacy to the government, with commissions controlled by the party to remove questionable representatives, with only the local level chosen directly, serving on County People's Congresses
\2 The ability to have citizen nominated representatives, chosen by secret ballot elections, with multiple candidates is a reform from the 80s toward democracy
\1 The political elite of the Old Guard formed from the Long March, with guanxi/personal connections between them, still present in the government
\2 New political elite is recruited through the nomenklature, but also through the guanxi patron-client network, the latter leading to factioning
\2 Before Mao's death, the factions of the radicals under Jiang Qing, military under Lin Biao, and the reformers under Zhou Enlai formed, though after, the leaders were dead, such that Deng Xiaoping was able to unite the factions
\2 From Tiananmen Square, factions began to grow again based in ideology, with Conservatives fearing the loss of Party and centralized power, moving against any independent decentralizing officials and democracy, led by Li Peng
\3 Li's retirement in 2003 from Premier and NPC Chair caused the faction to lose large amounts of their influence
\3 The Liberal faction lost power after Tiananmen Square, but were based on democratic and economic reform, led by Hu Yuobang, whose death started the protests, and the Premier and General Secretary Zhao Ziyang, who lost power for his support of it
\4 While Hu Yuobang mentored Hu Juntao, the former president, Juntao did not give support to democratization or reform during his regime
\2 Factions related to guanxi grew after Deng's death, during Jiang Zemin's presidency, with the Shanghai Gang of his friends from his mayorship of Shanghai leading to capitalism, WTO, and open door trade for personal gain
\3 The Princeling class of aristocratic families of early revolutionaries has also formed, such as Xi Jinping, though they are split over Westernization for personal economic gain and returning to equality of socialism
\3 The Tuanpai, or Chinese Communist Young League, is a faction of young adults led by Hu Jintao, generally working for the urban and rural poor, but with little ideological ties
\3 The first two factions currently make up the Standing Committee, the first moreso, but due to relying on specific people, could lose power at any time
\4 This follows the process of fang-shou, with each faction gaining and losing power, similar to the dynastic cycle, though Deng created a legacy of economic reform, that still exists
\3 Factions often commonly overlap as leadership changes, based on personal connections and objectives
\2 The guanxi combined with the economic growth has led to rampant corruption and bribery, which threatens the legitimacy of the party, with many arrests of corrupt officials under Hu Jintao and later Xi Jinping's 2012 anti-corruption program
\3 In 2007, the head of the Chinese FDA was arrested for corruption after a series of dangerous products were sold abroad, reported by Chinese media as a threat to other officials
\3 Bribery by luxury goods is considered an important revenue for manufactures, denied by the government
\3 In 2012, Bo Xilai, a party leader, was arrested for the cover-up of his wife murdering a business partner, showing the corruption even within the ruling elite
\3 After the arrest of several Chinese executives at a British drug company for bribes to doctors and officials, the government was accused of allowing domestic corruption, only arresting corrupt foreign Chinese national executives
\1 Interest groups are under party control, creating large organizations to prevent the growth of illegal groups, based around occupation or social class, the former most prevalent through the danwei urban social units
\2 Danwei provide medical care, marriage, income, promotion, and other public services, and are used to implement social policies by the government, portraying strong state corporatism
\2 With the growth of private business, danwei power has declined over private lives, and NGOs have grown, though NGOs have little policy power
\2 The 2007 Worker Protection Law against temporary workers and in favor of written contracts gave complete bargaining power to the CPC All-China Federation of Trade Untions
\2 Farmers have no union representation by the state, encouraging petitions and protests due to lack of legitimate means of change
\2 The government prevents multiple special interest groups with overlap from existing, to prevent competition and keep all interest groups under party control
\1 Until Deng took power, all media was government controlled, though major outlets remain, with state-owned Xinhua employing a vast news monopoly,  with 10k workers, even the People's Daily CPC newspaper relying on it
\2 Chinese Central Television (CCTV) is the main television source, controlled by the Party, subject to heavy censorship and regulatory agencies
\2 The move toward economic liberalization has led to more investigative reporting and less censorship as news outlets began to compete, and the spread of internet cafes has hurt the ability to censor
\end{outline*}
\subsubsection{Governmental Institutions}
\begin{outline*}
\1 The relationship between the CPC and the government is controlled by the dual role of government vertical supervision by the next government level, and horizontal supervision from the equal CPC level
\1 The structure of the PRC government is similar to the USSR, created by the Soviets, with three branches, all controlled by the CPC
\2 The legislative exists as the National People's Congress as the national body, with smaller regional bodies, with little legislative power, but in which the Standing Committee decisions are announced, under Party control
\3 The NPC also chooses the President and VP of China, though the party-sanction decision is always chosen
\2 The President and VP head the executive branch, serving 5 year terms, with a maximum of 2 terms, and an age requirement of 45+ years, serving as ceremonial head of state, generally the Secretary General of the Party
\3 The Premier heads the government, appointed by the President, currently Li Keqiang, directing the State Council of ministers, who control the bureaucracy
\3 The State Council, Premier, President, and VP often consist largely of Politburo members, supervised by the Politburo and the Standing Committee
\3 Lower executive levels are held by cadres/bureaucrats, often members of the CPC, but not definitively
\2 The judiciary is made up of the People's Court system in a hierarchy, with prosecutors and defenders provided by the People's Procuratorate organization
\3 Under Mao, there was no rule of law, but it was estbalished after, such that all are bound under the law, though arrests of dissidents on minor charges are prominent and supported by the court
\3 The courts work quickly, choosing to convict 99\% of the time, without a proper appeals process, with thousands executed during government anti-crime campaigns
\1 The People's Liberation Army has held power since the civil wars, with 2.3 million active workers, 12 million reserves, far smaller than the US, but growing rapidly, possibly to control Taiwan
\2 After US arms sales to Taiwan in 2010, top level military exchanges were cut off by the PLA
\2 Many political leaders were military officials, such as Mao, the Old Guard, or Deng, though they were loyal to the Party first, under Party control
\2 The PLA is represented in the government by the Central Military Commission
\2 After Tiananmen Square in which the military used force to capture the square, the public image was hurt, but still holds political power, with Jiang Zemin controlling it even after stepping down as president
\3 It was later passed to Hu Jintao and then Xi Jinping, as a show of transfer of power
\3 Several Politburo members, as well as 20\% of the Central Committee is also still military members, such that it is integrated with the Party/government
\end{outline*}
\subsection{Public Policy}
\begin{outline*}
\1 China has managed to continually defy the idea that a market economy requires eventual democratization, though tensions have been shown in fang-shou
\2 Fang-shou created the cycle of economic reform by the government, political movements by the people, and tighter control by the CPC, the first leading to the second, leading to the third, which may lead back to the first or second
\2 On the other hand, the lack of government transparency has been a constant throughout the PRC, with government talks hidden from the public
\1 Democracy and human rights has created tensions, leading to competitive low level elections, the rule of law, and the NPC having some input on Politburo policies
\2 The most prominent example of the conflict was the Tiananmen square massacre after the death of Hu Yaobang, started by students and intellectuals, growing to hundreds of thousands, calling for democracy
\3 This eventually led to the PLA attacking and arresting protestors, with over 1000 dying, leading to international anger
\3 Publicly, the CPC has not shown any desire to democratize under any of the leaders, censoring Liu Xiaobo's 2010 Nobel Peace Prize for calling for the end of the 1 party system and being jailed
\4 The Party also called for other nations to boycott, leading 15 countries to not attend the ceremony
\3 After the Wukan village in Hong Kong elected anti-government leaders as their chiefs, protests broke out as the Party tried to stop it, eventually forcing the CPC to concede
\2 The rule of law is not acknowledged by communists, but rather thought to be a way to oppress the proletariat, such that legal codes were destroyed in the Cultural Revolution
\3 During Deng's rule, regulations were put in place to allow an international Chinese economy to develop, and criminal codes were made to prevent crimes resulting from the formation of the market economy, leading to the Procuratorate creation
\3 The 1982 Constitution restrained the CPC to the rule of law, but the non-independent judiciary and common death penalty threaten this, though the latter has begun to decline in recent years by thousands
\2 Civil liberties were hoped to grow under Hu Jintao, mentored by Yaobang, and while he spoke in favor of transparency, dissent was cracked down on, arresting writers, regulating the internet, and calling for education based in Party ideology
\1 While Mao encouraged growing population and believed that population management was designed to hurt economies, in 1976, two-child families was advocated due to it taking up a majority of the yearly GDP growth
\2 At the same time, contraceptive, sterilizations, and abortion services were provided to lower the birth rate, and in 1979, the one-child policy was implemented
\3 Economic benefits, as well as penalties and fines to those who didn't follow it, were used, though in 1984, rural areas had it relaxed due to labor needed, reinstated in 2002 due to unreported births
\3 On the other hand, this led to female infanticide and selective abortions, due to males valued traditionally, leading to a national gender gap, creating even lower birth rate in the future, and a surplus of elderly, with fewer young people
\3 As a result, minorities, married couples who have no siblings, and rural peasants have begun to been allowed a second-child, and academics have advocated for it as a universal policy
\2 The move toward urbanization to allow for a domestic consumer economy, rather than an industrial export economy has drastically changed rural from 80\% to 50\%, creating vast new cities, new products, and migration incentives
\3 This has led to a modification of hukou traditions, binding peasants to their land, and creates the need for vast infrastructure and public service spending
\1 Until Deng, democratic centralism and a command economy were used, based on the idea of the iron rice bowl of giving all necessities automatically, and with industrial and agricultural quotas
\2 Deng created a socialist market economy, replacing the people's communes under the Great Leap Forward of thousands of peasants, which were unable to incentivize labor or raise standard of living, with the household responsibility system
\3 The Household Resp. System made individual families in charge of production and marketing, paying taxes and village contract fees, able to keep other production to incentivize labor
\2 In 1988, private business was created under the party, acting as capitalist industries under party control with heavy regulations, but no price controls, growing to replace state businesses
\3 Township and village enterprises run by local government and entrepeneurs slowed migration to cities, acting as capitalist small businesses, drastically raising standard of living
\3 Under Jiang Zemin, private business restrictions were lowered, and many became privatized, gaining foreign investment, and overtaking the TVEs which relied on local government loans
\2 Unemployment has become a major issue after marketization, nonexistant before, and inequality has resulted, creating a floating rural migrant population, moving from city to city for work
\3 It is hoped that as the economy adapts to marketization, unemployment will decrease, but the migrants create crime and overcrowding, and further the urban-rural divide
\3 With high unemployment, it is difficult to create a consumer economy as well, hurting the nation's economic growth
\2 State-sector inefficiency is an issue, with only 1/4 remaining public, relying on state subsidies to stay in business, but necessary to prevent further unemployment
\2 Pollution has also become a major problem, with acid rain falling on South Korea and Japan as a result of coal plants, the leading producer of greenhouse gases, with dangerous effects on the population
\3 While there have been pollution targets, they have failed, though the State Environmental Protection Administration was recently promoted to a ministry, showing possibility of increased focus
\3 In 2009, water, clean energy, and recycling regulations were implemented, and in 2013, cities began monitoring and reporting pollutants
\2 Product safety has been an issue since a series of dangerous and fake products found in 2007, due to the loss of central control over industry, allowing local governments to be bribed
\2 Due to the export-based economy, there was expected to be a drastic effect to the 2008 recession, instead rebounding quickly
\3 This is due to pump-priming, immediately using large stimulus spending, able to be done due to lack of high national debt
\3 On the other hand, this is hindered by the lack of a consumer economy and an innovation (better, rather than cheaper goods) economy
\1 Foreign policy under Mao supported third world revolutionary movements, such as Korea or Vietnam, working with the USSR until after Stalin, before then depending on them for political, technological, and economic stability
\2 In the late 1950s, Mao believes the USSR was no longer truly communist, implementing the Great Leap Forward and Cultural Revolution to move away
\2 During that time, US-Chinese relations remained stagnant, due to Mao's anti-capitalist opinions, until Zhou Enlai arranged a meeting while Mao was sick in 1972
\3 Deng's open door policy further improved relations, and the large scale imports aided the economy, though there has been debate over the US wanting currency devaluation and fewer illegal exports, both unmet
\3 Internet hacking of western companies and governments has also been a subject of debate, with the American company Mandiant, linking it to Unit 61398 of the PLA
\2 As of 2013, China officially focused on becoming a world economic power, rather than internal growth, such as in 2008 warning the US that China is watching the US economy closely due to their holdings
\3 This began to portray them as equals, later suggesting the G-2 summit with the US in 2009 to fix the world economy
\2 Special Economic Zones were made in 1979, giving foreign investors tax incentives, making more zones over time to expand the economy, creating a merchant class
\3 This led to joining the WTO, and having Hong Kong given back to China by the UK in 1997, placed under the one country, two systems agreement
\4 This was signed in 1984, giving autonomy to allow its culture, capitalism, civil liberties, democratic legislature, and legal system
\4 While it is under the Chinese government, most power is given to local government
\4 Hong Kong was hit strongly by the financial crisis, and the Chinese attempt to make Shanghai into a global trading center provides competition, though they still invest in Hong Kong as well
\2 Taiwan's autonomy was protected by the US in the Cold War, though it lost its spot on the Security Council in 1971, and China was recognized in 1979 by the US, such that few countries recognize Taiwan now
\3 China claims authority due to historical rule over Taiwan, while Taiwan is split on reconciling or standing up to China, though there are large trade relations, pushing people in favor of the former
\3 In 2008, commerce, communications, and transportation (or the three links), began to reform in talks with the Kuomintang, with mail and aircraft crossings of the Taiwan Strait becoming common
\3 China in 2013 began measures to integrate them economically, but still responded angrily to the US arming them, or treating them as independent
\end{outline*}
\section{Chapter 7 - Iran}
\subsection{Sovereignty, Authority, and Power}
\begin{outline*}
\1 Iran sovereignty originates from the Achemenian/Persian Empire, the largest in the world and rivals to Greece from 600s to 400s BCE
\2 Persia was based on dry lands north of the Persian Gulf, with centralized military leadership going to the Aegean Sea, opposite of the meritime, decentralized, internally warring Greeks
\2 This was the first fighting between the culture of the Middle East and that of the West, eventyally both taken over by the Macedonians under Alexander the Great, who spread Greek culture to Persia, but Persian government to Greece
\2 Under this rule, military kings controlled Persia, with Darius building Persepolis and creating roads to trade and conquer with nearby nations, having people from across the region pay tribute
\3 His rule was based on state-wide Zoroastrianism, with power still centralized, though not as strictly as under the Persians
\3 Zoroastrianism remained the national religion under the Sassanid Dynasty from 200-600 CE
\1 After the Sassanids, there was a series of invasions and lack of political unity until the 1500s, but an Arab invasion brought Shia Islam to the region, uniting the people even with political chaos, such as the 1200s Mongol invasion
\2 Shia was established as a state religion in the 1500s by Ismail who created the Safavid Empire, based on the idea that the Islamic caliph must be descended from Muhammad, rather than from existing Islam leadership
\3 Shia believed the imam/heir was the true caliph, until the 12th disappeared in the 800s as a child, called the Hidden Imam
\3 Shia was the religious minority, both worldwide, and the only state of it in the region, after Ali, the first imam and Muhammeds son-in-law, was killed by Sunnis
\3 Similarly, the Ottomans in Turkey became the seat of Sunni power, such that there were many wars between, creating modern religious-political borders
\2 In the 1900s, Pahlavi shahs, or authoritarian leaders, attempted to secularize, but were prevented by the Ayatollah Khomeini, who took power in the Revolution of 1979, based on military, charismatic, and religious legitimacy
\3 He created a theocracy, and compared himself to Ismail, protecting the true faith, calling himself the new Imam
\3 He also wrote the Constitution of 1979 and the 40 amendments of 1989, creating the legal legitimacy of the democratic theocracy, stating it was based on the future return of the Hidden Imam and Islamic law
\3 This also included the doctrine of jurist's guardianship, and created a conflict of popular soverignty combined with divine clerical rule, creating opposing parties of reformers and conservatives as well as creating conflict in the Qom clerical seminaries
\1 Political culture is based mainly on historical influences, based on authoritarian, but not totalitarian rule, claiming to be all-powerful, but allowing local leaders and civil society to have power
\2 With the exception of the Qajars and the Pahlavis, there was a tradition of the union of religion and politics, with Shiism and Sharia/Islamic law as central to government
\2 Iran was not colonized, but there were strong European political influences, such that there was motivation to avoid European influence and colonization
\2 Geographic limitations of the desert not allowing agriculture incentivized expansion in the West and population density in the more fertile northwest
\2 There is strong Iranian/Persian nationalism, not speaking Arab after invasion like a majority of the Middle East, keeping Persian culture in modern day, creating strong Iranian nationalism, even over Muslim identity
\end{outline*}
\subsection{Political and Economic Change}
\begin{outline*}
\1 Iran was originally the first large empire, based on strong military centralization around religious rule, losing centralization as secularization occured
\2 It also lost importance on the world stage, while maintaining independence, during imperialism, due to the lack of resources, until oil was discovered
\1 Under the Safavid Empire from 1501-1722, converting 90\% to Shia, though the Kurds in the Northwest, Turks in the Northeast, Baluchis in the Southeast, and Arabs in the Southwest remained Sunni
\2 They respected all People of the Book, monotheistic text-based religions (Jews, Christians, and Zoroastrians)
\2 They ruled from Isfahan, where Persian was spoken with Persian bureaucratic leadership, but without access to naval trade and with the Silk Road disintegrating
\3 This prevented the money for a large bureaucracy or a standing army, using mostly independent local rulers for controlling the empire, with absolute power only in theory
\3 It was also fragmented by mountains, such that clerics were fully outside government control, leading to a gradual loss of power in the 1700s
\1 After the Afghans invaded in 1722, there was political chaos until the Turkish Qajars took power in 1794, remaining until 1925, moving the capital to Tehran, but maintaining Shiism
\2 Due to the lack of Shia Imam leadership, the clerics contained the main religious power, creating government secularization
\2 Land was lost due to Russian growth, and drilling rights were sold to the UK, becoming greatly in debt to the bank of England due to the expensive lifestyle of the rulers as well as modernization and infrastructure development
\3 The Qajars assumed a middle class would form to pay the debt, and there was rampant corruption, such that debt remained
\3 This led to the Constitutuional Revolution of 1905-1909, with merchants protesting customs collections to Europeans, angry at paying off foreign debt instead of domestic
\3 In 1906, due to UK influences, they demanded a Constitution, providing direct elections, separation of power, elected legislatures, popular soverignty, and a bill of rights
\3 This occurred similarly and simultaneously in the Ottoman Empire as the Young Turks took power due to European economic imperialism
\2 This legislature moved toward secularization, creating the Majles legislature, with seats given to all Peoples of the Book, able to appoint cabinet ministers who reported to them, rather than the shah
\3 On the other hand, only Shiites could be in the cabinet, Shiism was the official religion, and the Guardian Council of clerics could veto any laws
\2 Division by Europeans into three pieces, one to the UK, one to Russia during WWI led to infighting, a polarized Majles, and political chaos
\1 Under Colonel Reza Khan, the Cossack Brigade, or the remaining miliary organization, took power, forming the Pahlavis who ruled until 1979, establishing authoritarian rule without the Majles
\2 In 1941, he gave power to Muhammad Reza Shah, his son, but democratic opposition grew, including the communist Tudeh Party, and the National Front led by Mosadeq, based on Iranian nationalism, such as nationalizing oil and removing the army from shah control
\3 In 1951, Mossadeq was elected prime minister, and the Shah fled in 1953, but the US help a coup in favor of the shah to prevent the spread of communist influence
\2 Iran became a rentier state, generating income from foreign purchase of a single commodity, rather than from the people
\3 The shah encouraged import substitution industrialization, or the growth of industry for societal product needs, though most still came foreign
\2 The shahs created a centralized state, with nationally-controlled banks, media, and oil production, with one of the largest militaries, and little local autonomy
\3 The courts and legal systems were secularized, and the White Revolution to counter communism had the government buy land from large owners and sell it to small farmers for low prices, to encourage irrigation and entrepreneurialism
\3 The White Revolution also secularized it more, giving women rights, ending polygamy, and allowing women to work in the public
\2 The shah also seized property of the public, kept a majority of the oil wealth, and created the Pahlavi Foundation as a patronage system for himself and his supporters
\2 In 1975, he created the Resurgence Party as the only party, removing the Islamic calendar, and creating the Religious Corps to modify Iranian Islam
\1 In 1979, the Ayatollah Khomeini led one of the only fundementally religious revolutions of the century, creating a theocracy, due to the shah secularizing too fast, and was thought to be a totalitarian, creating a patron-clienelistic state without civil society allowed
\2 In addition, the Shah had ties to the West, angering both nationalists and the clergy, such that the revolution was in
\end{outline*}
\section{Chapter 8 - Nigeria}
\subsection{Sovereignty, Authority, and Power}
\begin{outline*}
\1 The national question is the debate about the method of governing the nation, if it should remain as one nation, under which the nation is fully divided
\2 The constitution was first written in 1914, rewritten 8 times since, amended often since the most recent version in 1999, such that there is no constitutionalism in the political culture of the nation
\2 This results in leaders not following and rewriting in favor of themselves
\1 Due to the lack of constitutionality, there is little legal-rational legitimacy, and strong cleavages creating fragmentation, such that the military lends stability and provides legitimacy, such that most recent presidents are military
\2 While there was rule of law under British rule, in the 1960s when it became independent, military rule took over without rule of law
\3 Corruption was especially prominent under General Babangida in the late 80s, and General Abacha until 1998, stealing from taxation
\2 This issues resulted in little trust or belief in government
\3 This was not aided by the corrupt election of Yar'Adua in 2007, though it has been eased a bit by the election of Goodluck Jonathan in 2010
\2 Since military rule ended after 1998, sharia law became prominent in the public sphere, with Hisbah enforcing Islamic law and stoning violators in Islamic courts
\3 Since 2008, there has been a government crackdown on Islamic militia, with the secular police no longer enforcing the most strict Islamic laws of the state, but rather a more limited form
\3 Sharia law is still used to encourage female literacy by Islamic-secular combined education, or recycling, but not in the severe forms
\1 During the pre-colonial era from 800-1860 AD, centralized states developed in Nigeria, especially in the savanna North, rather than the forest South, resulting in government being needed to irrigate in the North
\2 Trade connections, provided by the ocean and Niger River, as well as through the Sahara desert, were established early on
\2 This served to create Islam influences, replacing native religions, especially in the elite due to emphasizing their political authority over common citizens, making women socially and politically inferior
\2 Kinship politics due to only having local governments, especially in the South, were common, though the division in government size between the North and South was not definite, but often broken
\2 There is also a tradition of democratic accountability, as representatives of the people, from the Yoruba and Igbo
\1 During the colonial era until 1960, the British created the idea of rule of law, but also the idea of authoritarian rule, strengthening the rule of chiefs loyal to the British, making them accountable only to the British, not the people
\2 It also created the ideas of individualism and capitalism, especially among the elite, ruling for themselves
\2 They created the idea of the state/chiefs intervening for the good of the economy, but did not give ideas of individual rights or free market, increasin ginterventionist authoritarianism
\2 Christianity was also brought to the South and the West, creating a divide, and ethnic politics increased between the Hausa-Fulani, Igbo, and Yoruba, due to the British giving special benefits, such as education or bureaucratic jobs to some
\3 There was also ethnic appeals to create the independence movements before 1960, further increasing tensions
\1 Since 1960, they attempted to create a parliamentary system until 1979, but ethnic cleavage prevented a majority, creating a presidential system instead, in theory with an independent judiciary and legislature
\2 Ethnic conflict also grew originally, due to the large size of the Hausa-Fulani leading to large power in the parliament, joining with the Igbo, pushing out the Yoruba, leading to conflict and Igbo military takeover in 1966
\2 Military rule was used, claiming to stop corruption and violence, but led to a new coup against Ironsi in 1967, and eventually the Igbo declaring Biafra independent, leading to a civil war until 1970
\3 Corruption has also been prominent from colonial rule
\3 More recently, Buhari, the dictator from 1983-85, has been elected the new democratically elected president, creating uncertainty
\2 Federalism was implemented to prevent ethnic conflicts, but failed under military rule, due to lack of local soverignty, with all power in Abuja
\2 The economy is also largely dependent on oil, hurting stability, due to leading the elite to attempt to take political power for control of the oil reserves, and preventing diversification of the economy
\1 Political culture has been built based on ethnic diversity, military leadership, and corruption throughout history
\2 It is also based on religious conflict, such that when Christian missionaries arrived and quickly spread it, it led to conflict with the Muslims, over the role of Sharia law
\2 Patron-clientism, called prebendalism, is based on political leaders providing favors to other elites in favor of support, generally based on ethnicity or religious divides
\3 There is also largescale corruption of the Nigerian National Petrolium Corporation, staterun by the government
\2 There has always been an authoritarian state, but at the same time, an inability to control the large civil society of social, economic, and political organizations, such as religions, NGOs, or labor groups
\2 There is a difficulty becoming modern due to tensions toward traditional divides, such as ethnic conflict or communal leadership over the state control necissary to modernize
\2 Geographic divides also play a role, located at West Africa, far larger than neihboring countries due to higher employment and resources
\3 The Northwest is controlled mainly by the large Muslim Hausa-Fulani, the Northeast by many small Muslim groups, like the Kanuri
\3 The Middle Belt is a mix of small groups of both Muslims and Christians, while the Southwest is controlled by the Yoruba, of a mix of Abrahamic and local religions
\3 The Southeast is controlled by the Roman Catholic and Protestant Igbo, while the Southern Zone at the Niger delta is ethnically diverse
\end{outline*}
\subsection{Political and Economic Change}
\begin{outline*}
\1 In the pre-colonial era, geographic proximity to Berber traders in the Sahara led to the diffusion of Islam gradually, until the Fulani took over the North by jihad, making the Sokoto Caliphate in the entire North
\2 It was based on a centralized government, trading with the Europeans, taken over by the British in the early 1900s
\2 In the South, there was more Atlantic trade with Europeans, spreading Christianity, but also creating slave trade, starting with the Portugese, growing rapidly in the 1600s, losing many young men
\1 In the colonial era, natives of specific ethnic groups were trained to be bureaucrats in exchange for favor, and the economy was turned into a source of cheap labor and resources, with the strongest influence in the Southern ports
\2 The Northern government was left as it was, emphasizing the power of the elites, but keeping them seperate in exchange for support, creating regional and ethnic divide
\2 On the other hand, western education was implemented in the 1930s and later, creating literacy, but mainly for the Southern ethnic elite, forming a larger class, ethnic, and geographic divide
\3 This education was especially given to bureaucrats, creating western political ideologies of democracy and freedom, leading to decolonization
\1 In the modern era, there was a series of coup d'etats by the military, leading to the dictator in 1979, Obasanjo, stepping down for democratically elected Shagari, who was quickly removed by a coup in 1983
\2 In 1999, Obasanjo was elected as a civilian after a series of further coups, thouh there was large scale corruption in the 1999, 2003, and 2007 elections, creating the possbility of breaking up as a nation
\2 Ethnic conflict has grown, with military generals dividing by ethnicity, competing for control of the nation, and with institutional corruption and stealing of public funds, especially by Abacha from 1993 to 1998 and Babangida from 1985 to 1993
\2 Military dictors claim to take power until the country is stable enough to transfer it, apparently in 1993 with Abiola, until Babangida refused
\3 In 1998, Abubakar took power after Abacha's death, promising to hand it over to the election winner, working for the transition, giving power to Obasanjo
\end{outline*}
\subsection{Citizens, Society, and the State}
\begin{outline*}
\1 Democratization is difficult, due to rampant poverty, almost 60\% of the population, low literacy rates compared to the global rates in the mid-80s, with males at 72\%, females only at 50\%
\2 There are also health issues, such as the HIV epidemic, costing the economy and population by large amounts, without much government support, mainly relying on NGOs to give medication and aid
\2 There is also a huge income inequality gap, without much chance of economic growth to allow it to close
\1 There are huge cleavages in Nigeria, hurting the legitimacy and productivity of the government, with over 300 ethnic groups, dozens of major languages, little interaction between, and unable to speak other major groups languages
\2 There are also large religious divides, split between Muslim and Christian, especially based on the nicer treatment of the British to the Christians, also manifesting in the regional divide
\2 Urban-rural divide sis found, with most political organizations and protests taking place within cities, as well as a social class divide, created by the corrupt political elite, though some do want to stop the corruption
\2 Most governmental leaders gained power by appealing to some divide, generally ethnic, to gain a support
\1 While there is an active civil society, most citizens do not think of themselves as participants in the government, preventing democratic institutions from developing
\2 There is large-scale prebendalism, or clientalism based on personalized rule as fiefdoms, performing favors for constituents in exchange for support, appointed to gain loyalty
\3 Most favors are through the political elite, and persist through personal connections, though this is present in all aspects of society
\2 Civil society is based on many interest groups, acting both to unite and break apart Nigeria, such as the Movement for the Survival of the Ogoni People, founded by Saro-Wiwa in the 90s, to protect the Ogoni financially and protect the environment
\3 This has served to both unite Nigerians around environmentalism, but also create ethnic boundries more prominently
\3 Trade unions have also become prominent to protect workers rights, while professional organizations protect the growing middle/professional class
\2 There are low levels of trust in the government, due to fluid political parties created mainly around a single charismatic candidate, cancelled elections, fraud, making it difficult to track voter behavior
\3 On the other hand, it does appear that voting in recent presidential elections has risen to 55-65\% approximately, showing progress, and 38\% rated the election in 2011 as free and fair, which is improvement
\3 Originally, there were statistically less ethnic divisions and lack of faith in democracy, such that military rule of Babangida and Abacha created these issues in the 80s and 90s
\4 The happiness with democracy is also decreasing, though a majority still believe it is the ideal form of gobvernment
\3 The vast majority believe the economy is bad, the government does not aid it, and would not go to the police due to the need for bribes
\3 In addition, Transparency International rated it as one of the most corrupt governments in the world, due to extreme patron-clientelism in all levels of government
\4 Under Yar'Adua in 2008, Ribadu, the head of the anti-corruption commission who had charged and prosecuted many politicians, including 7 governors, one of whom, Ibori of the Delta States, was a funder and party leader of the People's Democratic which Yar'Adua was a part of, was removed as a result
\2 Since democratic leaders took over in the 90s, protests, especially by ethnic or religious groups, or against oil companies, have began to take place
\3 Under Obasanjo, many protests against oil companies became violent, with rebels kidnapping foreign workers, destroying pipelines, hurting the international energy market
\4 This forced Obasanjo to suppress and crack down on protest to keep oil companies in Nigeria for the economy
\3 Under Jonathan, the North protested two successive Southerners, breaking an informal rule, leading to Boko Haram committing terrorist attacks to hurt his authority
\4 It is specifically working for sharia law in Nigeria, leading to military retaliation and illegal arrests and assassinations of Boko Haram, critisized internationally
\4 In 2013, they had killed more than 1000, leading to state of mergency, leading to a military invasion of northern Baga, leading many to flee the region
\subsection{Political Institutions}
\subsubsection{Linkage Institutions}
\begin{outline*}

\end{outline*}
\subsubsection{Government Institutions}
\begin{outline*}

\end{outline*}
\subsection{Public Policy}
\begin{outline*}

\end{outline*}
\section{Articles}
\begin{enumerate}
\item \hypertarget{1}{``The Clash of Civilizations?'' by Samuel Huntington}
\item \hypertarget{2}{``BBC News, Just how important is the Magna Carta 800 years on?'' by Nick Higham}
\item \hypertarget{3}{``The Guardian, Magna Carta - 800 years on'' by David Carpenter}
\item \hypertarget{4}{``How the Murdoch Gang Got Away'' by Geoffrey Wheatcroft}
\item \hypertarget{5}{``The Scots and the English: Some Guilty Thoughts'' by Andrew Sullivan}
\item \hypertarget{6}{``The Breaking Point'' by Cambridge and Cumbernauld}
\item \hypertarget{7}{``Does Europe Have a Future?'' by Stephen Walt}
\end{enumerate}

\en:qd{document}
