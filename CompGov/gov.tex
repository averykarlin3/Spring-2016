\documentclass[11 pt, twoside]{article}
\usepackage{textcomp}
\usepackage[margin=1in]{geometry}
\usepackage[utf8]{inputenc}
\usepackage{color}
\usepackage{indentfirst} %Comment out for no first paragraph indent
\usepackage[parfill]{parskip}
\usepackage{setspace}
\usepackage{tikz}
\usepackage{amsmath}
\usepackage{amsfonts}
\usepackage{amssymb}
\usepackage{enumitem}
\usepackage{outlines}

\usepackage{fancyhdr}
\pagestyle{fancy}
\cfoot{\hyperlink{content}{\thepage}}
\lhead{}
\chead{}
\rfoot{}
\lfoot{}
\rhead{}
\renewcommand{\headrulewidth}{0pt}
\renewcommand{\footrulewidth}{0pt}

\newcommand{\footer}[1]{\hyperlink{#1}{$_#1$}}

\usepackage{hyperref}
\hypersetup {
	colorlinks,
	citecolor=black,
	filecolor=black,
	linkcolor=black,
	urlcolor=black
}
\newcommand{\sepitem}{0pt} %Added room between items on the list, not including a list and its sublist
\newcommand{\seppar}{0pt} %Between items and lists overall

\setenumerate[1]{itemsep=\sepitem, parsep=\seppar}
\setenumerate[2]{itemsep=\sepitem, parsep=\seppar}
\setenumerate[3]{itemsep=\sepitem, parsep=\seppar}
\setenumerate[4]{itemsep=\sepitem, parsep=\seppar}

\newenvironment{outline*}
{
	\begin{outline}[enumerate]
	}
	{\end{outline}
}

\begin{document}

\title{AP Comparative Government}
\author{Avery Karlin}
\date{Spring 2016}
\newcommand{\textbook}{Ethel Wood's AP Comparative Government}
\newcommand{\teacher}{Trainor}

\maketitle
\newpage
\hypertarget{content}{\tableofcontents}
\vspace{11pt}
\noindent
\underline{Primary Textbook}: \textbook\\
\underline{Teacher}: \teacher
\newpage

\section{Chapter 1 - Introduction to Comparative Government}

\subsection{Comparative Method}
\begin{outline*}
\1 Government is the leadership and institutions which make national policy decisions, while comparative government is the study of the flow of power from different people and groups within a government
\2 Politics are the activities associated with the governance of a country or area, especially the debate or conflict among individuals or parties with or hoping to gain power
\1 Political science, as a social science, can either be done based on empirical data, or facts and statistics, or normative issues, which are based on value judgements
\2 Since it is a science, it requires a hypothesis of a relationship with variables, to find a causation from the independent to dependent variable, such that one causes the other
\2 Correlations are when the change in both variables are simultaneous, implying the possibility of a causation, but not acting as evidence
\1 The main comparative model is the three-world approach from the Cold War, dividing into the first world of the US and its allies, the second of the USSR and its allies, and the third world of economically underdeveloped, unaffiliated nations
\2 It is used in modern day, based on communist, post-communist, and capitalist nations, advanced, economically developed democracies, and developing nations
\2 Developing and communist nations are more likely to become authoritarian nations, rather than economically developed capitalist nations, which are likely to be democratic
\2 This model integrates political and economic systems, due to the economy having a strong factor in citizen interaction with the government, allows observation of the impact of political change since the fall of the USSR, and the impact of informal politics, or the interaction of citizens and the civil society, or the organization and defining of citizen activism, with politics
\1 The Huntington model\hyperlink{1}{$_1$} states the next great conflict will be between different civilizations (the broadest level of identity/cultural similarites), having moved from monarch territories to nations (from the 1790s to 1910s) to ideologies
\2 After the Cold War between Western civilizations, non-Western groups began to move away from being puppets of the West, responding to Western ideas of appropriate policies in the UN and IMF, to going back to traditional ideas, and due to increased interaction between civilizations
\3 Educated foreigners are now being educated in their own culture, while Western culture spreads through globalization, reversing past trends
\3 Finally, trade has begun moving backward toward mainly regional in recent years
\3 Most particularly the Western idea of a universal civilization contrasts the Eastern idea of particularism and differences between civilizations, forcing civilization to bandwagon with the west, isolate themselves, or modernize to create a balance of power (though only Japan has done it without moving towards Western)
\2 The civilizations include Western, Confucian, Japanese, Slavic-Orthodox, Hindu, Islamic, Latin American, and African civilizations, due different views on relationships, rights, hierarchy, and religion, such that intellectual debate is not possible
\2 Economic progress and social change has also weakened the power of nations, resulting in religious divisions taking over, leading to fundementalism
\2 This is also the movement back to the original, unchanging institutions, rather than ideological divisions, spreading both influence between civilization and local territory along fault lines
\2 This has manifested most commonly in economic rivalries such as US-Japan or US-China and civilization support for minor conflicts, but also in ethnic clensing, Islamic fundementalism, movement from democratic institutions, and military conflict
\3 In addition, the double standard of exempting similar nations from human rights regulations, while condemning others, leads to conflict
\3 This leads to Western attempts to ban the production of non-Western weaponry, in an attempt to not hurt their interests, while other nations define it as equal protection, leading to middle eastern and eastern Weapons States
\2 Torn countries between multiple civilizations must have the poltical and economic elite, the general public, and the majority of the new civilization agree to be able to take a new identity, found within Mexico
\3 On the other hand, in the case of Russia, none are present with the Post-Cold War attempts to join the West
\end{outline*}

\subsection{Sovereignty, Authority, and Power}
\subsubsection{Nation-States}
\begin{outline*}
\1 States are organizations that define the use of violence within a specific territory, through military and weapon restrictions, with institutions to create policy and promote general welfare
\2 States thus have soverignty, or the ability to create their own policies without influence
\2 States without soverignty are subject to corruption, used by internal and external organizations for their own ends, often in undeveloped nations
\1 Nations are groups of people with a common political identity, such that nationalism is the send of belonging, often resulting in patriotism for the nation
\1 Nation-states are the main form of a state, such that borders are drawn around a specific nation, providing the identity for those in the main nationality
\2 Bi/multi-national states are those containing multiple nations, such as the USSR, such that minority groups began protests for independence, decaying into nationstates, though the same issue has applied to the multinational Russia since
\2 Stateless nations are those without a state, such as the Kurds, often causing fierce nationalism
\2 Nations generally expanded from core areas, until they reached another nation-state, creating boundaries, with periphery areas around the core areas, with more open land and fewer towns
\2 Multicore states often have inner-conflict as the result of multiple groups having competing interests, and can hurt stability, such as in Nigeria, though often not, like the US
\end{outline*}

\subsubsection{Governmental Regimes}
\begin{outline*}
\1 Regimes are the sets of rules that states set and follow, generally divided into authoritarian and democratic regimes
\1 Democratic regimes can either be indirect, electing representatives for the people, or direct, with individuals directly having a say in government, generally only direct with small populations
\2 Parliamentary systems are those where the legislature is elected, and those officials determine the executive, while presidential systems have both elected, with seperation of powers between
\2 While there are different levels of economic regulation, democracies have independent corporations from the government
\2 Most democracies are divided into a legislative, executive, and judicial branches
\2 Semi-presidential systems can also exist, such as in the 1993 Russian constitution, with both a parliamentary prime minister and a president sharing power, though the president has taken far more under Putin
\2 Democratic regimes rely on pluralism, or power split among many interest groups attempting to influence
\1 Authoritarian regimes have power held by the political elites without citizen input, either by a dictator, hereditary monarch, aristocrats, or single political party, controlling both the government and economy
\2 In these societies, there is no limits on the power of the leaders, responsibility to the public, or restriction of civil rights
\2 Communist countries are controlled by the party, controlling all aspects of life, following Marx or Zedong economic philosophy
\2 Corporatism is the supervision of government policies by some labor or business group, though it may be some other patron-client system
\3 Corporatism often results from authoritarian regimes trying to provide the appearence of citizen involvement to gain co-optation, or citizen support, while banning other groups 
\3 Patron-clientelism is the system of benefits provided to a specific group in exchange for vocal support
\3 It can also result from economic regulation or nationalization of industries resulting in close ties between government and industry
\3 Democratic corporatism can be shown by recognition of specific groups by the state, while forcing others to require recognition, legally bound to the state, working on behalf of the state
\2 Totalitarian regimes are a subset of authoritarian, whcih attempt to control all aspects of political and economic systems, often based on a strong ideological goal, such as communism
\3 Totalitarian governments especially use violence to remove opposition, and are more illegitimate, in that they are not accepted by the people, which authoritarian governments may be
\2 Military rule is a common form of authoritarian, often taking power in a forced takeover/coup d'etat during unrest, generally with public support, restricting civil liberties to preserve order, joining with the bureaucracy
\3 This can lead to a weak state, forcing other coup d'etats, in a series of weak regimes
\1 The Democratic Index was published by the Economist since 2007, ranking countries based on the electoral process and pluralism, civil liberties, government functioning, political participation, and political culture
\2 It also catagorizes into democracies (like the UK), flawed democracies, authoritarian (like Nigeria, Russia, China, and Iran), and hybrid regimes (like Mexico)
\end{outline*}

\subsubsection{Legitimacy}
\begin{outline*}
\1 Legitimacy, or the right to rule, is determined by the citizens, and is catagorized as either traditional, charismatic, or rational-legal
\2 Legitimacy is easier to maintain in economic prosperity and with high government performance approval
\1 Transitional is based on tradition, such as hereditary rulers, often based on myths, legends, religion, or divine right, with ceremonies, symbols, and artifacts to encourage the idea of legitimacy
\1 Charismatic is often based on personality or military talent, such as Napoleon, though when he lost militarily, it faded, generally a shortlived form of legitimacy, unable to be passed on after death
\1 Rational-legal is based on institutional laws and procedures, preserved through belief in the rule of law and acceptance of the authority of the state, such that shared political culture is important
\2 It can be based on common law, or legal tradition and precedents
\2 Legitimacy of rational-legal in democratic governments can be the result of the loss of the legitimacy of the electoral system
\1 In modern states, the main form of legitimacy is from rational-legal, though traditional and charismatic allow easier gain of power, or influencing politics easier within interest groups
\2 Many states also preserve some form of traditional legitimacy, to add legitimacy to the legal-rational democratic form of government
\end{outline*}

\subsubsection{Political Culture and Ideologies}
\begin{outline*}
\1 Political culture is the collection of political beliefs, values, practices, and institutions which a country is based on, such that for a government to remain, it must be based on that culture
\2 Social capital, or reciprocity and trust between citizens and the state or other citizens of all levels, can be used to measure how democratic it is, such that more democratic makes it greater
\3 On the other hand, social capital theory, which predicts difficulty in Islam or Confucian regions, has been critiqued for ignoring countries such as Turkey or India
\2 Consensual political culture is agreement on what issues should be solved and the process by which decisions are made, such that legitimacy of the government is accepted
\2 Conflictual political culture by fundemental economic, religious, or polticial differneces often leads to conflict, and prevents effective rule
\1 Political ideologies are sets of political values of the basic goal of government, held by the individual
\2 Liberalism values political and economic freedom, maximizing rights and freedoms, and allowing citizens to disagree with the state and attempt to influence decisions
\2 Communism values equality, believing freedom won't create general prosperity, believing eventually a wealthy class will form and take control of the government, advocating state control of all resources to protect economic equality
\2 Socialism values a combination of freedom and equality, believing in the free market and private owenership, but believes the state have heavy control of the economy to provide benefits and preserve equality
\2 Fascism values strength, believing some groups are inherantly inferior, attempting to create the strongest possible state, such that rights must be taken away by the authoritarian state to preserve it
\2 Religious ideologies also play a large role in many nations, often having an official state religion, or having special interest groups influencing it
\end{outline*}
\subsection{Political and Economic Change}
\begin{outline*}
\1 Change generally happens both politically and economically simultaneously, and most countries experience it over time, but when happening seperately, creates tensions
\1 Change can occur through reform, attempting to use standard political and economic institutions to create change
\2 Revolution attempts to change the political and economic institutions through the overthrow or revision of the institutions, generally impacted economic, political, and social systems, regardless of the intent
\2 Coup d'etats replace the government with new leaders by force, often carried out by the military, but can cause instability
\1 The strongest attitude toward change is radicalism, or the belief in rapid, dramatic changes, often believing the institutions cannot be fixed, leading to revolutions
\2 Liberalism as an attitude is the belief in reform and gradual change, beliving in repairing and improving existing systems, with the goal often of leading to a complete transformation over time
\2 Conservatism believes change is disruptive and causes unexpected negative outcomes, believing in the need to preserve legitimacy of government, basic societal values, and law and order
\2 Reactionaries believe that the current state has already move too far from basic societal values, wanting to use revolutionary means to return to old institutions
\1 The first major trend of modern change is democratization, based on the idea of competitive elections, with many countries moving further to liberal/substantive democracies, instead of illiberal/procedural
\2 Liberal democracies have belief in neutrality of the judiciary, checks and balances on power, civil liberties, rule of law, civilian control of the military, and open civil society
\3 Illiberal democracies often have an unchecked executive and restricted citizen groups, preventing truly free elections, but are necissary before a society can become a liberal democracy
\2 Huntington believes there are three waves of democratization, the first gradual until WWII, the second after WWII involving de-colonization, and the third involving the defeat of totalitarians after the Cold War
\2 Democratization is due to the legitimacy of authoritarians, expansion of urban middle class, human rights emphasis, and international snowball effect
\2 Democratization happens after a trigger event taking place, after a revolution of rising expectations of high living standards, causing democratic consolidation of the elites and public willing to share power, spreading throughout society called political liberalization
\1 The second trend is economic liberalism, moving to market economies, such that it is under debate, due to influence such as China, if democracy and market economies inherantly move together
\2 19th century European reformists were generally middle-class bourgeoisie, who wanted their views represented in government, and economic goals unrestricted to allow economic mobility
\2 Radicals, on the other hand, believed that freedom clashed with equality, and thus a free market was not the ideal, including Marx, beliving instead of a command economy of government owned businesses
\3 These economies, in the USSR and China, had a state planning committee, with economic production blueprints and quotas in 5 year plans
\3 On the other hand, these generally, while creating economic growth, did not lead to higher living standards
\2 In recent years, most command economies have moved toward market ecnomies with less government regulation, with the current debate between a mixed (with significant regulation and control from government) or market economy
\2 Economic liberalization is based on the failure of many command economies and the belief that government is too large
\3 Thus, many command economies have had marketization toward market economies and privatization toward private ownership
\3 The main downside, the business cycle, has led most to adopt a mixed economy to lower the dangers of the business cycle
\1 The third trend is fragmentation, or divisions based on culture/ethnicity, moving away from prior globalization toward nationalism, especially found in the politicization of religion and increase in fundementalism
\2 It is argued that those who believe in the clash of civilizations underestimate this factor of cultural differences
\end{outline*}
\subsection{Citizens, Society, and the State}
\begin{outline*}
\1 Social cleavages are divisions in society that are outside politics, but impact political policymaking based on causing deep political identification, including social, ethnic, religious, and regional cleavages
\2 Coinciding cleavages are those which divides the same groups against each other on issues, while cross-cutting cleavages are those which divide groups that agree on some issues, but disagree with others
\2 Regional conflicts are often the results of different levels of economic development, or religious divisions between regions
\2 The depth of the cleavages in the social structure of a society, the level of political party alignment with cleavages, and specific cleavages not involved in the political process determine the importance
\1 Government-citizen relationships are based first on political efficacy, or the citizen ability to understand and influence political events, determining if they feel the government cares about their opinion
\2 This works to create active voting behavior, and political participation, rather than interacting purely through subject activities (obeying laws), creating political attitudes
\2 Attitudes in other respects, such as trust or the ability of the government to impact their lives also play a role
\2 Government transparency also changes interactions, preventing corruption, as well as the methods of learning about political actions to create immediate views
\1 Social movements are organized, collective activities to create desired policies, using nontraditional reform methods and bring non-mainstream positions to mainstream society
\1 Civil society are voluntary organizations outside the government to aid identification and advancement of personal interests, encouraged in liberal democracies
\2 They can be either political or apolitical (not politically active), rather just to promote goals and interests, preventing tyranny of the majority
\2 It has been argued that globalization has led to cosmopolitanism, or a universal political order and civil society based on worldwide identity and values, found within international, political, nongovernmental organizations (NGOs)
\2 Authoritarian nations are against civil society, dividing purely based on social clevages
\3 Civil societies are often formed later through civil education of democratic rights, and through NGO involvement
\end{outline*}
\subsection{Political Institution}
\begin{outline*}
\1 Political institutions are structures which carry out governing, though they cannot be assumed to have the same powers in each nation
\1 Unitary systems have all policymaking centralized in one location, while federal systems divides between central government and sub-units, and confederal has power almost purely in sub-units, with a weak central
\2 Federal and confederal systems are criticized for inefficiency, due to local governments with possibly competing interests, such that very few governments are confederal for that reason
\1 
\end{outline*}
\subsection{Public Policy}
\section{Articles}
\begin{enumerate}
\item \hypertarget{1}{``The Clash of Civilizations?'' by Samuel Huntington}
\end{enumerate}

\end{document}
